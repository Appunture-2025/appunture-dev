% ----------------------------------------------------------
% Capítulo 6 - APRESENTAÇÃO DO SISTEMA
% ----------------------------------------------------------
\chapter{APRESENTAÇÃO DO SISTEMA} \label{cha:apresentacao}

Neste capítulo o \textit{Appunture} é apresentado, pontuando os aspectos técnicos e as funcionalidades que constituem a aplicação.

\section{ARQUITETURA DO SISTEMA} \label{sec:arq_sistema}

O \textit{Appunture} foi desenvolvido seguindo uma arquitetura híbrida e modular que integra funcionalidades \textit{offline} e \textit{online} para garantir alta disponibilidade, desempenho e usabilidade mesmo em ambientes com conexão limitada ou instável. A arquitetura se divide em dois módulos principais: aplicativo móvel (React Native/Expo) e \textit{backend} (Java + Spring Boot).

\subsection{Arquitetura Híbrida (Offline + Online)}

No núcleo do aplicativo móvel, o \textit{Appunture} utiliza um banco de dados local SQLite para armazenar informações do atlas, incluindo os 361 pontos dos meridianos principais, pontos extras, sintomas, relações terapêuticas, favoritos e anotações do usuário. Esta abordagem garante que o aplicativo funcione integralmente em modo \textit{offline}, permitindo consultas rápidas e uso em qualquer lugar, sem depender da conexão com a internet.

Quando a conexão estiver disponível, o sistema realiza sincronizações automáticas e seguras com um \textit{backend} remoto, desenvolvido em Java com Spring Boot, que hospeda o banco de dados Firestore centralizado. Esta sincronização atualiza dados do atlas, envia registros do usuário e mantém o histórico sempre preservado e atualizado.

\subsection{Comunicação e Sincronização}

A comunicação entre o app e o \textit{backend} é realizada via API RESTful, utilizando autenticação Firebase/JWT para controle seguro de acesso. O \textit{backend} disponibiliza \textit{endpoints} públicos para consulta de dados do atlas e \textit{endpoints} privados para operações que envolvem dados do usuário, como sincronização e autenticação.

A sincronização implementa um mecanismo de controle de versões dos dados do atlas para evitar \textit{downloads} desnecessários e garantir que o aplicativo trabalhe sempre com as informações mais recentes. O processo ocorre automaticamente em \textit{background} sempre que uma conexão de internet é detectada.

\subsection{Assistente de IA para Busca Inteligente}

Para viabilizar a busca inteligente por sintomas, o aplicativo utiliza um assistente de inteligência artificial integrado ao \textit{backend}. Esta funcionalidade requer conexão com a internet para processamento.

\textbf{Arquitetura RAG (Retrieval-Augmented Generation):}
O sistema utiliza a arquitetura RAG integrada ao \textit{backend} Java. O assistente inteligente, alimentado pelo Google Gemini 1.5 Flash via Spring AI, interpreta perguntas do usuário sobre sintomas e pontos de acupuntura, fornecendo respostas contextualizadas com a base de dados técnica do sistema. Esta abordagem reduz alucinações do modelo e aumenta a confiabilidade das informações apresentadas.

\textbf{Navegação Offline:}
Quando sem conexão, o usuário pode navegar manualmente pelo atlas anatômico e acessar as informações dos pontos já sincronizados no banco de dados local. A busca inteligente por sintomas fica indisponível até que uma conexão seja restabelecida.

\subsection{Interface Interativa e Navegação}

A interface do app é construída para promover uma experiência intuitiva e visualmente rica, com mapas anatômicos interativos baseados em SVG, que possibilitam a interação direta com os pontos de acupuntura. A navegação é estruturada para facilitar o acesso rápido às principais funcionalidades, como busca, visualização detalhada, favoritos e anotações, promovendo fluidez e usabilidade.

\subsection{Backend e Gestão de Dados}

O \textit{backend} centraliza a gestão dos dados oficiais, incluindo o cadastro e atualização dos pontos, sintomas e suas relações. A gestão do conteúdo é realizada através de APIs REST protegidas no \textit{backend}, com acesso administrativo gerenciado pelos consoles do Firebase e Google Cloud Platform.

\section{AUTENTICAÇÃO E SEGURANÇA} \label{sec:autenticacao}

Para garantir a segurança dos usuários e a integridade dos dados, o sistema implementa autenticação via Firebase Authentication, integrada ao Spring Security no \textit{backend}.

\textbf{Métodos de Autenticação:}
\begin{itemize}
    \item Login com email e senha;
    \item Login social via Google (\textit{Google Sign-In});
    \item Validação de \textit{tokens} JWT (Firebase ID Token) no \textit{backend}.
\end{itemize}

Durante o processo de login, o Firebase Authentication gerencia a validação das credenciais e gera um \textit{token} JWT assinado. Este \textit{token} é utilizado para manter a sessão do usuário e validar o acesso aos recursos protegidos da API.

A sessão é gerenciada pelo aplicativo móvel, que armazena o \textit{token} de forma segura no \textit{storage} local do dispositivo. Desta forma, o usuário permanece logado mesmo após fechar e reabrir o aplicativo. A cada interação que requer comunicação com o \textit{backend}, o \textit{token} é enviado via cabeçalho HTTP para validação e controle de acesso baseado em \textit{roles} (RBAC).

\section{FUNCIONALIDADES DO SISTEMA} \label{sec:funcionalidades}

\subsection{Tela Inicial e Onboarding}

Quando o usuário inicia a utilização do aplicativo pela primeira vez, é apresentada uma sequência de telas de apresentação (\textit{onboarding}) que explicam as principais funcionalidades e benefícios do \textit{Appunture}. Esta experiência introdutória inclui informações sobre o uso \textit{offline}, a busca inteligente e os mapas anatômicos interativos.

Na tela inicial, o usuário pode escolher entre realizar o processo de cadastro ou direcionar-se diretamente para a tela de login. O design foi pensado para transmitir profissionalismo e confiança, utilizando a identidade visual do projeto com cores que remetem à medicina tradicional chinesa e à modernidade tecnológica.

\subsection{Cadastro e Autenticação via Firebase}

O processo de autenticação do \textit{Appunture} é inteiramente gerenciado pelo \textit{Firebase Authentication}, serviço da Google que oferece segurança robusta e escalabilidade. A tela de cadastro foi projetada para ser simples e rápida, solicitando apenas as informações essenciais:

\begin{itemize}
    \item \textbf{Nome completo}: Armazenado no \textit{displayName} do perfil Firebase;
    \item \textbf{Email}: Identificador único do usuário no Firebase;
    \item \textbf{Senha}: Com validação de força mínima (6 caracteres);
    \item \textbf{Confirmação de senha}: Para evitar erros de digitação;
    \item \textbf{Aceite dos termos}: Concordância com os termos de uso e política de privacidade.
\end{itemize}

Ao submeter o formulário, o aplicativo utiliza a função \texttt{createUserWithEmailAndPassword} do Firebase SDK para criar a conta. Em seguida, o \texttt{displayName} é atualizado com o nome informado via \texttt{updateProfile}. Por fim, os dados do usuário são sincronizados com o \textit{backend} Java através de uma chamada à API REST.

O sistema também oferece autenticação via \textit{Google Sign-In}, implementada com o \textit{expo-auth-session}. Neste fluxo, o usuário autoriza o acesso via OAuth 2.0, e o \textit{token} retornado pelo Google é usado para criar uma credencial Firebase via \texttt{GoogleAuthProvider.credential}, completando a autenticação.

Após a autenticação (por qualquer método), o \textit{Firebase ID Token} é obtido e armazenado de forma segura no dispositivo. Este \textit{token} é enviado em todas as requisições ao \textit{backend} para validação e autorização.

Todos os usuários têm acesso igualitário a todas as funcionalidades do atlas educativo, independentemente do método de cadastro utilizado.

\subsection{Tela Principal e Navegação}

A tela principal do aplicativo apresenta as principais funcionalidades de forma organizada e acessível. A navegação é estruturada por abas (\textit{tabs}) na parte inferior da tela, permitindo acesso rápido às seções principais:

\begin{itemize}
    \item \textbf{Início}: Tela inicial com acesso rápido às funcionalidades principais;
    \item \textbf{Buscar}: Sistema de busca por nome de ponto ou código;
    \item \textbf{Sintomas}: Lista de sintomas com pontos relacionados;
    \item \textbf{Meridianos}: Navegação por meridianos com mapas SVG;
    \item \textbf{Assistente}: Chat com IA para consultas contextualizadas;
    \item \textbf{Favoritos}: Lista de pontos favoritados pelo usuário;
    \item \textbf{Perfil}: Configurações do usuário e preferências.
\end{itemize}

\subsection{Mapa Anatômico Interativo}

Uma das funcionalidades mais destacadas do \textit{Appunture} é o mapa anatômico interativo, implementado com tecnologia SVG (\textit{Scalable Vector Graphics}). Este recurso permite que o usuário visualize o corpo humano de forma detalhada e interaja diretamente com os pontos de acupuntura mapeados.

Cada ponto é representado por um marcador visual que, ao ser tocado, exibe informações básicas sobre o ponto em um \textit{tooltip}. O usuário pode então optar por acessar a tela de detalhes completa do ponto selecionado.

O mapa oferece 15 visualizações vetoriais de alta fidelidade, organizadas por meridianos:

\begin{itemize}
    \item 12 meridianos principais: Pulmão (LU), Intestino Grosso (LI), Estômago (ST, ST2), Baço (SP), Coração (HT), Intestino Delgado (SI), Bexiga (BL), Rim (KI), Pericárdio (P), Triplo Aquecedor (TW), Vesícula Biliar (GB) e Fígado (LV);
    \item 2 vasos extraordinários: Vaso Governador (GV/Du Mai) e Vaso Concepção (CV/Ren Mai).
\end{itemize}

Cada mapa SVG é carregado sob demanda, permitindo navegação fluida entre as diferentes visualizações de frente e costas do corpo.

\subsection{Detalhes dos Pontos de Acupuntura}

A tela de detalhes apresenta informações completas sobre cada ponto de acupuntura, organizadas em seções para facilitar a consulta:

\begin{itemize}
    \item \textbf{Identificação}: Nome em português, nome em chinês (pinyin), código internacional;
    \item \textbf{Localização}: Descrição anatômica precisa da localização do ponto;
    \item \textbf{Meridiano}: Identificação do meridiano ao qual o ponto pertence;
    \item \textbf{Indicações}: Lista de condições clínicas para as quais o ponto é indicado;
    \item \textbf{Contraindicações}: Alertas sobre situações em que o ponto não deve ser utilizado;
    \item \textbf{Funções}: Descrição das funções terapêuticas do ponto.
\end{itemize}

\subsection{Sistema de Busca Inteligente}

O sistema de busca do \textit{Appunture} utiliza inteligência artificial para oferecer resultados relevantes mesmo quando o usuário não conhece o nome exato do ponto que procura. A busca pode ser realizada por:

\begin{itemize}
    \item Nome do ponto (em português ou pinyin);
    \item Localização anatômica;
    \item Sintomas ou condições clínicas;
    \item Meridiano;
    \item Funções terapêuticas.
\end{itemize}

O assistente de IA interpreta a intenção do usuário e retorna os pontos mais relevantes, mesmo com erros de digitação ou variações na escrita. Esta funcionalidade requer conexão com a internet.

\subsection{Assistente Inteligente com IA Generativa}

O assistente inteligente integrado ao \textit{Appunture} permite que o usuário descreva sintomas em linguagem natural e receba informações educativas sobre pontos de acupuntura. A funcionalidade utiliza a arquitetura RAG (\textit{Retrieval-Augmented Generation}):

\textbf{Funcionamento do Sistema:}
\begin{enumerate}
    \item O sistema intercepta a pergunta do usuário;
    \item Busca contexto relevante na base de dados (pontos e sintomas relacionados);
    \item Envia um \textit{prompt} enriquecido para o modelo Google Gemini 1.5 Flash via Spring AI;
    \item Retorna respostas contextualizadas, baseadas nos dados validados do sistema.
\end{enumerate}

\textbf{Benefícios da Abordagem:}
\begin{itemize}
    \item Respostas contextualizadas com a base de dados técnica;
    \item Redução significativa de alucinações do modelo através da abordagem RAG;
    \item Maior confiabilidade das informações apresentadas;
    \item Rastreabilidade das fontes utilizadas nas respostas.
\end{itemize}

\subsection{Sistema de Favoritos e Anotações Pessoais}

O \textit{Appunture} permite que o usuário salve pontos de acupuntura como favoritos para acesso rápido posterior. Além disso, é possível adicionar anotações pessoais a cada ponto, registrando observações de estudo, experiências ou lembretes.

Estas informações são armazenadas localmente no dispositivo e sincronizadas com o \textit{backend} quando há conexão disponível, garantindo que o usuário não perca seus dados ao trocar de dispositivo.



\section{GESTÃO DE DADOS} \label{sec:admin}

A gestão do conteúdo do \textit{Appunture} é realizada através de duas abordagens complementares:

\textbf{APIs REST Protegidas:} O \textit{backend} disponibiliza \textit{endpoints} para operações administrativas que podem ser consumidos por ferramentas externas ou futuras interfaces. A documentação completa está disponível via Swagger/OpenAPI no \textit{endpoint} \texttt{/swagger-ui.html}.

\textbf{Consoles Administrativos:} A gestão operacional dos dados é realizada diretamente através dos consoles do Firebase (Firestore, Authentication, Storage) e Google Cloud Platform, que oferecem interfaces completas para:

\begin{itemize}
    \item Visualização e edição de documentos no Firestore;
    \item Gerenciamento de usuários e permissões no Firebase Authentication;
    \item \textit{Upload} e organização de imagens no Firebase Storage;
    \item Monitoramento de logs e métricas no Google Cloud Console.
\end{itemize}

\textbf{Point Mapper:} Para o mapeamento das coordenadas dos pontos sobre as imagens SVG do atlas, foi desenvolvida uma ferramenta web auxiliar que permite definir visualmente a posição de cada ponto e exportar os dados em formato JSON.
\section{EXPERIÊNCIA DO USUÁRIO E USABILIDADE} \label{sec:ux}

O desenvolvimento da interface do \textit{Appunture} seguiu princípios de design centrado no usuário, priorizando:

\begin{itemize}
    \item \textbf{Simplicidade}: Interface limpa e intuitiva, sem excesso de informações;
    \item \textbf{Consistência}: Padrões visuais e de interação consistentes em todo o aplicativo;
    \item \textbf{Feedback}: Respostas visuais claras para todas as ações do usuário;
    \item \textbf{Acessibilidade}: Suporte a diferentes tamanhos de tela e orientações;
    \item \textbf{Performance}: Carregamento rápido e transições fluidas.
\end{itemize}

\section{TELAS DO SISTEMA} \label{sec:telas}

Esta seção apresenta as principais telas do aplicativo \textit{Appunture} implementado em React Native. As capturas de tela demonstram a interface final do sistema, seguindo os princípios de design centrado no usuário descritos anteriormente.

\subsection{Tela de Login}

A tela de login é o ponto de entrada do aplicativo para usuários que já possuem conta. A interface apresenta um design limpo e profissional, com os seguintes elementos:

\begin{itemize}
    \item \textbf{Logo e identidade visual}: O logotipo do \textit{Appunture} é exibido no topo, reforçando a identidade da marca;
    \item \textbf{Campos de autenticação}: Email e senha com validação em tempo real;
    \item \textbf{Login social}: Botão para autenticação via Google, integrado ao Firebase Authentication;
    \item \textbf{Modo visitante}: Opção ``Continuar como visitante'' que permite explorar o aplicativo sem criar conta, com acesso limitado às funcionalidades que requerem internet;
    \item \textbf{Links auxiliares}: Acesso rápido para cadastro e recuperação de senha.
\end{itemize}

\figura{TELA DE LOGIN}{0.35}{fig/tela-login.png}{Os autores (2025)}{tela-login}{}{}

\subsection{Tela de Cadastro}

O formulário de cadastro foi projetado para ser simples e rápido, solicitando apenas as informações essenciais para criação da conta:

\begin{itemize}
    \item \textbf{Nome completo}: Para personalização da experiência;
    \item \textbf{Email}: Utilizado como identificador único da conta;
    \item \textbf{Senha}: Com indicador visual de força e requisitos mínimos de segurança;
    \item \textbf{Confirmação de senha}: Para evitar erros de digitação;
    \item \textbf{Termos de uso}: Checkbox de aceite obrigatório.
\end{itemize}

O cadastro também pode ser realizado via Google Sign-In para maior conveniência.

\figura{TELA DE CADASTRO}{0.35}{fig/tela-cadastro.png}{Os autores (2025)}{tela-cadastro}{}{}

\subsection{Tela Principal (Home)}

A tela principal é o hub central do aplicativo, apresentando uma visão geral das funcionalidades e informações relevantes ao usuário:

\begin{itemize}
    \item \textbf{Saudação personalizada}: Exibe o nome do usuário e status de sincronização;
    \item \textbf{Indicador de conectividade}: Mostra se o aplicativo está online ou offline;
    \item \textbf{Acesso rápido}: Cards para as principais funcionalidades (Buscar Pontos, Mapa Corporal, Assistente IA, Favoritos);
    \item \textbf{Estatísticas}: Quantidade de pontos disponíveis e meridianos cadastrados;
    \item \textbf{Pontos recentes}: Lista dos últimos pontos visualizados para acesso rápido.
\end{itemize}

A navegação principal é feita através da barra de abas na parte inferior da tela, seguindo o padrão de aplicativos móveis modernos.

\figura{TELA PRINCIPAL (HOME)}{0.35}{fig/tela-home.png}{Os autores (2025)}{tela-home}{}{}

\subsection{Tela de Busca}

A funcionalidade de busca permite encontrar pontos de acupuntura através de consultas em linguagem natural, processadas pela inteligência artificial:

\begin{itemize}
    \item \textbf{Campo de busca}: Aceita termos como nome do ponto, código, meridiano ou sintomas;
    \item \textbf{Resultados em tempo real}: A lista é atualizada conforme o usuário digita;
    \item \textbf{Cards de resultado}: Cada ponto é exibido com código, nome, meridiano e indicações principais;
    \item \textbf{Destaque de termos}: Os termos buscados são destacados nos resultados;
    \item \textbf{Ação de favoritar}: Botão para adicionar/remover dos favoritos diretamente na lista.
\end{itemize}

A busca utiliza o modelo Gemini 1.5 Flash via RAG (\textit{Retrieval-Augmented Generation}) para encontrar pontos relevantes mesmo quando o usuário descreve sintomas em vez de usar termos técnicos.

\figura{TELA DE BUSCA}{0.35}{fig/tela-busca.png}{Os autores (2025)}{tela-busca}{}{}

\subsection{Tela de Meridianos}

Esta tela apresenta os 12 meridianos principais da medicina tradicional chinesa em um layout de grade visual e intuitivo:

\begin{itemize}
    \item \textbf{Cards coloridos}: Cada meridiano possui uma cor característica baseada no seu elemento (Metal, Terra, Fogo, Água, Madeira);
    \item \textbf{Código e contagem}: Abreviação internacional (LU, LI, ST, etc.) e quantidade de pontos;
    \item \textbf{Nome e caracteres}: Nome em português e caracteres chineses para cada meridiano;
    \item \textbf{Elemento e horário}: Informações sobre o elemento Wu Xing e horário de pico de energia;
    \item \textbf{Órgão relacionado}: Sistema orgânico associado ao meridiano.
\end{itemize}

Ao tocar em um card, o usuário é direcionado para a lista de pontos daquele meridiano específico.

\figura{TELA DE MERIDIANOS}{0.35}{fig/tela-meridianos.png}{Os autores (2025)}{tela-meridianos}{}{}

\subsection{Mapa Anatômico Interativo}

O mapa corporal é uma das funcionalidades mais destacadas do \textit{Appunture}, permitindo visualização espacial dos pontos de acupuntura:

\begin{itemize}
    \item \textbf{Visualização SVG}: Imagens vetoriais de alta qualidade baseadas no atlas oficial da OMS;
    \item \textbf{Alternância de vistas}: Botões para alternar entre vista frontal e posterior;
    \item \textbf{Navegação por camadas}: Controles para navegar entre as 15 camadas anatômicas disponíveis (8 frontais e 7 posteriores);
    \item \textbf{Marcadores interativos}: Pontos clicáveis sobre a imagem anatômica;
    \item \textbf{Lista de pontos visíveis}: Seção inferior listando os pontos presentes na camada atual.
\end{itemize}

Ao tocar em um marcador, o usuário é direcionado para a tela de detalhes do ponto correspondente.

\figura{MAPA ANATÔMICO INTERATIVO}{0.35}{fig/tela-mapa.png}{Os autores (2025)}{tela-mapa}{}{}

\subsection{Detalhes do Ponto de Acupuntura}

A tela de detalhes apresenta informações completas sobre cada ponto, organizadas em seções para facilitar a consulta:

\begin{itemize}
    \item \textbf{Galeria de imagens}: Carrossel com fotos ilustrativas do ponto e sua localização;
    \item \textbf{Identificação}: Código internacional, nome em pinyin e caracteres chineses;
    \item \textbf{Meridiano}: Indicação do meridiano ao qual o ponto pertence;
    \item \textbf{Localização anatômica}: Descrição detalhada de como encontrar o ponto;
    \item \textbf{Anatomia}: Informações sobre inervação e vascularização;
    \item \textbf{Indicações}: Lista das condições tratadas pelo ponto;
    \item \textbf{Função terapêutica}: Descrição dos efeitos e ações do ponto;
    \item \textbf{Técnicas aplicáveis}: Métodos de estimulação recomendados;
    \item \textbf{Precauções}: Alertas sobre contraindicações quando aplicável.
\end{itemize}

O botão de favorito no cabeçalho permite salvar o ponto para acesso rápido posterior.

\figura{DETALHES DO PONTO DE ACUPUNTURA}{0.35}{fig/tela-detalhes.png}{Os autores (2025)}{tela-detalhes}{}{}

\subsection{Assistente de Inteligência Artificial}

O chatbot com IA é a interface conversacional do sistema, permitindo consultas em linguagem natural:

\begin{itemize}
    \item \textbf{Interface de chat}: Design familiar de aplicativos de mensagens;
    \item \textbf{Mensagens do usuário}: Exibidas à direita em cor primária;
    \item \textbf{Respostas da IA}: Exibidas à esquerda com suporte a formatação Markdown;
    \item \textbf{Sugestões de pontos}: Cards clicáveis dentro das respostas para acesso direto aos detalhes;
    \item \textbf{Indicador de digitação}: Feedback visual enquanto a IA processa a resposta;
    \item \textbf{Histórico da conversa}: Mantido durante a sessão para contexto.
\end{itemize}

O assistente utiliza a arquitetura RAG com o modelo Gemini 1.5 Flash, combinando o conhecimento do atlas de pontos com a capacidade de compreensão de linguagem natural para fornecer recomendações precisas baseadas nos sintomas descritos.

\figura{ASSISTENTE DE INTELIGÊNCIA ARTIFICIAL}{0.35}{fig/tela-chat.png}{Os autores (2025)}{tela-chat}{}{}

\subsection{Tela de Favoritos}

A funcionalidade de favoritos permite ao usuário criar uma lista personalizada de pontos para consulta rápida:

\begin{itemize}
    \item \textbf{Lista personalizada}: Exibe apenas os pontos marcados como favoritos pelo usuário;
    \item \textbf{Contagem}: Indica o número total de pontos favoritados;
    \item \textbf{Informações completas}: Cada card exibe código, nome, meridiano e indicações;
    \item \textbf{Gestão rápida}: Botão para remover da lista de favoritos;
    \item \textbf{Acesso aos detalhes}: Toque no card para visualizar informações completas.
\end{itemize}

Os favoritos são sincronizados com a nuvem quando o usuário está autenticado, permitindo acesso em múltiplos dispositivos.

\figura{TELA DE FAVORITOS}{0.35}{fig/tela-favoritos.png}{Os autores (2025)}{tela-favoritos}{}{}
