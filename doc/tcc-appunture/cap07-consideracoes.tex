% ----------------------------------------------------------
% Capítulo 7 - CONSIDERAÇÕES FINAIS
% ----------------------------------------------------------
\chapter{CONSIDERAÇÕES FINAIS} \label{cha:consideracoes}

O desenvolvimento do aplicativo \textit{Appunture} representou um desafio multidisciplinar que envolveu conhecimentos de engenharia de software, design de interfaces, inteligência artificial e compreensão do domínio da acupuntura e medicina tradicional chinesa. O resultado é uma ferramenta digital robusta, acessível e funcional, capaz de contribuir significativamente para a formação e prática dos profissionais da área.

O \textit{Appunture} representa um avanço significativo na tecnologia aplicada à acupuntura, unindo a tradição milenar da MTC com tecnologias de ponta como Inteligência Artificial Generativa (Spring AI + Google Gemini) e arquiteturas móveis modernas.

Os principais objetivos propostos no início do projeto foram alcançados:

\begin{itemize}
    \item A implementação de um sistema multiplataforma (Android, iOS e Web) garante amplo acesso à ferramenta;
    \item A arquitetura \textit{offline-first} permite o uso do aplicativo mesmo em ambientes sem conectividade;
    \item O atlas anatômico digital com 15 visualizações vetoriais (SVG) oferece uma experiência visual rica e intuitiva para localização dos pontos;
    \item O assistente inteligente com RAG (Retrieval-Augmented Generation) fornece respostas clinicamente precisas baseadas na base de dados validada;
    \item A diferenciação de perfis de usuário respeita os limites éticos e legais de cada categoria profissional;
    \item O painel administrativo permite a manutenção e atualização contínua do conteúdo e da base de conhecimento que alimenta a IA;
    \item A ferramenta Point Mapper garantiu precisão milimétrica no mapeamento dos 361 pontos de acupuntura.
\end{itemize}

A adoção da metodologia Scrum permitiu um desenvolvimento iterativo e incremental, com entregas frequentes e ajustes baseados em \textit{feedback} contínuo. A escolha das tecnologias --- React Native com Expo para o mobile, Java com Spring Boot para o \textit{backend}, e React para o painel administrativo --- mostrou-se adequada para os requisitos do projeto, oferecendo robustez, performance e facilidade de manutenção.

O \textit{Appunture} surge em um momento oportuno, dado o crescimento das práticas integrativas e complementares no Brasil e a necessidade de ferramentas que auxiliem na padronização e segurança da prática da acupuntura. Com mais de 370 mil acupunturistas estimados no país e a perspectiva de regulamentação da profissão, o aplicativo pode desempenhar um papel importante na educação continuada e no suporte à prática clínica.

Como trabalhos futuros, sugerem-se:

\begin{itemize}
    \item Implementação de recursos de realidade aumentada para visualização dos pontos em modelos 3D;
    \item Integração com sistemas de prontuário eletrônico;
    \item Desenvolvimento de módulos de avaliação e certificação;
    \item Expansão da base de conhecimento para incluir outras técnicas da MTC;
    \item Implementação de recursos de acessibilidade avançados;
    \item Tradução do conteúdo para outros idiomas;
    \item Aprimoramento contínuo do assistente IA com novos modelos e técnicas de RAG;
    \item Integração com \textit{wearables} para monitoramento de tratamentos.
\end{itemize}

Em síntese, o \textit{Appunture} representa uma contribuição significativa para o ecossistema de ferramentas digitais voltadas à saúde, demonstrando que é possível desenvolver soluções tecnológicas que aliem qualidade técnica, usabilidade, inteligência artificial e responsabilidade social. A solução atende à demanda por ferramentas confiáveis, seguras e acessíveis, contribuindo para a qualificação profissional e segurança do paciente no Brasil. O aplicativo está preparado para evoluir e se adaptar às novas demandas do mercado, mantendo seu compromisso com a segurança, a acessibilidade e a qualidade da informação.
