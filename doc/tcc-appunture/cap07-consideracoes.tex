% ----------------------------------------------------------
% Capítulo 7 - CONSIDERAÇÕES FINAIS
% ----------------------------------------------------------
\chapter{CONSIDERAÇÕES FINAIS} \label{cha:consideracoes}

O desenvolvimento do \textit{Appunture} representou um desafio multidisciplinar que envolveu conhecimentos de engenharia de software, design de interfaces e compreensão básica do domínio da acupuntura. O resultado é uma ferramenta educativa que busca facilitar o acesso a informações sobre os pontos de acupuntura, servindo como recurso complementar de apoio ao estudo.

Os principais objetivos propostos no início do projeto foram alcançados:

\begin{itemize}
    \item A implementação de um sistema multiplataforma (Android, iOS e Web) permite o acesso à ferramenta em diferentes dispositivos;
    \item A arquitetura \textit{offline-first} permite o uso do aplicativo mesmo em ambientes sem conectividade;
    \item O atlas anatômico digital com 15 visualizações vetoriais (SVG) auxilia na visualização dos pontos de acupuntura;
    \item O assistente baseado em IA auxilia na busca e correlação entre sintomas e pontos;
    \item O painel administrativo permite a atualização dos conteúdos do sistema;
    \item A ferramenta de mapeamento auxiliou no posicionamento das coordenadas dos pontos sobre as imagens.
\end{itemize}

A adoção da metodologia Scrum permitiu um desenvolvimento iterativo e incremental, com entregas frequentes e ajustes baseados em \textit{feedback} contínuo. A escolha das tecnologias --- React Native com Expo para o mobile, Java com Spring Boot para o \textit{backend}, e React para o painel administrativo --- mostrou-se adequada para os requisitos do projeto, oferecendo robustez, performance e facilidade de manutenção.

O \textit{Appunture} surge em um contexto de crescimento das práticas integrativas e complementares no Brasil. A ferramenta busca contribuir como um recurso de apoio ao estudo, facilitando o acesso a informações organizadas sobre acupuntura. É importante ressaltar que o aplicativo não substitui a formação profissional adequada, mas pode servir como material complementar de consulta.

Como trabalhos futuros, sugerem-se:

\begin{itemize}
    \item Implementação de recursos de realidade aumentada para visualização dos pontos em modelos 3D;
    \item Integração com sistemas de prontuário eletrônico;
    \item Desenvolvimento de módulos de avaliação e certificação;
    \item Expansão da base de conhecimento para incluir outras técnicas da MTC;
    \item Implementação de recursos de acessibilidade avançados;
    \item Tradução do conteúdo para outros idiomas;
    \item Aprimoramento contínuo do assistente IA com novos modelos e técnicas de RAG;
    \item Integração com \textit{wearables} para monitoramento de tratamentos.
\end{itemize}

Em síntese, o \textit{Appunture} representa uma contribuição para o ecossistema de ferramentas educativas voltadas à acupuntura. A plataforma busca facilitar o acesso a informações organizadas e servir como recurso complementar de apoio ao estudo. Espera-se que a ferramenta possa evoluir com base no feedback dos usuários, mantendo seu foco na acessibilidade e na qualidade da informação disponibilizada.
