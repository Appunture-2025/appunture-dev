% ----------------------------------------------------------
% Capítulo 7 - CONSIDERAÇÕES FINAIS
% ----------------------------------------------------------
\chapter{CONSIDERAÇÕES FINAIS} \label{cha:consideracoes}

O desenvolvimento do \textit{Appunture} representou um desafio multidisciplinar que envolveu conhecimentos de engenharia de software, design de interfaces e compreensão básica do domínio da acupuntura. O resultado é uma ferramenta educativa que busca facilitar o acesso a informações sobre os pontos de acupuntura, servindo como recurso complementar de apoio ao estudo.

Os principais objetivos propostos no início do projeto foram alcançados:

\begin{itemize}
    \item A implementação do aplicativo Android com React Native 0.79.6 e Expo SDK 53 permite o acesso à ferramenta em dispositivos móveis com alta performance;
    \item A arquitetura \textit{offline-first} com banco de dados SQLite local permite o uso do aplicativo mesmo em ambientes sem conectividade, com sincronização automática quando online;
    \item O atlas anatômico digital com 15 visualizações vetoriais (SVG) organizadas por meridianos auxilia na visualização precisa dos pontos de acupuntura;
    \item O sistema de busca inteligente e o assistente baseado em IA (arquitetura RAG com Google Gemini) auxiliam na correlação entre sintomas e pontos;
    \item O \textit{backend} Java com Spring Boot 3.2.5 e Google Cloud Firestore fornece APIs REST robustas para gestão de dados e autenticação via Firebase;
    \item O sistema de favoritos com sincronização na nuvem permite que usuários salvem e acessem seus pontos preferidos em múltiplos dispositivos;
    \item A ferramenta Point Mapper auxiliou no mapeamento preciso das coordenadas dos pontos sobre as imagens SVG do atlas.
\end{itemize}

A adoção da metodologia Scrum permitiu um desenvolvimento iterativo e incremental, com entregas frequentes e ajustes baseados em \textit{feedback} contínuo. O gerenciamento de estado com Zustand, combinado ao sistema de persistência via AsyncStorage e SecureStore, garantiu uma experiência fluida e segura para os usuários.

A escolha do \textit{stack} tecnológico --- React Native com Expo para o aplicativo móvel e Java com Spring Boot para o \textit{backend} --- mostrou-se adequada para os requisitos do projeto, oferecendo robustez, performance e facilidade de manutenção. A integração com Firebase Authentication proporcionou autenticação segura com suporte a login tradicional e social (Google).

O \textit{Appunture} surge em um contexto de crescimento das práticas integrativas e complementares no Brasil. A ferramenta busca contribuir como um recurso de apoio ao estudo, facilitando o acesso a informações organizadas sobre acupuntura. É importante ressaltar que o aplicativo não substitui a formação profissional adequada, mas pode servir como material complementar de consulta.

Como trabalhos futuros, sugerem-se:

\begin{itemize}
    \item Implementação de recursos de realidade aumentada para visualização dos pontos em modelos 3D;
    \item Integração com sistemas de prontuário eletrônico;
    \item Desenvolvimento de módulos de avaliação e certificação para verificar o aprendizado;
    \item Expansão da base de conhecimento para incluir outras técnicas da MTC, como auriculoterapia e moxabustão;
    \item Implementação de recursos de acessibilidade avançados, incluindo suporte a leitores de tela;
    \item Tradução do conteúdo para outros idiomas, ampliando o alcance da ferramenta;
    \item Aprimoramento contínuo do assistente IA com novos modelos e técnicas de RAG para respostas mais precisas;
    \item Integração com \textit{wearables} para monitoramento e acompanhamento de tratamentos;
    \item Desenvolvimento de versão para iOS e distribuição na App Store.
\end{itemize}

Em síntese, o \textit{Appunture} representa uma contribuição para o ecossistema de ferramentas educativas voltadas à acupuntura. A plataforma busca facilitar o acesso a informações organizadas e servir como recurso complementar de apoio ao estudo. Com uma base técnica sólida e arquitetura preparada para expansões, espera-se que a ferramenta possa evoluir continuamente com base no feedback dos usuários, mantendo seu foco na acessibilidade e na qualidade da informação disponibilizada.
