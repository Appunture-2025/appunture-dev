% Pacotes básicos 
% ----------------------------------------------------------
\usepackage[T1]{fontenc}		% Selecao de codigos de fonte.
\usepackage[utf8]{inputenc}		% Codificacao do documento (conversão automática dos acentos)
\usepackage{lastpage}			% Usado pela Ficha catalográfica
\usepackage{indentfirst}		% Indenta o primeiro parágrafo de cada seção.

\usepackage{ifthen}		    	% para montar condicionais
\usepackage[brazil]{babel}		% para utilizar termos em portugues
\usepackage[final]{pdfpages}    % para incluir páginas de arquivos pdf
\usepackage{amsmath}			% simbolos matematicos

% Pacote para listagens de código
% ----------------------------------------------------------
\usepackage{listings}
\usepackage{xcolor}

% Configuração do listings para código
\lstset{
    basicstyle=\ttfamily\footnotesize,
    breaklines=true,
    frame=single,
    numbers=left,
    numberstyle=\tiny\color{gray},
    keywordstyle=\color{blue},
    commentstyle=\color{green!60!black},
    stringstyle=\color{orange},
    showstringspaces=false,
    tabsize=2,
    inputencoding=utf8,
    extendedchars=true,
    literate={á}{{\'a}}1 {é}{{\'e}}1 {í}{{\'i}}1 {ó}{{\'o}}1 {ú}{{\'u}}1
             {Á}{{\'A}}1 {É}{{\'E}}1 {Í}{{\'I}}1 {Ó}{{\'O}}1 {Ú}{{\'U}}1
             {ã}{{\~a}}1 {õ}{{\~o}}1 {Ã}{{\~A}}1 {Õ}{{\~O}}1
             {ç}{{\c{c}}}1 {Ç}{{\c{C}}}1
             {à}{{\`a}}1 {À}{{\`A}}1
}

% Linguagem JSON para listings
\lstdefinelanguage{json}{
    basicstyle=\ttfamily\footnotesize,
    morestring=[b]",
    stringstyle=\color{orange},
}

% Ambiente quadro (similar a tabela)
% ----------------------------------------------------------
\usepackage{float}
\newfloat{quadro}{htbp}{loq}[chapter]
\floatname{quadro}{Quadro}
\newcommand{\listofquadrosname}{Lista de Quadros}
\newlistof{listofquadros}{loq}{\listofquadrosname}
\newlistentry{quadro}{loq}{0}
\renewcommand{\cftquadroname}{QUADRO\space}
\renewcommand*{\cftquadroaftersnum}{\hfill--\hfill}

% ===========================================================
% CITAÇÕES - Usando abntex2cite (NATIVO do abnTeX2)
% ===========================================================
% [alf] = estilo autor-data (alfabético)
% [num] = estilo numérico
% Comandos disponíveis: \cite{}, \citeonline{}, \citeauthor{}, \citeyear{}
\usepackage[alf]{abntex2cite}

% Formatando o avanço dos títulos no sumário 
% ----------------------------------------------------------
\makeatletter
	\pretocmd{\chapter}{\addtocontents{toc}{\protect\addvspace{-12\p@}}}{}{}
	\pretocmd{\section}{\addtocontents{toc}{\protect\addvspace{-3\p@}}}{}{}
\makeatother

% https://groups.google.com/g/abntex2/c/ZYwE4t9uTFM
\makeatletter
\let\oldcontentsline\contentsline
\def\contentsline#1#2{%
	\expandafter\ifx\csname l@#1\endcsname\l@section
	\expandafter\@firstoftwo
	\else
	\expandafter\@secondoftwo
	\fi
	{%
		\oldcontentsline{#1}{\MakeTextUppercase{#2}}%
	}{%
		\oldcontentsline{#1}{#2}%
	}%
}
\makeatother

% Para ajustar o tamanho da fonte do número da primeira página do capítulo
% ----------------------------------------------------------
\makepagestyle{chapfirst}
\makeoddhead{chapfirst}{}{}{\footnotesize{\thepage}}

% Criar um novo estilo de cabeçalhos e rodapés
\makepagestyle{simplestextual}
  \makeevenhead{simplestextual}{}{}{\footnotesize \thepage}
  \makeoddhead{simplestextual}{}{}{\footnotesize \thepage}
  \makeevenfoot{simplestextual}{}{}{}
  \makeoddfoot{simplestextual}{}{}{}

