% ----------------------------------------------------------
% Capítulo 1 - INTRODUÇÃO
% ----------------------------------------------------------
\chapter{INTRODUÇÃO} \label{cha:introducao}

A prática da acupuntura no Brasil, embora cada vez mais difundida e procurada tanto no sistema público quanto no privado de saúde, ainda enfrenta desafios significativos no que diz respeito à regulamentação profissional, padronização de condutas, segurança técnica e formação dos praticantes. Trata-se de uma técnica milenar baseada na Medicina Tradicional Chinesa (MTC), que se expandiu no país por meio de cursos livres, técnicos e pós-graduações voltadas a diferentes profissionais da área da saúde. No entanto, a ausência de regulamentação federal clara e a disputa entre conselhos profissionais sobre quem pode exercer legalmente a acupuntura têm gerado um ambiente de insegurança jurídica, afetando não apenas os acupunturistas, mas também o acesso qualificado da população à prática.

Além das disputas institucionais, o cenário atual é marcado pela fiscalização fragmentada, formação heterogênea e riscos crescentes à biossegurança e à saúde do paciente, especialmente em ambientes informais. Segundo diretrizes da Organização Mundial da Saúde (OMS), o uso inadequado de agulhas, a falta de higienização correta ou a aplicação indevida de técnicas podem resultar em infecções, lesões internas e outros agravos que comprometem gravemente a integridade dos usuários da acupuntura \cite{oms1999}.

Outro desafio relevante é o caráter subjetivo e complexo do diagnóstico energético, próprio da MTC. O raciocínio clínico baseado em padrões de desarmonia (como deficiência de Qi, estagnação de sangue, entre outros) exige extenso treinamento e sensibilidade clínica. A falta de padronização e apoio didático dificulta a atuação de profissionais iniciantes ou com formação limitada, podendo comprometer a eficácia terapêutica.

Diante desse contexto, o uso de Tecnologias Digitais de Informação e Comunicação (TDIC) apresenta-se como uma ferramenta estratégica para apoiar a qualificação da prática da acupuntura. Soluções como aplicativos educativos e interativos podem auxiliar na visualização de pontos, organização de informações terapêuticas, revisão diagnóstica e adequação de conteúdos conforme o nível de formação do usuário, promovendo maior segurança, padronização e confiabilidade.

Neste cenário, propõe-se o desenvolvimento do \textit{Appunture}, uma plataforma educativa multiplataforma (Android, iOS e Web) que funciona como um atlas digital de acupuntura, voltada a estudantes e profissionais interessados na área. A ferramenta oferece acesso \textit{offline} a informações sobre os 361 pontos dos meridianos principais, pontos extras, e um atlas anatômico composto por 15 visualizações vetoriais (SVG), organizadas por meridianos e vistas (frontal e posterior). O aplicativo também disponibiliza um assistente baseado em Inteligência Artificial para auxiliar na correlação entre sintomas e pontos.

O \textit{Appunture} foi desenvolvido com arquitetura ``\textit{offline-first}'', permitindo o acesso aos dados mesmo sem conexão à internet, com sincronização automática quando online. Para auxiliar no posicionamento dos pontos sobre as imagens anatômicas, foi desenvolvida uma ferramenta auxiliar de mapeamento (Point Mapper). A gestão dos conteúdos do sistema é realizada através das APIs REST do \textit{backend} e dos consoles administrativos do Firebase e Google Cloud.

É importante ressaltar que o \textit{Appunture} não substitui a formação profissional adequada nem o acompanhamento de profissionais qualificados. A ferramenta busca ser um recurso complementar de apoio ao estudo e à consulta de informações sobre acupuntura, facilitando o acesso a informações organizadas e padronizadas.

\section{OBJETIVOS} \label{sec:objetivos}

O principal objetivo deste projeto é desenvolver o \textit{Appunture}, uma plataforma digital educativa voltada ao estudo da acupuntura. A ferramenta busca facilitar o acesso a informações sobre os pontos de acupuntura, oferecendo recursos visuais e de busca que possam auxiliar estudantes e profissionais em seu processo de aprendizagem e consulta.

\subsection{Objetivos Específicos}

Entre os objetivos específicos, destacam-se:

\begin{itemize}
    \item \textbf{Visualização dos Pontos de Acupuntura}: Disponibilizar um atlas anatômico digital com 15 visualizações vetoriais (SVG), cobrindo vistas frontais e posteriores dos meridianos principais.
    
    \item \textbf{Assistente de Consulta}: Implementar um módulo baseado em IA para auxiliar na busca e correlação entre sintomas e pontos, utilizando os dados cadastrados no sistema.
    
    \item \textbf{Apoio ao Estudo}: Disponibilizar conteúdos organizados para auxiliar estudantes e profissionais no processo de aprendizagem.
    
    \item \textbf{Organização de Informações}: Disponibilizar informações sobre os 361 pontos clássicos e pontos extras, incluindo localização, indicações e funções.
    
    \item \textbf{Acesso Offline}: Permitir o acesso às informações mesmo sem conexão à internet.
    
    \item \textbf{Ferramenta de Mapeamento}: Desenvolver uma ferramenta auxiliar para posicionamento das coordenadas dos pontos sobre as imagens anatômicas.
    
    \item \textbf{Painel de Atualização}: Disponibilizar uma interface para atualização dos conteúdos do sistema.
\end{itemize}

Dessa forma, o \textit{Appunture} propõe-se como um recurso digital complementar de apoio ao estudo da acupuntura, buscando facilitar o acesso a informações organizadas e contribuir para o processo de aprendizagem dos usuários interessados na área.
