%%
%% Trabalho de Conclusão de Curso - APPUNTURE
%% Baseado no modelo ufpr-abntex2
%% Universidade Federal do Paraná
%%
%% Autores: Bruno Brugnerotto de Lara, Gabriel Francelino Voidaleski, Pedro Henrique Lopes
%% Orientador: Prof. Dr. Paulo Sobreira Moraes
%%
%% 2025
%%

\documentclass[
        12pt,
        openright,
        oneside,
        a4paper,
        chapter=TITLE,
        section=TITLE,
        subsection=Title,
        english,
        portugues,
        ]{abntex2}

\usepackage{UFPR}
% Pacotes básicos 
% ----------------------------------------------------------
\usepackage[T1]{fontenc}		% Selecao de codigos de fonte.
\usepackage[utf8]{inputenc}		% Codificacao do documento (conversão automática dos acentos)
\usepackage{lastpage}			% Usado pela Ficha catalográfica
\usepackage{indentfirst}		% Indenta o primeiro parágrafo de cada seção.

\usepackage{ifthen}		    	% para montar condicionais
\usepackage[brazil]{babel}		% para utilizar termos em portugues
\usepackage[final]{pdfpages}    % para incluir páginas de arquivos pdf
\usepackage{amsmath}			% simbolos matematicos

% Pacote para listagens de código
% ----------------------------------------------------------
\usepackage{listings}
\usepackage{xcolor}

% Configuração do listings para código
\lstset{
    basicstyle=\ttfamily\footnotesize,
    breaklines=true,
    frame=single,
    numbers=left,
    numberstyle=\tiny\color{gray},
    keywordstyle=\color{blue},
    commentstyle=\color{green!60!black},
    stringstyle=\color{orange},
    showstringspaces=false,
    tabsize=2,
    inputencoding=utf8,
    extendedchars=true,
    literate={á}{{\'a}}1 {é}{{\'e}}1 {í}{{\'i}}1 {ó}{{\'o}}1 {ú}{{\'u}}1
             {Á}{{\'A}}1 {É}{{\'E}}1 {Í}{{\'I}}1 {Ó}{{\'O}}1 {Ú}{{\'U}}1
             {ã}{{\~a}}1 {õ}{{\~o}}1 {Ã}{{\~A}}1 {Õ}{{\~O}}1
             {ç}{{\c{c}}}1 {Ç}{{\c{C}}}1
             {à}{{\`a}}1 {À}{{\`A}}1
}

% Linguagem JSON para listings
\lstdefinelanguage{json}{
    basicstyle=\ttfamily\footnotesize,
    morestring=[b]",
    stringstyle=\color{orange},
}

% Ambiente quadro (similar a tabela)
% ----------------------------------------------------------
\usepackage{float}
\newfloat{quadro}{htbp}{loq}[chapter]
\floatname{quadro}{Quadro}
\newcommand{\listofquadrosname}{Lista de Quadros}
\newlistof{listofquadros}{loq}{\listofquadrosname}
\newlistentry{quadro}{loq}{0}
\renewcommand{\cftquadroname}{QUADRO\space}
\renewcommand*{\cftquadroaftersnum}{\hfill--\hfill}

% ===========================================================
% CITAÇÕES - Usando abntex2cite (NATIVO do abnTeX2)
% ===========================================================
% [alf] = estilo autor-data (alfabético)
% [num] = estilo numérico
% Comandos disponíveis: \cite{}, \citeonline{}, \citeauthor{}, \citeyear{}
\usepackage[alf]{abntex2cite}

% Formatando o avanço dos títulos no sumário 
% ----------------------------------------------------------
\makeatletter
	\pretocmd{\chapter}{\addtocontents{toc}{\protect\addvspace{-12\p@}}}{}{}
	\pretocmd{\section}{\addtocontents{toc}{\protect\addvspace{-3\p@}}}{}{}
\makeatother

% https://groups.google.com/g/abntex2/c/ZYwE4t9uTFM
\makeatletter
\let\oldcontentsline\contentsline
\def\contentsline#1#2{%
	\expandafter\ifx\csname l@#1\endcsname\l@section
	\expandafter\@firstoftwo
	\else
	\expandafter\@secondoftwo
	\fi
	{%
		\oldcontentsline{#1}{\MakeTextUppercase{#2}}%
	}{%
		\oldcontentsline{#1}{#2}%
	}%
}
\makeatother

% Para ajustar o tamanho da fonte do número da primeira página do capítulo
% ----------------------------------------------------------
\makepagestyle{chapfirst}
\makeoddhead{chapfirst}{}{}{\footnotesize{\thepage}}

% Criar um novo estilo de cabeçalhos e rodapés
\makepagestyle{simplestextual}
  \makeevenhead{simplestextual}{}{}{\footnotesize \thepage}
  \makeoddhead{simplestextual}{}{}{\footnotesize \thepage}
  \makeevenfoot{simplestextual}{}{}{}
  \makeoddfoot{simplestextual}{}{}{}


%%%%%%%%%%%%%%%%%%%%%%%%%%%%%%%%%%%%%%%%%%%%%%%%%%%%%%%
% Arquivo para entrada de dados para a parte pré textual
% TCC APPUNTURE - UFPR 2025
%%%%%%%%%%%%%%%%%%%%%%%%%%%%%%%%%%%%%%%%%%%%%%%%%%%%%%%

% Informações de dados para CAPA e FOLHA DE ROSTO
%----------------------------------------------------------------------------- 
\tipotrabalho{Trabalho de Conclusão de Curso}

% Marcar Sim para as partes que irão compor o documento pdf
%----------------------------------------------------------------------------- 
\providecommand{\terCapa}{Sim}
\providecommand{\terFolhaRosto}{Sim}
\providecommand{\terTermoAprovacao}{Nao}
\providecommand{\terDedicatoria}{Nao}
\providecommand{\terFichaCatalografica}{Nao}
\providecommand{\terEpigrafe}{Nao}
\providecommand{\terAgradecimentos}{Nao}
\providecommand{\terErrata}{Nao}
\providecommand{\terListaFiguras}{Sim}
\providecommand{\terListaQuadros}{Sim}
\providecommand{\terListaTabelas}{Sim}
\providecommand{\terSiglasAbrev}{Sim} 
\providecommand{\terSimbolos}{Nao}
\providecommand{\terResumos}{Sim}
\providecommand{\terSumario}{Sim}
\providecommand{\terAnexo}{Nao}
\providecommand{\terApendice}{Sim}
\providecommand{\terIndiceR}{Nao}
%----------------------------------------------------------------------------- 

\titulo{APPUNTURE: Atlas Digital Educativo de Acupuntura}

% Para múltiplos autores, use o comando autor com todos os nomes
\autor{Bruno Brugnerotto de Lara \and Gabriel Francelino Voidaleski \and Pedro Henrique Lopes}

\local{Curitiba}
\data{2025}

% Orientador ou Orientadora
\orientador{Prof. Dr. Paulo Sobreira Moraes}
\orientadora{}

% Coorientador ou Coorientadora (se houver)
\coorientador{}
\coorientadora{}

% Segundo Coorientador ou Segunda Coorientadora
\scoorientador{}
\scoorientadora{}

% ----------------------------------------------------------
% Arquivo de referências (usado pelo abntex2cite com bibtex)
% O comando \bibliography{referencias} está no main.tex

% ----------------------------------------------------------
\instituicao{Universidade Federal do Paraná}

\def \ImprimirSetor{Setor de Educação Profissional e Tecnológica}

\def \ImprimirProgramaPos{}

\def \ImprimirCurso{Curso de Tecnologia em Análise e Desenvolvimento de Sistemas}

\preambulo{Trabalho de Conclusão de Curso apresentado como requisito parcial à obtenção do grau de Tecnólogo em Análise e Desenvolvimento de Sistemas, Setor de Ciências Exatas da Universidade Federal do Paraná.}

%----------------------------------------------------------------------------- 

\newcommand{\imprimirCurso}{Tecnologia em Análise e Desenvolvimento de Sistemas}

\newcommand{\imprimirDataDefesa}{Dezembro de 2025}

\newcommand{\imprimircdu}{004.5:615.814.1}

% ----------------------------------------------------------
% Errata (se necessário)
\newcommand{\imprimirerrata}{}

% ----------------------------------------------------------
% Assinaturas do Termo de Aprovação
%\newcommand{\AssinaAprovacao}{
 %  \assinatura{Prof. Dr. Paulo Sobreira Moraes \\ Orientador - UFPR}
  % \assinatura{Professor(a) Avaliador(a) 1 \\ UFPR}
   %\assinatura{Professor(a) Avaliador(a) 2 \\ UFPR}
    %  
   %\begin{center}
    %\vspace*{0.5cm}
    %\imprimirlocal, \imprimirDataDefesa.
    %\vspace*{1cm}
  %\end{center}
%}

% ----------------------------------------------------------
% Epígrafe
% \newcommand{\EpigrafeTexto}{
% \textit{``A jornada de mil milhas começa com um único passo.''}\\
% (Lao Tzu)
% }

% ----------------------------------------------------------
% RESUMO em Português
\newcommand{\ResumoTexto}{
O \textit{Appunture} é um atlas digital educativo voltado a estudantes e profissionais interessados em acupuntura, com o objetivo de facilitar o acesso a informações sobre os pontos de acupuntura corporal. Em um cenário onde a formação na área é heterogênea e o acesso a materiais de qualidade pode ser limitado, a ferramenta busca auxiliar no processo de aprendizagem e consulta, disponibilizando um atlas anatômico digital organizado por meridianos.

A plataforma foi desenvolvida para Android, apresentando 15 visualizações vetoriais (SVG) organizadas por meridianos e vistas (frontal e posterior). O sistema oferece funcionalidades de busca, detalhes dos 361 pontos principais, e um assistente baseado em Inteligência Artificial para auxiliar na correlação entre sintomas e pontos.

O desenvolvimento seguiu uma arquitetura ``offline-first'', permitindo o acesso aos dados mesmo sem conexão à internet. Para auxiliar no mapeamento dos pontos sobre as imagens anatômicas, foi desenvolvida uma ferramenta interna de posicionamento de coordenadas.

É importante ressaltar que o \textit{Appunture} não substitui a formação profissional adequada nem o acompanhamento de profissionais qualificados. A ferramenta foi concebida exclusivamente como um recurso educativo complementar de apoio ao estudo e à consulta de informações sobre acupuntura, não devendo ser utilizada para fins de diagnóstico, prescrição ou tratamento.
} 

\newcommand{\PalavraschaveTexto}{Acupuntura; Atlas Digital; Aplicativo Educativo; Medicina Tradicional Chinesa; Tecnologia Educacional.}

% ----------------------------------------------------------
% ABSTRACT em Inglês
\newcommand{\AbstractTexto}{
\textit{Appunture} is an educational digital atlas designed to facilitate access to information about acupuncture points for students and professionals interested in the field. The Android application features a digital anatomical atlas with 15 vector views (SVG), organized by meridians. The system includes search functionalities, details of 361 main points, and an AI-based assistant to help correlate symptoms and points.

Developed with an ``offline-first'' architecture using React Native and Java Spring Boot, the application allows data access even without internet connection. An internal mapping tool was developed to assist in positioning point coordinates on the anatomical images.

It is important to note that \textit{Appunture} does not replace proper professional training. The tool was designed exclusively as a complementary educational resource to support learning and information lookup about acupuncture, and should not be used for diagnosis, prescription, or treatment purposes.
}

\newcommand{\KeywordsTexto}{Acupuncture; Digital Atlas; Educational Application; Traditional Chinese Medicine; Educational Technology.}

% Resumos em outros idiomas (francês e espanhol) - deixar vazio se não usar
\newcommand{\Resume}{}
\newcommand{\Motscles}{}
\newcommand{\Resumen}{}
\newcommand{\Palabrasclave}{}

% ----------------------------------------------------------
% Agradecimentos
%\newcommand{\AgradecimentosTexto}{
%Agradecemos primeiramente a Deus por nos dar força e sabedoria para concluir este trabalho.

%Ao nosso orientador, Prof. Dr. Paulo Sobreira Moraes, pela orientação, paciência e dedicação durante todo o desenvolvimento deste projeto.

%À Universidade Federal do Paraná, pela oportunidade de formação e pelo ambiente propício ao aprendizado e à pesquisa.

%Aos nossos familiares e amigos, pelo apoio incondicional e compreensão nos momentos de ausência.

%A todos os profissionais da área de acupuntura que contribuíram com suas experiências e conhecimentos para a construção deste aplicativo.

%Por fim, agradecemos a todos que, direta ou indiretamente, contribuíram para a realização deste trabalho.
%}

% ----------------------------------------------------------
% Dedicatória
\newcommand{\DedicatoriaTexto}{
\textit{Dedicamos este trabalho a todos os estudantes e\\profissionais que buscam constantemente aprimorar\\seus conhecimentos na área da acupuntura.}
}

% ----------------------------------------------------------
% Lista de Siglas e Abreviaturas
% Utilizar no texto: \sigla{MTC}{Medicina Tradicional Chinesa}
% ----------------------------------------------------------
\criarsigla{API}{Application Programming Interface}
\criarsigla{JWT}{JSON Web Token}
\criarsigla{LLM}{Large Language Model}
\criarsigla{MMKV}{Memory Mapped Key-Value}
\criarsigla{MTC}{Medicina Tradicional Chinesa}
\criarsigla{OMS}{Organização Mundial da Saúde}
\criarsigla{RAG}{Retrieval-Augmented Generation}
\criarsigla{RBAC}{Role-Based Access Control}
\criarsigla{SDK}{Software Development Kit}
\criarsigla{SQL}{Structured Query Language}
\criarsigla{SVG}{Scalable Vector Graphics}
\criarsigla{UML}{Unified Modeling Language}

% ----------------------------------------------------------
% Lista de Símbolos
% ----------------------------------------------------------


\makeindex

\begin{document}

% Adequando o uppercase titulo dos elementos nas suas respectivas legendas
\renewcommand{\bibname}{{REFER\^ENCIAS}}
\renewcommand{\tablename}{TABELA }
\renewcommand{\figurename}{FIGURA }
\renewcommand{\figureautorefname}{FIGURA}
\renewcommand{\tableautorefname}{TABELA}
\newcommand{\equationname}{equa\c{c}\~ao~}
\renewcommand{\equationautorefname}{equa\c{c}\~ao~}
\newcommand{\quadroautorefname}{QUADRO~}

\aliaspagestyle{chapter}{chapfirst}

\input{00-pretextual}

% ----------------------------------------------------------
% ELEMENTOS TEXTUAIS
% ----------------------------------------------------------
\textual
\pagestyle{simplestextual}

% Capítulos
% ----------------------------------------------------------
% Capítulo 1 - INTRODUÇÃO
% ----------------------------------------------------------
\chapter{INTRODUÇÃO} \label{cha:introducao}

A prática da acupuntura no Brasil, embora cada vez mais difundida e procurada tanto no sistema público quanto no privado de saúde, ainda enfrenta desafios significativos no que diz respeito à regulamentação profissional, padronização de condutas, segurança técnica e formação dos praticantes. Trata-se de uma técnica milenar baseada na Medicina Tradicional Chinesa (MTC), que se expandiu no país por meio de cursos livres, técnicos e pós-graduações voltadas a diferentes profissionais da área da saúde. No entanto, a ausência de regulamentação federal clara e a disputa entre conselhos profissionais sobre quem pode exercer legalmente a acupuntura têm gerado um ambiente de insegurança jurídica, afetando não apenas os acupunturistas, mas também o acesso qualificado da população à prática.

Além das disputas institucionais, o cenário atual é marcado pela fiscalização fragmentada, formação heterogênea e riscos crescentes à biossegurança e à saúde do paciente, especialmente em ambientes informais. Segundo diretrizes da Organização Mundial da Saúde (OMS), o uso inadequado de agulhas, a falta de higienização correta ou a aplicação indevida de técnicas podem resultar em infecções, lesões internas e outros agravos que comprometem gravemente a integridade dos usuários da acupuntura \cite{oms1999}.

Outro desafio relevante é o caráter subjetivo e complexo do diagnóstico energético, próprio da MTC. O raciocínio clínico baseado em padrões de desarmonia (como deficiência de Qi, estagnação de sangue, entre outros) exige extenso treinamento e sensibilidade clínica. A falta de padronização e apoio didático dificulta a atuação de profissionais iniciantes ou com formação limitada, podendo comprometer a eficácia terapêutica.

Diante desse contexto, o uso de Tecnologias Digitais de Informação e Comunicação (TDIC) apresenta-se como uma ferramenta estratégica para apoiar a qualificação da prática da acupuntura. Soluções como aplicativos educativos e interativos podem auxiliar na visualização de pontos, organização de informações terapêuticas, revisão diagnóstica e adequação de conteúdos conforme o nível de formação do usuário, promovendo maior segurança, padronização e confiabilidade.

Neste cenário, propõe-se o desenvolvimento do \textit{Appunture}, um aplicativo multiplataforma (Android, iOS e Web) que funciona como um atlas digital de acupuntura interativo, voltado tanto para estudantes quanto para profissionais da área da saúde. A ferramenta oferece acesso \textit{offline} a um banco de dados completo contendo os 361 pontos dos meridianos principais, pontos extras, e um atlas anatômico composto por 15 visualizações vetoriais (SVG) de alta fidelidade, organizadas por meridianos e vistas (frontal e posterior). O aplicativo também conta com um assistente inteligente integrado que utiliza Inteligência Artificial Generativa (Spring AI + Google Gemini) para correlacionar sintomas e sugerir protocolos baseados em evidências.

Com interface moderna e uso de tecnologias híbridas, o \textit{Appunture} foi desenvolvido com arquitetura ``\textit{offline-first}'', garantindo acesso total aos dados mesmo sem conexão à internet, com sincronização automática quando online. Para garantir a precisão dos dados anatômicos, foi desenvolvida uma ferramenta proprietária de mapeamento de pontos, assegurando a exatidão das coordenadas em relação às estruturas corporais. Complementarmente, um painel administrativo web permite que gestores e especialistas atualizem os conteúdos do sistema com segurança, mantendo a base sempre atualizada.

Essa solução busca não apenas facilitar o acesso à informação qualificada, mas também contribuir para a padronização e segurança no uso da acupuntura no Brasil.

\section{OBJETIVOS} \label{sec:objetivos}

O principal objetivo deste projeto é desenvolver e implementar o aplicativo \textit{Appunture}, uma ferramenta digital educativa e interativa voltada à prática da acupuntura. O aplicativo tem como foco principal promover a padronização de condutas, aumentar a segurança técnica e oferecer suporte ao raciocínio clínico dos profissionais da área, por meio de recursos acessíveis, visuais e inteligentes. A iniciativa também busca contribuir para a formação contínua e o fortalecimento da prática multiprofissional, especialmente em um contexto de regulamentação incerta e formação heterogênea.

\subsection{Objetivos Específicos}

Entre os objetivos específicos, destacam-se:

\begin{itemize}
    \item \textbf{Visualização Didática e Interativa dos Pontos}: Oferecer um atlas anatômico digital composto por 15 visualizações vetoriais (SVG) de alta resolução, cobrindo vistas frontais e posteriores e todos os meridianos principais, permitindo interação precisa e responsiva.
    
    \item \textbf{Assistência Clínica com Inteligência Artificial}: Implementar um módulo de IA Generativa (RAG - \textit{Retrieval-Augmented Generation}) utilizando Spring AI e Google Gemini, capaz de interpretar perguntas complexas sobre sintomas e tratamentos, fornecendo respostas contextualizadas com a base de dados técnica do sistema.
    
    \item \textbf{Apoio à Formação Multiprofissional}: Disponibilizar conteúdos adaptados aos diferentes níveis de formação e áreas de atuação dos profissionais.
    
    \item \textbf{Organização Estruturada de Informações Terapêuticas}: Base de dados completa dos 361 pontos clássicos e pontos extras, com informações detalhadas sobre localização, indicações, contraindicações e funções energéticas.
    
    \item \textbf{Sincronização de Dados e Suporte Offline}: Arquitetura ``\textit{Offline-First}'' robusta, permitindo operação plena sem internet, com sincronização automática quando conectado.
    
    \item \textbf{Ferramenta de Mapeamento de Precisão}: Desenvolvimento de uma ferramenta web interna para mapeamento manual e validação das coordenadas de todos os pontos de acupuntura sobre os mapas vetoriais.
    
    \item \textbf{Gestão e Atualização de Conteúdo via Painel Administrativo Web}: Interface administrativa segura para gestores e especialistas, permitindo o cadastro e atualização de pontos, sintomas e relações clínicas.
\end{itemize}

Dessa forma, o \textit{Appunture} propõe-se não apenas como um recurso digital de apoio ao estudo da acupuntura, mas como um instrumento estratégico de educação, padronização e valorização profissional, alinhado às diretrizes das Práticas Integrativas e Complementares (PICS) e às demandas contemporâneas da saúde pública e privada.

% ----------------------------------------------------------
% Capítulo 2 - FUNDAMENTAÇÃO TEÓRICA
% ----------------------------------------------------------
\chapter{FUNDAMENTAÇÃO TEÓRICA} \label{cha:fundamentacao}

A construção de uma solução digital eficaz voltada ao ensino e à prática da acupuntura exige respaldo teórico multidisciplinar, abrangendo desde os fundamentos da acessibilidade na saúde até os princípios de usabilidade e desenvolvimento de tecnologias educativas. Este capítulo busca contextualizar e embasar as escolhas metodológicas e funcionais do aplicativo \textit{Appunture}, discutindo as bases conceituais que orientam sua proposta. Para isso, são explorados os aspectos relacionados à democratização do acesso à informação técnica em saúde, o papel das Tecnologias Digitais de Informação e Comunicação (TDICs) no apoio à formação e à prática clínica, além da comparação com outras ferramentas disponíveis no mercado. A fundamentação aqui apresentada justifica a necessidade de um aplicativo como o \textit{Appunture} e oferece suporte acadêmico às decisões adotadas no desenvolvimento do projeto.

\section{ACESSIBILIDADE COMO DIREITO E FERRAMENTA DE IGUALDADE NA SAÚDE} \label{sec:acessibilidade}

A acessibilidade ao conhecimento e à prática segura em saúde deve ser compreendida como um direito fundamental, especialmente em contextos nos quais o acesso à formação e à informação de qualidade é desigual. No caso da acupuntura, esse desafio se intensifica pela diversidade de formações entre os praticantes --- que podem ser médicos, fisioterapeutas, terapeutas integrativos ou técnicos em estética --- e pela ausência de uma regulamentação federal unificada que determine diretrizes claras sobre a prática no Brasil. Essa lacuna acarreta diferentes formas de aprendizado, muitas vezes empíricas ou informalizadas, o que torna ainda mais necessária a disponibilização de conteúdos confiáveis e acessíveis.

\citeonline{freire1996} em sua abordagem pedagógica libertadora, enfatiza que o conhecimento só adquire sentido real quando é compartilhado, situado e contextualizado. Nesse sentido, a disseminação de informações em saúde --- sobretudo em áreas complexas como a acupuntura --- deve ir além do conteúdo técnico e considerar o contexto cultural, social e educacional de quem aprende. O acesso à informação, portanto, não pode depender apenas de condições econômicas, infraestrutura local ou formação prévia, pois isso aprofundaria as desigualdades já existentes entre os profissionais de saúde nas diversas regiões do país.

A inclusão digital torna-se uma aliada crucial nesse processo. Segundo \citeonline{booth2021}, tecnologias bem projetadas são capazes de reduzir barreiras estruturais no acesso à formação em saúde, desde que respeitem os diferentes níveis de letramento digital, necessidades pedagógicas e realidades locais dos usuários. Isso significa que o design de soluções digitais deve ser centrado no usuário e incluir funcionalidades como interfaces intuitivas, conteúdos multimodais (texto, áudio, imagem, vídeo), acessibilidade para pessoas com deficiência e responsividade para dispositivos móveis.

Dados do \textit{TIC Saúde 2023} revelam que mais de 65\% dos profissionais de saúde no Brasil usam \textit{smartphones} como ferramenta principal de acesso a informações clínicas \cite{ticsaude2023}. Isso reforça a urgência de produzir plataformas móveis eficazes, que funcionem bem mesmo em condições técnicas limitadas, como conexões lentas ou dispositivos com pouca capacidade de processamento.

Além disso, conforme apontam \citeonline{souza2020}, aplicativos de apoio à prática clínica devem não apenas apresentar informações confiáveis, mas também oferecer recursos de interação e personalização, que favoreçam o aprendizado ativo e a construção de confiança do profissional em sua atuação.

Dessa forma, um aplicativo como o proposto neste trabalho, que visa tornar o conhecimento da acupuntura mais acessível, confiável e intuitivo, cumpre um papel social importante: reduzir desigualdades, promover a formação segura de profissionais, e, consequentemente, aumentar a qualidade do atendimento prestado à população.

\section{TECNOLOGIAS DIGITAIS COMO APOIO À FORMAÇÃO E PRÁTICA SEGURA} \label{sec:tecnologias_digitais}

A utilização de tecnologias digitais aplicadas à saúde vem transformando não apenas os meios de acesso à informação, mas também a forma como os profissionais aplicam seu conhecimento na prática clínica. No campo da acupuntura, que exige domínio anatômico preciso e entendimento das interações energéticas do corpo humano, o uso de recursos visuais e interativos pode potencializar o aprendizado e a execução segura das técnicas.

Segundo \citeonline{moraes2020}, o uso de aplicativos, plataformas interativas e realidade aumentada na educação em saúde estimula a aprendizagem significativa, reduz o tempo de assimilação de conteúdos complexos e melhora a retenção do conhecimento. No caso da acupuntura, isso se torna ainda mais relevante ao considerar que muitos dos pontos utilizados não são visíveis a olho nu, exigindo visualização em profundidade, correlação com estruturas internas e treino constante.

Além disso, estudos como o de \citeonline{oliveira2022} destacam que alunos e profissionais da área de terapias integrativas relataram maior segurança em suas práticas após utilizar aplicativos educativos que oferecem representações tridimensionais e localização anatômica de pontos de acupuntura.

Portanto, investir no desenvolvimento de soluções digitais adaptadas à acupuntura não apenas moderniza o ensino, mas também amplia a segurança da prática clínica. Tais recursos são especialmente relevantes para estudantes em formação e profissionais em início de carreira, que frequentemente enfrentam dificuldades em aplicar corretamente os pontos e técnicas aprendidos de forma teórica.

\section{APLICATIVOS DE ACUPUNTURA: EXPERIÊNCIAS E RECURSOS TECNOLÓGICOS} \label{sec:aplicativos_acupuntura}

O desenvolvimento de tecnologias na área da saúde deve considerar mais do que apenas conteúdo técnico. A forma como o usuário interage com a interface, compreende as informações e se orienta dentro do aplicativo é determinante para a sua eficácia. No contexto da acupuntura --- que envolve a localização precisa de pontos no corpo, o entendimento de trajetos de meridianos e a aplicação de protocolos --- a usabilidade se torna um pilar essencial.

De acordo com \citeonline{norman2013}, um bom design deve ``tornar possível o uso intuitivo'', ou seja, permitir que o usuário compreenda e utilize o sistema sem esforço cognitivo excessivo. Em aplicativos de saúde, isso significa que menus confusos, excesso de informações, termos técnicos não explicados ou falta de \textit{feedback} visual comprometem a experiência do usuário e até mesmo colocam a segurança da prática em risco.

A Organização Mundial da Saúde \cite{oms2020} recomenda que soluções tecnológicas em saúde digital adotem princípios de design centrado no usuário (DCU), considerando desde o início do desenvolvimento as reais necessidades, limitações e habilidades dos usuários. Isso é especialmente importante no caso de profissionais que atuam em contextos com restrições de tempo, estrutura ou formação técnica, como muitos terapeutas integrativos ou estudantes de cursos livres.

Além disso, \citeonline{ferreira2021} ressaltam que aplicativos com interfaces acessíveis e organizadas promovem maior retenção do conhecimento e reduzem erros clínicos, especialmente em áreas que exigem visualização corporal anatômica como a fisioterapia e a acupuntura.

Por isso, o aplicativo proposto neste TCC prioriza:

\begin{itemize}
    \item Interfaces limpas e intuitivas;
    \item Navegação fluida e segmentada por áreas do corpo;
    \item \textit{Feedback} visual ao selecionar pontos de acupuntura;
    \item Compatibilidade com dispositivos móveis de diferentes capacidades;
    \item Inclusão de recursos de busca inteligente, filtros por sintomas e suporte ao idioma técnico com explicações simplificadas.
\end{itemize}

Esses elementos reforçam o compromisso com a acessibilidade e a segurança, pilares fundamentais quando se pensa em soluções tecnológicas que impactam diretamente a formação e atuação clínica dos profissionais da saúde.

\section{COMPARATIVO COM APLICATIVOS EXISTENTES} \label{sec:comparativo}

Atualmente, existem diversos aplicativos disponíveis na Play Store que buscam auxiliar estudantes e profissionais da área da acupuntura por meio de recursos visuais e informativos. No entanto, ao analisar criticamente essas soluções, é possível perceber limitações significativas que comprometem sua eficácia como ferramentas de apoio à prática clínica e ao aprendizado, especialmente no contexto brasileiro.

O \textit{Anatomy Learning}, por exemplo, é um aplicativo renomado para o estudo da anatomia em 3D. Sua qualidade gráfica e riqueza de detalhes são inegáveis, porém o aplicativo não possui foco específico em acupuntura, o que o torna pouco funcional para quem busca compreender os pontos e os meridianos energéticos. Além disso, seu conteúdo é voltado a usuários com formação técnica mais avançada, o que pode representar uma barreira para estudantes de cursos livres ou profissionais de áreas menos técnicas. Outro ponto problemático é seu desempenho em dispositivos de menor capacidade, com relatos de travamentos e lentidão.

Já o \textit{Visual Acupuncture 3D} apresenta um escopo mais próximo ao esperado, oferecendo visualização de pontos e meridianos. No entanto, peca em aspectos de usabilidade. A interface é considerada pouco intuitiva por muitos usuários, com menus confusos, poluição visual e falta de clareza nos textos. Esses fatores dificultam seu uso durante a prática clínica, especialmente em momentos em que o profissional precisa localizar rapidamente um ponto ou revisar um protocolo. Além disso, a ausência de suporte ao idioma português reduz sua acessibilidade para profissionais brasileiros.

Outro aplicativo conhecido é o \textit{Acupuncture 3D}, que também disponibiliza visualizações anatômicas tridimensionais focadas em acupuntura. Entretanto, o aplicativo carece de atualizações e apresenta conteúdos desatualizados em relação às nomenclaturas e práticas adotadas no Brasil. Sua interface é rígida e limitada, com dificuldade de navegação e ausência de funcionalidades complementares como busca por sintomas, indicação de protocolos ou explicações detalhadas para aplicação dos pontos.

Em contraste com essas soluções, o aplicativo proposto neste TCC se destaca por adotar uma abordagem centrada no usuário, com interface limpa, navegação fluida e foco na aplicação prática da acupuntura em contexto clínico. Ele foi pensado para funcionar bem mesmo em dispositivos mais simples, garantindo acessibilidade digital a um público mais amplo. Além disso, traz conteúdo em português, explicações didáticas e filtros de busca por sintomas, aproximando-se mais das reais necessidades dos profissionais e estudantes brasileiros da área.

Essa comparação evidencia que, embora os aplicativos analisados ofereçam contribuições pontuais, nenhum deles contempla simultaneamente os pilares de acessibilidade, usabilidade, confiabilidade e contexto nacional. O desenvolvimento do aplicativo proposto surge, assim, como uma resposta direta às lacunas existentes no mercado digital de apoio à acupuntura, promovendo inclusão, segurança e aprimoramento profissional.

\section{USABILIDADE E DESIGN CENTRADO NO USUÁRIO} \label{sec:usabilidade}

Para que a tecnologia seja de fato acessível e eficaz, é essencial que siga princípios de usabilidade e design centrado no usuário. \citeonline{nielsen1994} define usabilidade como a capacidade de um sistema ser usado por seus usuários com eficiência, efetividade e satisfação. Esse conceito é especialmente importante em contextos de formação em saúde, onde a curva de aprendizado precisa ser suavizada.

Segundo \citeonline{siqueira2021}, aplicações móveis voltadas para profissionais da saúde que incorporam princípios de design centrado no usuário têm maiores taxas de adesão e melhor impacto educacional. A adaptação do conteúdo conforme o perfil do usuário, como proposto no \textit{Appunture}, está em consonância com o modelo de usabilidade responsiva, recomendado por padrões como a norma ISO/IEC 25010.

\section{NORMAS, DIRETRIZES E EVIDÊNCIAS CIENTÍFICAS} \label{sec:normas}

O desenvolvimento de soluções tecnológicas voltadas à saúde deve considerar diretrizes oficiais e evidências científicas. A OMS \cite{oms1999} publicou um guia para treinamento e segurança em acupuntura, enfatizando a necessidade de padronização de pontos, técnicas de punção e biossegurança.

Na área de software, normas como a ISO 9241-210 e a ISO/IEC 25010 orientam o desenvolvimento de interfaces acessíveis, seguras e eficientes, especialmente em sistemas críticos como os de saúde. A adoção dessas diretrizes no \textit{Appunture} garante não apenas a funcionalidade, mas também a segurança e a confiabilidade da ferramenta.

% ----------------------------------------------------------
% Capítulo 3 - MOTIVAÇÕES
% ----------------------------------------------------------
\chapter{MOTIVAÇÕES} \label{cha:motivacoes}

O desenvolvimento do aplicativo \textit{Appunture} surge a partir de uma série de desafios estruturais enfrentados pela prática da acupuntura no Brasil. Estes desafios envolvem a regulamentação da profissão, a formação heterogênea dos profissionais, a fragilidade nos processos de fiscalização, os riscos à biossegurança e a complexidade diagnóstica da Medicina Tradicional Chinesa (MTC). A seguir, são apresentados os principais fatores motivadores que embasam a proposta deste projeto.

\section{REGULAMENTAÇÃO E DISPUTA DE CATEGORIAS PROFISSIONAIS} \label{sec:regulamentacao}

A regulamentação da acupuntura no Brasil é um dos maiores entraves para sua prática estruturada e segura. Apesar de sua origem milenar na MTC e da crescente popularização a partir das décadas finais do século XX, ainda não há consenso nacional sobre quais categorias profissionais estão habilitadas a aplicar acupuntura. Isso tem gerado disputas entre conselhos de diferentes áreas, como medicina, fisioterapia, enfermagem, psicologia, odontologia e farmácia.

Enquanto o Conselho Federal de Medicina (CFM) defende a acupuntura como especialidade médica, conselhos de outras áreas reconhecem a prática multiprofissional e regulamentam internamente sua atuação. Essa disputa tem gerado insegurança jurídica e limitações à atuação de profissionais devidamente capacitados, ao mesmo tempo em que abre espaço para práticas irregulares.

A proposta do Projeto de Lei 5983/2019, atualmente em tramitação no Senado Federal, busca criar critérios nacionais para o exercício da acupuntura, independentemente da formação de base, desde que haja capacitação específica. Enquanto isso, soluções tecnológicas devem ser sensíveis a esse cenário, respeitando os limites legais e adaptando os conteúdos conforme o perfil profissional do usuário.

\section{FORMAÇÃO E FISCALIZAÇÃO INSUFICIENTES} \label{sec:formacao}

A ausência de um conselho exclusivo e de uma legislação unificada gera uma lacuna na fiscalização da prática da acupuntura no Brasil. Muitos profissionais atuam com formações diversas, que vão desde cursos livres até especializações, sem uma padronização mínima nacional. Além disso, terapeutas que não estão vinculados a conselhos de classe podem atuar sem supervisão ou responsabilização adequada.

Nesse contexto, um aplicativo como o \textit{Appunture} pode oferecer suporte educativo e técnico, contribuindo para o aperfeiçoamento de profissionais com diferentes formações. Por meio da diferenciação de perfis de usuários e de um sistema de conteúdos escalonados conforme a formação declarada, o aplicativo propõe uma experiência segura, ética e ajustada à realidade do usuário.

\section{SEGURANÇA DO PACIENTE E BIOSSEGURANÇA} \label{sec:seguranca}

A acupuntura, por envolver a inserção de agulhas no corpo humano, é considerada uma prática invasiva e requer protocolos rigorosos de biossegurança. A utilização inadequada de instrumentos ou a realização do procedimento por pessoas sem conhecimento técnico podem gerar complicações sérias, como infecções, lesões internas, hematomas e até pneumotórax.

Casos documentados, como o de uma paciente hospitalizada após perfuração pulmonar em São Paulo, demonstram os riscos reais da má prática. Muitos desses casos ocorrem em ambientes informais ou com baixa qualificação técnica.

A tecnologia pode ser uma aliada na disseminação de boas práticas. O aplicativo \textit{Appunture} fornece informações claras e baseadas em diretrizes de segurança da OMS, orientando sobre profundidade de pontuação, indicações, contraindicações e características especiais dos pontos, contribuindo para uma atuação mais segura e responsável.

\section{DIAGNÓSTICO ENERGÉTICO: COMPLEXIDADE E SUBJETIVIDADE} \label{sec:diagnostico}

O raciocínio clínico na acupuntura, segundo os princípios da MTC, exige uma abordagem subjetiva e holística, muitas vezes desafiadora para iniciantes. Diagnósticos são baseados em padrões energéticos, observações da língua e do pulso, e uma análise minuciosa do estado geral do paciente.

Diferentemente da medicina ocidental, não se trata apenas de associar um sintoma a uma causa direta, mas de identificar síndromes energéticas como estagnação de Qi, excesso de calor, umidade interna, entre outras. O mesmo sintoma, como dor abdominal, pode ter interpretações energéticas distintas, dependendo do conjunto de sinais apresentados.

Nesse sentido, o aplicativo \textit{Appunture} atua como um apoio ao raciocínio clínico, fornecendo descrições de padrões, relações entre sintomas e síndromes e, inclusive, oferecendo um sistema de busca inteligente por sintomas com suporte de NLP (Processamento de Linguagem Natural), que interpreta a linguagem do usuário e sugere os pontos mais indicados para o caso apresentado.

Diante dos desafios regulatórios, formativos e clínicos que permeiam a prática da acupuntura no Brasil, torna-se essencial o desenvolvimento de ferramentas tecnológicas que aliem conhecimento técnico, segurança e acessibilidade. O \textit{Appunture} surge como resposta a essas motivações, buscando oferecer um recurso digital estratégico para fortalecer a prática da acupuntura com ética, qualidade e base científica.

% ----------------------------------------------------------
% Capítulo 4 - TECNOLOGIAS E ARQUITETURA DO SISTEMA
% ----------------------------------------------------------
\chapter{TECNOLOGIAS E ARQUITETURA DO SISTEMA} \label{cha:tecnologias}

Este capítulo apresenta o \textit{stack} tecnológico utilizado no desenvolvimento do \textit{Appunture}, detalhando a arquitetura do sistema, as tecnologias empregadas e as decisões técnicas que fundamentam a implementação da solução.

\section{VISÃO GERAL DA ARQUITETURA} \label{sec:visao_arquitetura}

O \textit{Appunture} foi desenvolvido seguindo uma arquitetura híbrida distribuída, composta por quatro camadas principais:

\begin{itemize}
    \item \textbf{Aplicativo Móvel (React Native + Expo)}: Interface principal do usuário com estratégia ``\textit{Offline-First}'';
    \item \textbf{API Backend (Java + Spring Boot)}: Servidor de aplicação responsável pela lógica de negócio, sincronização e Inteligência Artificial;
    \item \textbf{Interface Web Administrativa (React)}: Painel de administração para gestão de conteúdo;
    \item \textbf{Ferramentas de Apoio (Python/HTML)}: Scripts de migração de dados e ferramenta de mapeamento de pontos.
\end{itemize}

Esta arquitetura garante alta disponibilidade, performance otimizada e experiência consistente mesmo em ambientes com conectividade limitada.

\subsection{Diagrama de Arquitetura}

O \textit{Appunture} foi desenvolvido seguindo uma arquitetura híbrida e modular, que integra funcionalidades \textit{offline} e \textit{online} para garantir alta disponibilidade, desempenho e usabilidade mesmo em ambientes com conexão limitada ou instável.

% Inserir figura do diagrama de arquitetura
% \figura{DIAGRAMA DE ARQUITETURA DO SISTEMA}{0.8}{fig/arquitetura.png}{Os autores (2025)}{arquitetura}{}{}

\section{ARQUITETURA HÍBRIDA (OFFLINE + ONLINE)} \label{sec:arquitetura_hibrida}

\subsection{Funcionamento Offline}

O aplicativo móvel foi projetado com estratégia ``\textit{offline-first}'', garantindo que as funcionalidades de navegação e consulta de dados funcionem independentemente da conectividade:

\textbf{Banco de Dados Local (SQLite):}
\begin{itemize}
    \item Armazena todos os 361 pontos de acupuntura dos meridianos principais;
    \item Contém pontos extras, sintomas e suas relações terapêuticas;
    \item Mantém favoritos e anotações do usuário;
    \item Estrutura otimizada para consultas rápidas;
    \item Garantia de performance instantânea e independência de rede.
\end{itemize}

\textbf{Nota:} A busca inteligente por sintomas utilizando o assistente de IA requer conexão com a internet. Quando \textit{offline}, o usuário pode navegar manualmente pelo atlas anatômico e acessar os detalhes dos pontos já sincronizados.

\textbf{Sistema de Cache Inteligente (MMKV):}
\begin{itemize}
    \item Armazena configurações de usuário e estados da aplicação;
    \item \textit{Cache} de imagens SVG e recursos visuais;
    \item Dados de sessão e preferências de interface;
    \item Performance até 30x superior ao AsyncStorage tradicional.
\end{itemize}

\subsection{Sincronização Inteligente}

Quando conectado à internet, o sistema implementa sincronização bidirecional automática:

\textbf{Sincronização Descendente (Backend $\rightarrow$ Mobile):}
\begin{itemize}
    \item Atualização automática de dados clínicos;
    \item \textit{Download} incremental baseado em versionamento;
    \item Compressão de dados para otimizar transferência;
    \item Validação de integridade dos dados recebidos.
\end{itemize}

\textbf{Sincronização Ascendente (Mobile $\rightarrow$ Backend):}
\begin{itemize}
    \item \textit{Upload} de favoritos e anotações do usuário;
    \item \textit{Backup} automático de preferências;
    \item \textit{Logs} de uso para análise de performance;
    \item Controle de conflitos com \textit{timestamp}.
\end{itemize}

\section{STACK TECNOLÓGICO DETALHADO} \label{sec:stack}

\subsection{Frontend Mobile (React Native + Expo)}

Segundo a documentação oficial \cite{reactnative2023}, ``React Native combina as melhores partes do desenvolvimento nativo com React, uma biblioteca JavaScript de ponta para construção de interfaces de usuário''.

\textbf{React Native 0.72+ com Expo SDK 49+:}
\begin{itemize}
    \item \textit{Framework} multiplataforma para desenvolvimento nativo;
    \item Expo Router para navegação baseada em arquivos;
    \item Renderização otimizada de SVGs para o atlas anatômico;
    \item Compatibilidade total com Android 8+ e iOS 12+.
\end{itemize}

\textbf{Atlas Anatômico Digital:}
\begin{itemize}
    \item 15 visualizações vetoriais (SVG) de alta fidelidade;
    \item Organizadas por meridianos e vistas (frontal e posterior);
    \item Zoom sem perda de qualidade;
    \item Interação precisa (toque) em áreas pequenas;
    \item Carga sob demanda das imagens para otimizar memória.
\end{itemize}

\subsection{Backend API (Java + Spring Boot)}

Segundo \citeonline{walls2016}, ``Spring Boot muda a forma como desenvolvemos aplicações Spring, oferecendo configuração automática e eliminando a necessidade de configurações XML complexas''.

\textbf{Java 17 LTS e Spring Boot 3.2+:}
\begin{itemize}
    \item Linguagem robusta, fortemente tipada e de alta performance;
    \item \textit{Framework} que simplifica o desenvolvimento de aplicações Java;
    \item Arquitetura baseada em injeção de dependência e inversão de controle;
    \item Integração nativa com serviços Google Cloud.
\end{itemize}

\textbf{Spring AI - Integração com Inteligência Artificial:}

Segundo a documentação oficial \cite{springai2024}, Spring AI é o \textit{framework} oficial do ecossistema Spring para integração com modelos de linguagem de grande escala (LLMs).

\begin{itemize}
    \item \textit{Framework} oficial do Spring para integração com modelos de IA;
    \item Suporte nativo a múltiplos provedores de LLM;
    \item Abstração unificada para chamadas de API de IA;
    \item Gerenciamento de \textit{prompts} e contexto.
\end{itemize}

\textbf{Spring Security + Firebase Auth:}

Segundo a documentação do Firebase \cite{firebase2024}, o Firebase Authentication fornece serviços de backend, SDKs fáceis de usar e bibliotecas de UI prontas para autenticar usuários em aplicativos.

\begin{itemize}
    \item Integração com Firebase Authentication para gestão de identidades;
    \item Validação de \textit{tokens} JWT no \textit{backend};
    \item Controle de acesso baseado em \textit{roles} (RBAC);
    \item Suporte a autenticação biométrica no dispositivo móvel.
\end{itemize}

\subsection{Banco de Dados}

\textbf{SQLite (Mobile - Local):}
\begin{itemize}
    \item Banco relacional embarcado de alta performance;
    \item Zero configuração e manutenção;
    \item ACID \textit{compliance} para integridade dos dados;
    \item Tamanho reduzido ideal para mobile.
\end{itemize}

\textbf{Google Cloud Firestore (Backend - Remoto):}
\begin{itemize}
    \item Banco de dados NoSQL flexível, escalável e de alta performance;
    \item Sincronização em tempo real e suporte \textit{offline} nativo;
    \item Modelo de dados baseado em documentos e coleções;
    \item Integração perfeita com o ecossistema Firebase e Google Cloud.
\end{itemize}

\section{INTELIGÊNCIA ARTIFICIAL E ASSISTÊNCIA CLÍNICA} \label{sec:ia}

Diferente de abordagens tradicionais baseadas apenas em palavras-chave, o \textit{Appunture} implementa um sistema de RAG (\textit{Retrieval-Augmented Generation}) no \textit{backend}, oferecendo assistência clínica contextualizada e precisa.

\subsection{Arquitetura RAG (Retrieval-Augmented Generation)}

A arquitetura RAG (\textit{Retrieval-Augmented Generation}) foi introduzida por \citeonline{lewis2020} como uma abordagem que combina a recuperação de informações com a geração de texto, melhorando significativamente a precisão e confiabilidade das respostas em tarefas de conhecimento intensivo.

\textbf{Modelo de IA:}

O Google Gemini 1.5 Flash \cite{gemini2024} é um modelo de linguagem multimodal otimizado para respostas rápidas e contextualizadas.

\begin{itemize}
    \item Google Gemini 1.5 Flash via Spring AI;
    \item Modelo otimizado para respostas rápidas e contextualizadas;
    \item Integração nativa com o ecossistema Google Cloud.
\end{itemize}

\textbf{Funcionamento do Sistema:}
\begin{enumerate}
    \item O sistema intercepta a pergunta do usuário;
    \item Busca contexto relevante na base de dados (pontos e sintomas relacionados);
    \item Envia um \textit{prompt} enriquecido para o modelo de IA;
    \item Retorna respostas clinicamente precisas, baseadas nos dados validados do sistema.
\end{enumerate}

\textbf{Benefícios da Abordagem RAG:}
\begin{itemize}
    \item Respostas baseadas em dados validados do sistema;
    \item Redução significativa de alucinações do modelo;
    \item Maior confiabilidade das informações clínicas;
    \item Contextualização específica para acupuntura;
    \item Rastreabilidade das fontes utilizadas nas respostas.
\end{itemize}

\subsection{Integração com Spring AI}

O \autoref{qua:springai} apresenta um exemplo simplificado da integração com Spring AI para o assistente de acupuntura:

\begin{quadro}[htb]
\caption{Exemplo de integração Spring AI}
\label{qua:springai}
\centering
\footnotesize
\begin{verbatim}
@Service
public class AcupunctureAssistantService {
    
    private final ChatClient chatClient;
    private final PointRepository pointRepository;
    
    public String getClinicAssistance(String userQuestion) {
        // Busca contexto relevante na base de dados
        List<Point> relevantPoints = 
            pointRepository.findBySymptoms(userQuestion);
        
        // Monta prompt enriquecido com contexto
        String enrichedPrompt = buildPromptWithContext(
            userQuestion, relevantPoints);
        
        // Chama o modelo de IA via Spring AI
        return chatClient.prompt()
            .user(enrichedPrompt)
            .call()
            .content();
    }
}
\end{verbatim}
\fonte{Os autores (2025)}
\end{quadro}

\section{FERRAMENTAS DE DESENVOLVIMENTO E MAPEAMENTO} \label{sec:ferramentas}

Para garantir a precisão da localização dos pontos de acupuntura sobre o atlas anatômico digital, foi desenvolvida uma ferramenta interna denominada ``\textit{Point Mapper}''.

\subsection{Point Mapper - Ferramenta de Mapeamento}

\textbf{Características:}
\begin{itemize}
    \item Interface Web para visualização dos 15 SVGs do atlas;
    \item Sistema de coordenadas percentuais para garantir responsividade;
    \item Funcionalidades de zoom e pan para precisão no posicionamento;
    \item Exportação de dados em formato JSON para integração com o banco;
    \item Validação cruzada com descrições anatômicas dos pontos.
\end{itemize}

\textbf{Sistema de Coordenadas:}
\begin{itemize}
    \item Utilização de coordenadas percentuais (0-100\%) ao invés de pixels;
    \item Garantia de responsividade em diferentes tamanhos de tela;
    \item Precisão milimétrica na localização dos pontos;
    \item Compatibilidade com diferentes resoluções de dispositivos.
\end{itemize}

\subsection{Scripts de Migração e Integração}

\textbf{Scripts Python:}
\begin{itemize}
    \item Migração de dados entre formatos (JSON, CSV, SQL);
    \item Validação de integridade dos dados mapeados;
    \item Geração automática de \textit{seeds} para o banco de dados;
    \item Integração com a ferramenta de mapeamento.
\end{itemize}

\section{TECNOLOGIAS DE SUPORTE} \label{sec:suporte}

\subsection{Gerenciamento de Estado (Mobile)}

\textbf{Zustand:}
\begin{itemize}
    \item Biblioteca de estado minimalista e performática;
    \item \textit{TypeScript-first} com inferência automática;
    \item \textit{Persist middleware} para sincronização com \textit{storage} local.
\end{itemize}

\textbf{React Query:}
\begin{itemize}
    \item \textit{Cache} inteligente de requisições HTTP;
    \item Invalidação automática e \textit{refetch} estratégico;
    \item \textit{Offline support} com \textit{background sync}.
\end{itemize}

\subsection{Interface Web Administrativa (React)}

\textbf{React 18+ com TypeScript:}
\begin{itemize}
    \item Biblioteca para interfaces reativas e componentizadas;
    \item \textit{Hooks} para gerenciamento de estado eficiente;
    \item \textit{Component library} customizada baseada em Material-UI.
\end{itemize}

\textbf{Funcionalidades Administrativas:}
\begin{itemize}
    \item CRUD completo de pontos de acupuntura;
    \item Gestão de sintomas e relações terapêuticas;
    \item Gestão da base de conhecimento que alimenta a IA;
    \item \textit{Dashboard} com métricas de uso;
    \item Sistema de permissões granular.
\end{itemize}

% ----------------------------------------------------------
% Capítulo 5 - METODOLOGIA
% ----------------------------------------------------------
\chapter{METODOLOGIA} \label{cha:metodologia}

Este capítulo apresenta as metodologias e ferramentas utilizadas para o desenvolvimento do aplicativo \textit{Appunture}, abrangendo desde as práticas de gerenciamento de projeto até as técnicas de modelagem e as tecnologias empregadas na implementação da solução.

\section{METODOLOGIA DE DESENVOLVIMENTO} \label{sec:metodologia_dev}

``A engenharia de software é uma disciplina tecnológica que se preocupa com todos os aspectos da produção de software'' \cite{pressman2016}.

Seguindo os princípios da engenharia de software moderna, buscou-se conciliar qualidade técnica e valor prático, conforme destaca \citeonline{pressman2016}, ao afirmar que o software deve ser desenvolvido de forma disciplinada, mas adaptável às necessidades do projeto. A metodologia permitiu ajustes rápidos com base em testes reais, além de facilitar a manutenção e escalabilidade da aplicação.

\subsection{Scrum}

Para a gestão do desenvolvimento do aplicativo \textit{Appunture}, optou-se pela utilização do Scrum, um \textit{framework} ágil amplamente adotado em projetos de software devido à sua flexibilidade e capacidade de adaptação a contextos complexos.

Conforme \citeonline{schwaber2020}, o Scrum é ``um \textit{framework} leve que ajuda pessoas, times e organizações a gerar valor por meio de soluções adaptativas para problemas complexos''. Ele permite organizar o trabalho de maneira iterativa e incremental, promovendo entregas frequentes e contínua melhoria do produto.

Durante a construção do \textit{Appunture}, adotaram-se ciclos de desenvolvimento quinzenais (\textit{Sprints}), nos quais o time planejava, desenvolvia e entregava partes funcionais da aplicação. Essa prática segue o conceito apresentado por \citeonline{sutherland2014}, segundo o qual ``a definição de objetivos sequenciais que devem ser concluídos em um período definido'' favorece foco, previsibilidade e evolução do produto.

Cada ciclo se iniciava com uma \textit{Sprint Planning}, onde era definido o \textit{backlog} da \textit{sprint} --- uma lista priorizada de funcionalidades e correções a serem entregues. O time se organizava de forma autogerenciada, decidindo coletivamente ``quem faria o quê, como e quando'', respeitando a autonomia e a responsabilidade individual de cada membro da equipe \cite{schwaber2020}.

No contexto deste projeto foram definidos os seguintes papéis:

\begin{itemize}
    \item \textbf{Product Owner}: Bruno Brugnerotto de Lara
    \item \textbf{Scrum Master}: Gabriel Francelino Voidaleski
    \item \textbf{Desenvolvedor}: Pedro Henrique Lopes
\end{itemize}

Os eventos de \textit{Daily Scrum} foram adaptados para ocorrer semanalmente, considerando a disponibilidade da equipe envolvida no projeto. Já o evento de \textit{Sprint Retrospective} foi mantido ao final de cada \textit{sprint}, promovendo momentos de reflexão e melhoria contínua do processo de desenvolvimento.

\section{MODELAGEM DO PROJETO} \label{sec:modelagem}

A modelagem é uma etapa fundamental no desenvolvimento de sistemas, pois proporciona uma representação visual da estrutura e do comportamento da aplicação. Segundo \citeonline{booch2005}, empresas que entregam sistemas de qualidade e atendem às necessidades dos usuários são aquelas que compreendem a importância da modelagem desde os estágios iniciais do projeto.

No desenvolvimento do aplicativo \textit{Appunture}, a modelagem permitiu visualizar de forma macro os elementos do sistema, facilitando a compreensão das funcionalidades e dos fluxos de dados. Como destacam os autores \citeonline{booch2005}, a construção de diagramas é essencial em sistemas orientados a objetos, pois oferece clareza sobre a estrutura e o funcionamento do software. Para isso, foi utilizada a UML (\textit{Unified Modeling Language}) como linguagem padrão, uma vez que, de acordo com os mesmos autores, trata-se de uma abordagem consolidada e amplamente adotada para representar graficamente os componentes e relações de um sistema.

\subsection{Diagrama de Casos de Uso}

Enquanto o diagrama de classes oferece uma perspectiva estrutural do sistema, representando seus componentes de forma estática, ele não contempla a dinâmica das interações entre os elementos. Segundo \citeonline{pressman2011}, é por meio dos diagramas de casos de uso que se visualiza como os usuários interagem com as funcionalidades do sistema, sendo uma ferramenta fundamental na modelagem orientada a objetos.

No projeto \textit{Appunture}, o diagrama de casos de uso foi elaborado com o objetivo de representar os diferentes perfis de usuários --- como profissionais da saúde, administradores e desenvolvedores --- e suas respectivas interações com o aplicativo.

\subsection{Histórias de Usuário}

A prática de expressar requisitos por meio de histórias surgiu no contexto do desenvolvimento ágil, mais especificamente no processo \textit{Extreme Programming}, conforme introduzido por \citeonline{beck2001}. As histórias de usuário consistem em descrições concisas de funcionalidades escritas sob a ótica do próprio usuário, facilitando a comunicação entre a equipe de desenvolvimento e os \textit{stakeholders}.

Segundo \citeonline{bernardo2014}, a principal característica dessas histórias está na simplicidade e objetividade, permitindo que necessidades do cliente sejam descritas de forma clara e centrada em valor. No desenvolvimento do aplicativo \textit{Appunture}, as funcionalidades foram documentadas com base nessa abordagem, sendo cada história composta por:

\begin{itemize}
    \item Uma descrição no formato: SENDO (perfil), QUERO (função) PARA (objetivo);
    \item Seus respectivos critérios de aceitação.
\end{itemize}

Conforme \citeonline{cohn2004}, os critérios de aceitação ``definem os limites de uma história de usuário, sendo usados para confirmar quando uma história está concluída e funcionando como esperado''.

\subsubsection{Histórias de Usuário do Projeto}

\begin{enumerate}
    \item Como usuário, desejo consultar pontos de acupuntura interativamente em um corpo humano, para localizar facilmente a região desejada.
    \begin{itemize}
        \item Critérios de aceitação: Deve ser possível tocar/clicar no ponto anatômico e acessar seus detalhes. Deve haver destaque visual nos pontos ativos do corpo.
    \end{itemize}
    
    \item Como usuário, desejo buscar pontos por nome, localização, meridiano ou função terapêutica, para encontrar informações de forma rápida.
    \begin{itemize}
        \item Critérios de aceitação: Deve aceitar buscas por texto. A pesquisa deve filtrar e exibir resultados dinâmicos.
    \end{itemize}
    
    \item Como usuário, desejo filtrar pontos com base em sintomas que estou estudando (ex: ``ansiedade'' + ``insônia''), para encontrar pontos relacionados.
    \begin{itemize}
        \item Critérios de aceitação: Deve aceitar múltiplos sintomas. Deve retornar pontos relacionados por frequência de associação na literatura.
    \end{itemize}
    
    \item Como usuário, desejo visualizar informações detalhadas sobre cada ponto, incluindo profundidade, punção, características especiais e funções.
    \begin{itemize}
        \item Critérios de aceitação: Todas as informações devem estar organizadas em seções. Imagens e descrições devem estar disponíveis \textit{offline}.
    \end{itemize}
    
    \item Como usuário, desejo usar o app sem internet, para ter acesso ao conteúdo mesmo em locais sem sinal.
    \begin{itemize}
        \item Critérios de aceitação: O app deve funcionar com banco local. Ao conectar-se, os dados são sincronizados com o \textit{backend}.
    \end{itemize}
    
    \item Como usuário, desejo registrar uma conta e fazer login, para salvar minhas preferências e histórico.
    \begin{itemize}
        \item Critérios de aceitação: Deve permitir login com email/senha. O app deve diferenciar usuários por tipo (ex: estudante, profissional etc.).
    \end{itemize}
    
    \item Como usuário, desejo usar um assistente inteligente para descrever sintomas com linguagem natural, para receber informações contextualizadas sobre pontos relacionados.
    \begin{itemize}
        \item Critérios de aceitação: Quando \textit{online}, deve utilizar IA Generativa (Spring AI + Google Gemini) para informações contextualizadas. Respostas devem ser baseadas na base de dados técnica do sistema.
    \end{itemize}
    
    \item Como administrador, desejo gerenciar os dados do sistema através dos consoles Firebase e Google Cloud, para manter o banco de dados atualizado.
    \begin{itemize}
        \item Critérios de aceitação: Acesso aos consoles Firebase (Firestore, Auth, Storage) e GCP. Operações de CRUD nos documentos.
    \end{itemize}
    
    \item Como administrador, desejo mapear as coordenadas dos pontos no atlas, para permitir a visualização correta pelos usuários.
    \begin{itemize}
        \item Critérios de aceitação: Ferramenta Point Mapper funcional. Exportação de coordenadas em JSON. Prévia visual do posicionamento.
    \end{itemize}
\end{enumerate}

\subsection{Diagrama Físico do Banco de Dados}

O diagrama físico de banco de dados é uma representação detalhada das tabelas que compõem o sistema, incluindo seus atributos, tipos de dados, chaves primárias e estrangeiras. Conforme observado por \citeonline{molina2009}, esse tipo de modelagem visa representar a estrutura real de armazenamento da informação, sendo vital para a implementação correta de um banco de dados relacional.

Para o \textit{Appunture}, o diagrama físico foi desenvolvido com base no diagrama de classes UML, refletindo tanto o banco de dados local (SQLite), utilizado para acesso \textit{offline} no dispositivo móvel, quanto o banco remoto (Firestore), hospedado no \textit{backend} acessado via API.

\section{PRODUCT BACKLOG} \label{sec:backlog}

O \autoref{qua:backlog} apresenta o \textit{Product Backlog} do projeto \textit{Appunture}, com os itens priorizados e suas respectivas estimativas em Pontos de Função (PF).

\begin{quadro}[htb]
\caption{Product Backlog do Appunture}
\label{qua:backlog}
\centering
\footnotesize
\begin{tabular}{|c|p{5cm}|c|c|c|}
\hline
\textbf{ID} & \textbf{Item do Backlog} & \textbf{Plataforma} & \textbf{PF} & \textbf{Prioridade} \\
\hline
1 & Setup do projeto e bibliotecas iniciais & Todas & 3 & Alta \\
\hline
2 & Sistema de navegação (tabs, stack, rotas protegidas) & Mobile/Web & 5 & Alta \\
\hline
3 & Tela de login e registro com autenticação JWT & Todas & 5 & Alta \\
\hline
4 & Tela de busca com filtro por sintomas ou nome de ponto & Mobile/Web & 8 & Alta \\
\hline
5 & Assistente IA com RAG (Spring AI + Gemini) & Mobile/Backend & 6 & Alta \\
\hline
6 & Tela de resultados com lista de pontos filtrados & Mobile/Web & 8 & Alta \\
\hline
7 & Tela do corpo humano com hotspots interativos (SVG) & Mobile & 14 & Alta \\
\hline
8 & Tela de detalhes de um ponto & Mobile/Web & 6 & Média \\
\hline
9 & Funcionalidade de favoritar pontos & Mobile & 3 & Média \\
\hline
10 & Anotações personalizadas por ponto & Mobile & 3 & Média \\
\hline
11 & Banco de dados local (SQLite/MMKV) + versionamento & Mobile & 4 & Alta \\
\hline
12 & Sistema de sincronização (favoritos, anotações) & Mobile/Backend & 6 & Alta \\
\hline
13 & Backend RESTful (Java Firestore + JWT) & Backend & 5 & Alta \\
\hline
14 & APIs REST para gestão de pontos e sintomas & Backend & 6 & Alta \\
\hline
15 & Ferramenta Point Mapper para coordenadas & Web & 3 & Média \\
\hline
16 & Controle de acesso por perfil (user/admin) & Backend & 3 & Alta \\
\hline
17 & Deploy e testes finais (Web Mobile) & Todas & 5 & Média \\
\hline
\end{tabular}
\fonte{Os autores (2025)}
\end{quadro}

\section{PLANEJAMENTO DE SPRINTS} \label{sec:sprints}

O desenvolvimento do \textit{Appunture} foi estruturado com base na metodologia SCRUM, dividindo o projeto em 8 \textit{sprints} quinzenais, ao longo de aproximadamente 15 semanas de trabalho efetivo. Cada \textit{sprint} contempla entregas incrementais de funcionalidades, priorizando valor de uso e estrutura progressiva do sistema.

A equipe de desenvolvimento é composta por três integrantes, com disponibilidade média de 4,5 horas semanais por pessoa, totalizando 13,5 horas por semana, ou 27 horas por \textit{sprint}. Dessa forma, o tempo estimado total do projeto é de aproximadamente 202,5 horas.

\subsection{Estimativa Geral por Pontos de Função (PF)}

O \autoref{qua:estimativa} apresenta a distribuição das estimativas por funcionalidade:

\begin{quadro}[htb]
\caption{Estimativa por Pontos de Função}
\label{qua:estimativa}
\centering
\begin{tabular}{|p{10cm}|c|}
\hline
\textbf{Funcionalidade} & \textbf{Estimativa (PF)} \\
\hline
Tela de busca com filtro por sintomas/finalidade & 8 \\
\hline
Tela de resultados com imagens & 8 \\
\hline
Tela do corpo humano com hotspots interativos (SVG) & 14 \\
\hline
Tela de detalhes de um ponto & 6 \\
\hline
Funcionalidade de favoritos e anotações & 6 \\
\hline
Backend com autenticação e CRUD de pontos/sintomas & 8 \\
\hline
Ferramenta Point Mapper para mapeamento de coordenadas & 4 \\
\hline
Sistema de sincronização (SQLite $\leftrightarrow$ Backend) & 6 \\
\hline
Sistema de navegação e layout geral & 5 \\
\hline
Deploy, testes finais e refinos & 5 \\
\hline
Integração do Assistente IA com RAG & 6 \\
\hline
\textbf{Total Estimado} & \textbf{72 PF} \\
\hline
\end{tabular}
\fonte{Os autores (2025)}
\end{quadro}

\subsection{Cálculo de Esforço}

\begin{itemize}
    \item \textbf{Tempo disponível por sprint}: 3 pessoas $\times$ 4,5 h/semana $\times$ 2 semanas = 27 horas/sprint
    \item \textbf{Tempo total estimado}: 15 semanas $\times$ 13,5 h/semana = 202,5 horas
    \item \textbf{Total de Pontos de Função estimado}: 72 PF, distribuídos ao longo de 7 sprints + 1 sprint de buffer
\end{itemize}

O \autoref{qua:sprints} apresenta o cronograma de \textit{sprints}:

\begin{quadro}[htb]
\caption{Cronograma de Sprints}
\label{qua:sprints}
\centering
\footnotesize
\begin{tabular}{|c|c|c|p{7cm}|}
\hline
\textbf{Sprint} & \textbf{Semana} & \textbf{PF} & \textbf{Entregas Planejadas} \\
\hline
Sprint 1 & 0--1 & 4 & Setup do projeto, autenticação, rotas iniciais \\
\hline
Sprint 2 & 2--3 & 8 & Login/registro, persistência de token, navegação protegida \\
\hline
Sprint 3 & 4--5 & 10 & Busca avançada, integração IA, exibição de resultados \\
\hline
Sprint 4 & 6--7 & 10 & Tela de resultados com imagens + início da tela de detalhes \\
\hline
Sprint 5 & 8--9 & 10 & Tela de corpo humano com SVG interativo \\
\hline
Sprint 6 & 10--11 & 10 & Favoritos, anotações, sincronização offline e online \\
\hline
Sprint 7 & 12--13 & 12 & Backend completo + ferramenta Point Mapper \\
\hline
Sprint 8 & 14--15 & 0 & Testes finais, deploy mobile/web \\
\hline
\end{tabular}
\fonte{Os autores (2025)}
\end{quadro}

\section{FERRAMENTAS DE DESENVOLVIMENTO} \label{sec:ferramentas}

Nesta seção estão descritas as diferentes ferramentas que, juntas, viabilizaram a construção e o desenvolvimento do aplicativo \textit{Appunture}.

\subsection{React Native}

React Native é um \textit{framework} de desenvolvimento móvel criado pelo Facebook em 2015, que permite criar aplicações nativas para iOS e Android usando JavaScript e React. Segundo a documentação oficial \cite{reactnative2023}, ``React Native combina as melhores partes do desenvolvimento nativo com React, uma biblioteca JavaScript de ponta para construção de interfaces de usuário''.

O \textit{framework} foi escolhido para este projeto por permitir desenvolvimento multiplataforma com alta performance, reduzindo significativamente o tempo de desenvolvimento e manutenção.

\subsection{Expo}

Expo é uma plataforma de código aberto para criar aplicações React Native universais. Segundo a documentação \cite{expo2023}, ``Expo é um conjunto de ferramentas e serviços construídos em torno do React Native que ajuda você a desenvolver, construir, implementar e iterar rapidamente em aplicações iOS, Android e web''.

A escolha do Expo se deu pela facilidade de configuração do ambiente de desenvolvimento, capacidade de testar aplicações em dispositivos reais através do Expo Go, e pelos serviços integrados como EAS Build para compilação na nuvem.

\subsection{Java}

Java é uma linguagem de programação orientada a objetos, robusta e segura, amplamente utilizada no desenvolvimento de sistemas corporativos. Segundo \citeonline{deitel2017}, ``Java é a linguagem de programação mais popular do mundo, permitindo o desenvolvimento de aplicações seguras, portáveis e de alto desempenho''.

Para o \textit{backend} do \textit{Appunture}, Java foi escolhido por sua tipagem estática, que previne erros em tempo de compilação, e por sua robustez no tratamento de regras de negócio complexas e integrações com serviços em nuvem.

\subsection{Spring Boot}

Spring Boot é um \textit{framework} que facilita a criação de aplicações baseadas em Spring autônomas e de nível de produção. Segundo \citeonline{walls2016}, ``Spring Boot muda a forma como desenvolvemos aplicações Spring, oferecendo configuração automática e eliminando a necessidade de configurações XML complexas''.

O Spring Boot foi adotado no projeto por sua capacidade de acelerar o desenvolvimento, integração nativa com serviços do Google Cloud (como Firestore e Firebase), e por fornecer recursos prontos para segurança (Spring Security) e monitoramento.

\subsection{Google Cloud Firestore}

O Firestore é um banco de dados NoSQL flexível e escalável para desenvolvimento móvel, web e de servidor, oferecido pelo Google Cloud Platform. Segundo a documentação oficial \cite{google2023}, ``o Firestore mantém seus dados em sincronia entre aplicativos clientes através de ouvintes em tempo real e oferece suporte \textit{offline} para dispositivos móveis e web''.

Para o \textit{Appunture}, o Firestore foi escolhido por sua estrutura de documentos flexível, ideal para armazenar dados hierárquicos de pontos de acupuntura, e por sua integração nativa com o ecossistema Firebase.

\subsection{SQLite}

SQLite é um mecanismo de banco de dados SQL embutido, que não requer servidor separado ou configuração. Segundo \citeonline{kreibich2010}, ``SQLite é uma biblioteca de software que implementa um mecanismo de banco de dados SQL transacional, autônomo, sem servidor e de configuração zero''.

No aplicativo móvel \textit{Appunture}, SQLite foi implementado para garantir funcionamento \textit{offline} completo, armazenando localmente todos os pontos de acupuntura, sintomas e dados do usuário.

% ----------------------------------------------------------
% Capítulo 6 - APRESENTAÇÃO DO SISTEMA
% ----------------------------------------------------------
\chapter{APRESENTAÇÃO DO SISTEMA} \label{cha:apresentacao}

Neste capítulo o \textit{Appunture} é apresentado, pontuando os aspectos técnicos e as funcionalidades que constituem a aplicação.

\section{ARQUITETURA DO SISTEMA} \label{sec:arq_sistema}

O \textit{Appunture} foi desenvolvido seguindo uma arquitetura híbrida e modular que integra funcionalidades \textit{offline} e \textit{online} para garantir alta disponibilidade, desempenho e usabilidade mesmo em ambientes com conexão limitada ou instável. A arquitetura se divide em três módulos principais: aplicativo móvel (React Native/Expo), \textit{backend} (Java + Spring Boot) e painel administrativo web (React).

\subsection{Arquitetura Híbrida (Offline + Online)}

No núcleo do aplicativo móvel, o \textit{Appunture} utiliza um banco de dados local SQLite para armazenar informações clínicas essenciais, incluindo os 361 pontos dos meridianos principais, pontos extras, sintomas, relações terapêuticas, favoritos e anotações do usuário. Esta abordagem garante que o aplicativo funcione integralmente em modo \textit{offline}, permitindo consultas rápidas e uso em campo, sem depender da conexão com a internet.

Quando a conexão estiver disponível, o sistema realiza sincronizações automáticas e seguras com um \textit{backend} remoto, desenvolvido em Java com Spring Boot, que hospeda o banco de dados Firestore centralizado. Esta sincronização atualiza dados clínicos, envia registros do usuário e mantém o histórico sempre preservado e atualizado.

\subsection{Comunicação e Sincronização}

A comunicação entre o app e o \textit{backend} é realizada via API RESTful, utilizando autenticação Firebase/JWT para controle seguro de acesso. O \textit{backend} disponibiliza \textit{endpoints} públicos para consulta de dados clínicos e \textit{endpoints} privados para operações que envolvem dados do usuário, como sincronização e autenticação.

A sincronização implementa um mecanismo de controle de versões dos dados clínicos para evitar \textit{downloads} desnecessários e garantir que o aplicativo trabalhe sempre com as informações mais recentes. O processo ocorre automaticamente em \textit{background} sempre que uma conexão de internet é detectada.

\subsection{Assistente de IA para Busca Inteligente}

Para viabilizar a busca inteligente por sintomas, o aplicativo utiliza um assistente de inteligência artificial integrado ao \textit{backend}. Esta funcionalidade requer conexão com a internet para processamento.

\textbf{Arquitetura RAG (Retrieval-Augmented Generation):}
O sistema utiliza a arquitetura RAG integrada ao \textit{backend} Java. O assistente inteligente, alimentado pelo Google Gemini 1.5 Flash via Spring AI, interpreta perguntas do usuário sobre sintomas e tratamentos, fornecendo respostas contextualizadas com a base de dados técnica do sistema. Esta abordagem reduz alucinações do modelo e aumenta a confiabilidade das sugestões clínicas.

\textbf{Navegação Offline:}
Quando sem conexão, o usuário pode navegar manualmente pelo atlas anatômico e acessar as informações dos pontos já sincronizados no banco de dados local. A busca inteligente por sintomas fica indisponível até que uma conexão seja restabelecida.

\subsection{Interface Interativa e Navegação}

A interface do app é construída para promover uma experiência intuitiva e visualmente rica, com mapas anatômicos interativos baseados em SVG, que possibilitam a interação direta com os pontos de acupuntura. A navegação é estruturada para facilitar o acesso rápido às principais funcionalidades, como busca, visualização detalhada, favoritos e anotações, promovendo fluidez e usabilidade.

\subsection{Backend e Web Admin}

O \textit{backend} centraliza a gestão dos dados oficiais, incluindo o cadastro e atualização dos pontos, sintomas e suas relações. Uma interface administrativa web, desenvolvida em React, permite que gestores e especialistas mantenham o conteúdo sempre atualizado, garantindo a qualidade e a segurança das informações disponíveis no app.

\section{AUTENTICAÇÃO E SEGURANÇA} \label{sec:autenticacao}

Para garantir a segurança dos usuários e a integridade dos dados, o sistema implementa autenticação via Firebase Authentication, integrada ao Spring Security no \textit{backend}.

\textbf{Métodos de Autenticação:}
\begin{itemize}
    \item Login com email e senha;
    \item Autenticação biométrica no dispositivo móvel (impressão digital ou Face ID);
    \item Validação de \textit{tokens} JWT no \textit{backend}.
\end{itemize}

Durante o processo de login, o Firebase Authentication gerencia a validação das credenciais e gera um \textit{token} JWT assinado. Este \textit{token} é utilizado para manter a sessão do usuário e validar o acesso aos recursos protegidos da API.

A sessão é gerenciada pelo aplicativo móvel, que armazena o \textit{token} de forma segura no \textit{storage} local do dispositivo. Desta forma, o usuário permanece logado mesmo após fechar e reabrir o aplicativo. A cada interação que requer comunicação com o \textit{backend}, o \textit{token} é enviado via cabeçalho HTTP para validação e controle de acesso baseado em \textit{roles} (RBAC).

\section{FUNCIONALIDADES DO SISTEMA} \label{sec:funcionalidades}

\subsection{Tela Inicial e Onboarding}

Quando o usuário inicia a utilização do aplicativo pela primeira vez, é apresentada uma sequência de telas de apresentação (\textit{onboarding}) que explicam as principais funcionalidades e benefícios do \textit{Appunture}. Esta experiência introdutória inclui informações sobre o uso \textit{offline}, a busca inteligente, os mapas anatômicos interativos e os diferentes perfis de usuário disponíveis.

Na tela inicial, o usuário pode escolher entre realizar o processo de cadastro ou direcionar-se diretamente para a tela de login. O design foi pensado para transmitir profissionalismo e confiança, utilizando a identidade visual do projeto com cores que remetem à medicina tradicional chinesa e à modernidade tecnológica.

\subsection{Cadastro e Perfis de Usuário}

A tela de cadastro permite que diferentes tipos de profissionais da saúde se registrem no sistema, incluindo médicos, fisioterapeutas, terapeutas, enfermeiros, estudantes e outros profissionais relacionados. É fundamental destacar que durante o cadastro, o usuário deve informar seu perfil profissional, pois esta informação é utilizada para adaptar o conteúdo apresentado conforme o nível de formação e os limites éticos e legais de cada profissão.

O sistema implementa diferentes níveis de acesso baseados no perfil declarado:

\begin{itemize}
    \item \textbf{Médicos e especialistas}: acesso completo a todas as informações técnicas;
    \item \textbf{Profissionais de saúde}: acesso adaptado conforme a regulamentação da categoria;
    \item \textbf{Estudantes}: acesso educativo com conteúdos didáticos;
    \item \textbf{Outros profissionais}: acesso personalizado conforme a regulamentação da categoria.
\end{itemize}

\subsection{Tela Principal e Navegação}

A tela principal do aplicativo apresenta as principais funcionalidades de forma organizada e acessível. A navegação é estruturada por abas (\textit{tabs}) na parte inferior da tela, permitindo acesso rápido às seções principais:

\begin{itemize}
    \item \textbf{Home}: Tela inicial com acesso rápido às funcionalidades principais;
    \item \textbf{Busca}: Sistema de busca inteligente por sintomas ou nome de ponto;
    \item \textbf{Corpo}: Mapa anatômico interativo;
    \item \textbf{Favoritos}: Lista de pontos favoritados pelo usuário;
    \item \textbf{Perfil}: Configurações do usuário e preferências.
\end{itemize}

\subsection{Mapa Anatômico Interativo}

Uma das funcionalidades mais destacadas do \textit{Appunture} é o mapa anatômico interativo, implementado com tecnologia SVG (\textit{Scalable Vector Graphics}). Este recurso permite que o usuário visualize o corpo humano de forma detalhada e interaja diretamente com os pontos de acupuntura mapeados.

Cada ponto é representado por um marcador visual que, ao ser tocado, exibe informações básicas sobre o ponto em um \textit{tooltip}. O usuário pode então optar por acessar a tela de detalhes completa do ponto selecionado.

O mapa oferece 15 visualizações vetoriais de alta fidelidade, organizadas por meridianos:

\begin{itemize}
    \item Vista frontal do corpo completo;
    \item Vista posterior do corpo completo;
    \item Vistas específicas por meridiano (12 meridianos principais);
    \item Vasos extraordinários (Ren Mai e Du Mai).
\end{itemize}

\subsection{Detalhes dos Pontos de Acupuntura}

A tela de detalhes apresenta informações completas sobre cada ponto de acupuntura, organizadas em seções para facilitar a consulta:

\begin{itemize}
    \item \textbf{Identificação}: Nome em português, nome em chinês (pinyin), código internacional;
    \item \textbf{Localização}: Descrição anatômica precisa da localização do ponto;
    \item \textbf{Meridiano}: Identificação do meridiano ao qual o ponto pertence;
    \item \textbf{Profundidade}: Indicação da profundidade de punção recomendada;
    \item \textbf{Indicações}: Lista de condições clínicas para as quais o ponto é indicado;
    \item \textbf{Contraindicações}: Alertas sobre situações em que o ponto não deve ser utilizado;
    \item \textbf{Funções energéticas}: Descrição das funções segundo a MTC;
    \item \textbf{Características especiais}: Pontos de alarme, pontos de assentimento, etc.
\end{itemize}

\subsection{Sistema de Busca Inteligente}

O sistema de busca do \textit{Appunture} utiliza inteligência artificial para oferecer resultados relevantes mesmo quando o usuário não conhece o nome exato do ponto que procura. A busca pode ser realizada por:

\begin{itemize}
    \item Nome do ponto (em português ou pinyin);
    \item Localização anatômica;
    \item Sintomas ou condições clínicas;
    \item Meridiano;
    \item Funções terapêuticas.
\end{itemize}

O assistente de IA interpreta a intenção do usuário e retorna os pontos mais relevantes, mesmo com erros de digitação ou variações na escrita. Esta funcionalidade requer conexão com a internet.

\subsection{Assistente Inteligente com IA Generativa}

O assistente inteligente integrado ao \textit{Appunture} permite que o usuário descreva sintomas em linguagem natural e receba sugestões de pontos de acupuntura baseadas em evidências. Esta funcionalidade utiliza a arquitetura RAG (\textit{Retrieval-Augmented Generation}):

\textbf{Funcionamento do Sistema:}
\begin{enumerate}
    \item O sistema intercepta a pergunta do usuário;
    \item Busca contexto relevante na base de dados (pontos e sintomas relacionados);
    \item Envia um \textit{prompt} enriquecido para o modelo Google Gemini 1.5 Flash via Spring AI;
    \item Retorna respostas clinicamente precisas, baseadas nos dados validados do sistema.
\end{enumerate}

\textbf{Benefícios da Abordagem:}
\begin{itemize}
    \item Respostas contextualizadas com a base de dados técnica;
    \item Redução significativa de alucinações do modelo;
    \item Maior confiabilidade das informações clínicas;
    \item Rastreabilidade das fontes utilizadas nas respostas.
\end{itemize}

\subsection{Sistema de Favoritos e Anotações Pessoais}

O \textit{Appunture} permite que o usuário salve pontos de acupuntura como favoritos para acesso rápido posterior. Além disso, é possível adicionar anotações pessoais a cada ponto, registrando observações clínicas, experiências ou lembretes.

Estas informações são armazenadas localmente no dispositivo e sincronizadas com o \textit{backend} quando há conexão disponível, garantindo que o usuário não perca seus dados ao trocar de dispositivo.

\subsection{Modo Auricular (Auriculoterapia)}

Além do mapa corporal completo, o \textit{Appunture} oferece um modo específico para auriculoterapia, apresentando um mapa detalhado da orelha com os pontos auriculares mapeados. Esta funcionalidade segue a mesma lógica de interação do mapa corporal, permitindo visualização e consulta de informações específicas sobre cada ponto auricular.

\section{PAINEL ADMINISTRATIVO WEB} \label{sec:admin}

\subsection{Gerenciamento de Conteúdo}

O painel administrativo web permite que gestores e especialistas mantenham o conteúdo do aplicativo sempre atualizado. As principais funcionalidades incluem:

\begin{itemize}
    \item CRUD completo de pontos de acupuntura;
    \item Gestão de sintomas e suas relações com pontos;
    \item \textit{Upload} e mapeamento de coordenadas SVG para os mapas anatômicos;
    \item Gestão de meridianos e categorias;
    \item Controle de versões do conteúdo;
    \item Gestão da base de conhecimento que alimenta o assistente IA.
\end{itemize}

\subsection{Controle de Usuários e Analytics}

O painel também oferece funcionalidades de gestão de usuários e análise de uso:

\begin{itemize}
    \item Visualização de usuários cadastrados;
    \item Relatórios de uso do aplicativo;
    \item Métricas de pontos mais acessados;
    \item Análise de buscas realizadas;
    \item Gestão de permissões e perfis.
\end{itemize}

\section{EXPERIÊNCIA DO USUÁRIO E USABILIDADE} \label{sec:ux}

O desenvolvimento da interface do \textit{Appunture} seguiu princípios de design centrado no usuário, priorizando:

\begin{itemize}
    \item \textbf{Simplicidade}: Interface limpa e intuitiva, sem excesso de informações;
    \item \textbf{Consistência}: Padrões visuais e de interação consistentes em todo o aplicativo;
    \item \textbf{Feedback}: Respostas visuais claras para todas as ações do usuário;
    \item \textbf{Acessibilidade}: Suporte a diferentes tamanhos de tela e orientações;
    \item \textbf{Performance}: Carregamento rápido e transições fluidas.
\end{itemize}

\section{PROTÓTIPO DE TELAS} \label{sec:prototipo}

O fluxo de telas do aplicativo foi projetado para proporcionar uma navegação intuitiva entre as diferentes funcionalidades. O usuário pode transitar facilmente entre login, cadastro, tela inicial, busca, detalhes dos pontos, favoritos e anotações, seguindo um fluxo lógico e previsível.

% Aqui podem ser inseridas figuras dos protótipos de tela
% \figura{TELA DE LOGIN}{0.5}{fig/tela-login.png}{Os autores (2025)}{tela-login}{}{}
% \figura{TELA PRINCIPAL}{0.5}{fig/tela-principal.png}{Os autores (2025)}{tela-principal}{}{}
% \figura{MAPA ANATÔMICO}{0.5}{fig/mapa-anatomico.png}{Os autores (2025)}{mapa-anatomico}{}{}

% ----------------------------------------------------------
% Capítulo 7 - CONSIDERAÇÕES FINAIS
% ----------------------------------------------------------
\chapter{CONSIDERAÇÕES FINAIS} \label{cha:consideracoes}

O desenvolvimento do \textit{Appunture} representou um desafio multidisciplinar que envolveu conhecimentos de engenharia de software, design de interfaces e compreensão básica do domínio da acupuntura. O resultado é uma ferramenta educativa que busca facilitar o acesso a informações sobre os pontos de acupuntura, servindo como recurso complementar de apoio ao estudo.

Os principais objetivos propostos no início do projeto foram alcançados:

\begin{itemize}
    \item A implementação de um sistema multiplataforma (Android, iOS e Web) permite o acesso à ferramenta em diferentes dispositivos;
    \item A arquitetura \textit{offline-first} permite o uso do aplicativo mesmo em ambientes sem conectividade;
    \item O atlas anatômico digital com 15 visualizações vetoriais (SVG) auxilia na visualização dos pontos de acupuntura;
    \item O assistente baseado em IA auxilia na busca e correlação entre sintomas e pontos;
    \item O painel administrativo permite a atualização dos conteúdos do sistema;
    \item A ferramenta de mapeamento auxiliou no posicionamento das coordenadas dos pontos sobre as imagens.
\end{itemize}

A adoção da metodologia Scrum permitiu um desenvolvimento iterativo e incremental, com entregas frequentes e ajustes baseados em \textit{feedback} contínuo. A escolha das tecnologias --- React Native com Expo para o mobile, Java com Spring Boot para o \textit{backend}, e React para o painel administrativo --- mostrou-se adequada para os requisitos do projeto, oferecendo robustez, performance e facilidade de manutenção.

O \textit{Appunture} surge em um contexto de crescimento das práticas integrativas e complementares no Brasil. A ferramenta busca contribuir como um recurso de apoio ao estudo, facilitando o acesso a informações organizadas sobre acupuntura. É importante ressaltar que o aplicativo não substitui a formação profissional adequada, mas pode servir como material complementar de consulta.

Como trabalhos futuros, sugerem-se:

\begin{itemize}
    \item Implementação de recursos de realidade aumentada para visualização dos pontos em modelos 3D;
    \item Integração com sistemas de prontuário eletrônico;
    \item Desenvolvimento de módulos de avaliação e certificação;
    \item Expansão da base de conhecimento para incluir outras técnicas da MTC;
    \item Implementação de recursos de acessibilidade avançados;
    \item Tradução do conteúdo para outros idiomas;
    \item Aprimoramento contínuo do assistente IA com novos modelos e técnicas de RAG;
    \item Integração com \textit{wearables} para monitoramento de tratamentos.
\end{itemize}

Em síntese, o \textit{Appunture} representa uma contribuição para o ecossistema de ferramentas educativas voltadas à acupuntura. A plataforma busca facilitar o acesso a informações organizadas e servir como recurso complementar de apoio ao estudo. Espera-se que a ferramenta possa evoluir com base no feedback dos usuários, mantendo seu foco na acessibilidade e na qualidade da informação disponibilizada.


% ----------------------------------------------------------
% ELEMENTOS PÓS-TEXTUAIS
% ----------------------------------------------------------
\postextual

\addtocontents{toc}{\vspace{-6pt}}

% Referências bibliográficas
\begingroup
\printbibliography[heading=bay,notkeyword={consulta}, notkeyword={npub-informal}]
\endgroup

\addtocontents{toc}{\vspace{4pt}}

% Apêndices
\ifthenelse{\equal{\terApendice}{Sim}}
{\begin{apendicesenv}
        \renewcommand{\thechapter}{\Alph{chapter}}
        % Apendices do TCC Appunture

\begin{apendicesenv}

% ==============================================================================
% APENDICE A - DIAGRAMA DE CASOS DE USO
% ==============================================================================
\chapter{DIAGRAMA DE CASOS DE USO}
\label{apendice:casos-uso}

O diagrama de casos de uso apresenta as principais funcionalidades do sistema Appunture e suas interacoes com os atores do sistema, organizados por ordem de acao do usuario.

\figura{DIAGRAMA DE CASOS DE USO DO SISTEMA APPUNTURE}{0.95}{fig/casos-de-uso.png}{Os autores (2025)}{casos-uso}{}{}

% ==============================================================================
% APENDICE B - DIAGRAMA DE CLASSES
% ==============================================================================
\chapter{DIAGRAMA DE CLASSES}
\label{apendice:classes}

O diagrama de classes representa a estrutura estatica do sistema, mostrando as principais entidades e seus relacionamentos.

\figura{DIAGRAMA DE CLASSES DO SISTEMA APPUNTURE}{0.75}{fig/classes.png}{Os autores (2025)}{classes}{}{}

% ==============================================================================
% APENDICE C - MODELO LOGICO DO BANCO DE DADOS
% ==============================================================================
\chapter{MODELO LOGICO DO BANCO DE DADOS}
\label{apendice:modelo-logico}

O sistema Appunture utiliza uma arquitetura de dados hibrida, com dois bancos de dados distintos que trabalham em conjunto para garantir o funcionamento \textit{offline-first} do aplicativo.

\section{Banco de Dados Local (SQLite)}

O banco de dados SQLite e utilizado no aplicativo movel para armazenamento local dos dados, permitindo o funcionamento completo mesmo sem conexao com a internet. A estrutura foi projetada para sincronizacao bidirecional com o Firestore.

\figura{MODELO LOGICO - SQLITE (MOBILE)}{0.75}{fig/modelo-logico-sqlite.png}{Os autores (2025)}{modelo-sqlite}{}{}

\section{Banco de Dados em Nuvem (Firestore)}

O Firestore e um banco de dados NoSQL orientado a documentos, utilizado no \textit{backend} para armazenamento centralizado e sincronizacao entre dispositivos. Por ser NoSQL, os relacionamentos sao implementados atraves de referencias por ID entre documentos.

\figura{MODELO DE DADOS - FIRESTORE (BACKEND)}{0.75}{fig/modelo-logico-firestore.png}{Os autores (2025)}{modelo-firestore}{}{}

% ==============================================================================
% APENDICE D - DIAGRAMAS DE SEQUENCIA
% ==============================================================================
\chapter{DIAGRAMAS DE SEQUENCIA}
\label{apendice:sequencia}

Os diagramas de sequencia apresentam o fluxo de interacao entre os componentes do sistema para cada historia de usuario implementada.

% ------------------------------------------------------------------------------
\section{HU-01: Busca de Pontos de Acupuntura}
\label{apendice:seq-busca}

Este diagrama representa o fluxo de busca de pontos de acupuntura no sistema.

\figura{DIAGRAMA DE SEQUENCIA - BUSCA DE PONTOS}{0.70}{fig/seq-busca.png}{Os autores (2025)}{seq-busca}{}{}

% ------------------------------------------------------------------------------
\section{HU-02: Detalhes do Ponto de Acupuntura}
\label{apendice:seq-detalhes}

Este diagrama representa o fluxo de visualizacao dos detalhes de um ponto de acupuntura.

\figura{DIAGRAMA DE SEQUENCIA - DETALHES DO PONTO}{0.70}{fig/seq-detalhes.png}{Os autores (2025)}{seq-detalhes}{}{}

% ------------------------------------------------------------------------------
\section{HU-03: Atlas Visual Interativo}
\label{apendice:seq-atlas}

Este diagrama representa o fluxo de interacao com o atlas visual do corpo humano.

\figura{DIAGRAMA DE SEQUENCIA - ATLAS VISUAL}{0.70}{fig/seq-atlas.png}{Os autores (2025)}{seq-atlas}{}{}

% ------------------------------------------------------------------------------
\section{HU-04: Gerenciamento de Favoritos}
\label{apendice:seq-favoritos}

Este diagrama representa o fluxo de adicao e remocao de pontos favoritos.

\figura{DIAGRAMA DE SEQUENCIA - FAVORITOS}{0.70}{fig/seq-favoritos.png}{Os autores (2025)}{seq-favoritos}{}{}

% ------------------------------------------------------------------------------
\section{HU-05: Sincronizacao de Dados}
\label{apendice:seq-sync}

Este diagrama representa o fluxo de sincronizacao de dados entre dispositivo e nuvem.

\figura{DIAGRAMA DE SEQUENCIA - SINCRONIZACAO}{0.70}{fig/seq-sync.png}{Os autores (2025)}{seq-sync}{}{}

% ------------------------------------------------------------------------------
\section{HU-06: Assistente de Inteligencia Artificial}
\label{apendice:seq-ia}

Este diagrama representa o fluxo de interacao com o assistente de IA do sistema.

\figura{DIAGRAMA DE SEQUENCIA - ASSISTENTE IA}{0.70}{fig/seq-ia.png}{Os autores (2025)}{seq-ia}{}{}

% ------------------------------------------------------------------------------
\section{HU-07: Mapeamento de Sintomas}
\label{apendice:seq-mapper}

Este diagrama representa o fluxo de mapeamento de sintomas para pontos de acupuntura.

\figura{DIAGRAMA DE SEQUENCIA - MAPEAMENTO DE SINTOMAS}{0.70}{fig/seq-mapper.png}{Os autores (2025)}{seq-mapper}{}{}

% ------------------------------------------------------------------------------
\section{HU-08: Autenticacao de Usuario}
\label{apendice:seq-auth}

Os diagramas a seguir representam os fluxos de autenticacao do sistema, incluindo login com email/senha, cadastro de novo usuario e login social via Google.

\figura{DIAGRAMA DE SEQUENCIA - LOGIN COM EMAIL/SENHA}{0.65}{fig/seq-loginemailsenha.png}{Os autores (2025)}{seq-login}{}{}

\figura{DIAGRAMA DE SEQUENCIA - CADASTRO DE USUARIO}{0.65}{fig/seq-cadastro.png}{Os autores (2025)}{seq-cadastro}{}{}

\figura{DIAGRAMA DE SEQUENCIA - LOGIN SOCIAL (GOOGLE)}{0.65}{fig/seq-loginsocial.png}{Os autores (2025)}{seq-google}{}{}

% ------------------------------------------------------------------------------
\section{HU-09: Navegacao por Meridianos}
\label{apendice:seq-meridians}

Os diagramas a seguir representam os fluxos de navegacao pelos meridianos do sistema.

\figura{DIAGRAMA DE SEQUENCIA - LISTAR MERIDIANOS}{0.65}{fig/seq-listarmeridianos.png}{Os autores (2025)}{seq-listarmeridianos}{}{}

\figura{DIAGRAMA DE SEQUENCIA - ACESSAR MERIDIANO}{0.65}{fig/seq-acessarmeridiano.png}{Os autores (2025)}{seq-acessarmeridiano}{}{}

\figura{DIAGRAMA DE SEQUENCIA - INFORMACOES DO MERIDIANO}{0.60}{fig/seq-infosmeridiano.png}{Os autores (2025)}{seq-infosmeridiano}{}{}

% ------------------------------------------------------------------------------
\section{HU-10: Configuracoes e Personalizacao}
\label{apendice:seq-config}

Os diagramas a seguir representam os fluxos de configuracao e personalizacao do aplicativo.

\figura{DIAGRAMA DE SEQUENCIA - ALTERNAR TEMA}{0.65}{fig/seq-apptheme.png}{Os autores (2025)}{seq-theme}{}{}

\figura{DIAGRAMA DE SEQUENCIA - NOTIFICACOES}{0.65}{fig/seq-onoffnotifs.png}{Os autores (2025)}{seq-notifs}{}{}

\figura{DIAGRAMA DE SEQUENCIA - MODO OFFLINE}{0.60}{fig/seq-offline.png}{Os autores (2025)}{seq-offline}{}{}

\figura{DIAGRAMA DE SEQUENCIA - LOGOUT}{0.65}{fig/seq-logout.png}{Os autores (2025)}{seq-logout}{}{}

\figura{DIAGRAMA DE SEQUENCIA - STATUS DE SINCRONIZACAO}{0.60}{fig/seq-statussync.png}{Os autores (2025)}{seq-statussync}{}{}

% ==============================================================================
% APENDICE E - HISTORIAS DE USUARIO
% ==============================================================================
\chapter{HISTORIAS DE USUARIO}
\label{apendice:historias}

Este apendice apresenta as historias de usuario detalhadas do sistema Appunture, incluindo criterios de aceitacao e prototipos de tela.

% ------------------------------------------------------------------------------
\section{HU-01: Busca de Pontos de Acupuntura}
\label{hu:busca}

\textbf{Como} estudante ou profissional de acupuntura, \\
\textbf{Quero} buscar pontos de acupuntura por nome, codigo ou funcao, \\
\textbf{Para que} eu possa encontrar rapidamente informacoes sobre pontos especificos.

\subsection*{Criterios de Aceitacao}

\begin{enumerate}
    \item O sistema deve permitir busca por nome do ponto em portugues
    \item O sistema deve permitir busca por codigo alfanumerico (ex: VG20, E36, IG4)
    \item O sistema deve exibir resultados em tempo real com \textit{debounce} de 300ms
    \item O sistema deve funcionar offline com dados em cache local (SQLite)
    \item O sistema deve permitir alternar favoritos diretamente nos resultados
    \item O sistema deve exibir mensagem quando nenhum resultado for encontrado
\end{enumerate}

\figura{PROTOTIPO - BUSCA DE PONTOS}{0.28}{fig/proto-busca.png}{Os autores (2025)}{hu01-busca}{}{}

% ------------------------------------------------------------------------------
\section{HU-02: Detalhes do Ponto de Acupuntura}
\label{hu:detalhes}

\textbf{Como} estudante ou profissional de acupuntura, \\
\textbf{Quero} visualizar informacoes detalhadas de um ponto de acupuntura, \\
\textbf{Para que} eu possa estudar suas caracteristicas e aplicacoes clinicas.

\subsection*{Criterios de Aceitacao}

\begin{enumerate}
    \item O sistema deve exibir nome em portugues, pinyin e caracteres chineses
    \item O sistema deve mostrar a localizacao anatomica precisa
    \item O sistema deve apresentar as funcoes terapeuticas do ponto
    \item O sistema deve listar as indicacoes clinicas baseadas na MTC
    \item O sistema deve exibir imagem ilustrativa da localizacao
    \item O sistema deve mostrar tecnicas de insercao recomendadas
    \item O sistema deve indicar precaucoes e contraindicacoes
\end{enumerate}

\figura{PROTOTIPO - DETALHES DO PONTO}{0.28}{fig/proto-detalhes.png}{Os autores (2025)}{hu02-detalhes}{}{}

% ------------------------------------------------------------------------------
\section{HU-03: Atlas Visual Interativo}
\label{hu:atlas}

\textbf{Como} estudante ou profissional de acupuntura, \\
\textbf{Quero} navegar por um atlas visual do corpo humano, \\
\textbf{Para que} eu possa localizar visualmente os pontos de acupuntura.

\subsection*{Criterios de Aceitacao}

\begin{enumerate}
    \item O sistema deve exibir modelo anatomico SVG interativo do corpo humano
    \item O sistema deve permitir navegacao entre multiplas camadas anatomicas
    \item O sistema deve permitir alternar entre visualizacao frontal e dorsal
    \item O sistema deve exibir marcadores nos pontos com coordenadas cadastradas
    \item O sistema deve permitir toque em marcador para navegar aos detalhes do ponto
    \item O sistema deve exibir nome da camada atual e total de camadas disponiveis
\end{enumerate}

\figura{PROTOTIPO - ATLAS VISUAL}{0.28}{fig/proto-mapa.png}{Os autores (2025)}{hu03-mapa}{}{}

% ------------------------------------------------------------------------------
\section{HU-04: Gerenciamento de Favoritos}
\label{hu:favoritos}

\textbf{Como} estudante ou profissional de acupuntura, \\
\textbf{Quero} salvar pontos de acupuntura como favoritos, \\
\textbf{Para que} eu possa acessa-los rapidamente no futuro.

\subsection*{Criterios de Aceitacao}

\begin{enumerate}
    \item O sistema deve exigir autenticacao para usar a funcionalidade de favoritos
    \item O sistema deve permitir adicionar pontos a lista de favoritos
    \item O sistema deve permitir remover pontos da lista de favoritos
    \item O sistema deve exibir lista de favoritos com contagem total
    \item O sistema deve sincronizar favoritos com a conta do usuario (\textit{offline-first})
    \item O sistema deve usar \textit{optimistic update} para feedback imediato ao usuario
\end{enumerate}

\figura{PROTOTIPO - FAVORITOS}{0.28}{fig/proto-favoritos.png}{Os autores (2025)}{hu04-favoritos}{}{}

% ------------------------------------------------------------------------------
\section{HU-05: Sincronizacao de Dados}
\label{hu:sync}

\textbf{Como} usuario do aplicativo, \\
\textbf{Quero} sincronizar meus dados entre dispositivos, \\
\textbf{Para que} eu possa acessar minhas informacoes em qualquer lugar.

\subsection*{Criterios de Aceitacao}

\begin{enumerate}
    \item O sistema deve sincronizar automaticamente quando conectado a internet
    \item O sistema deve funcionar offline com dados em cache
    \item O sistema deve resolver conflitos de sincronizacao automaticamente
    \item O sistema deve notificar o usuario sobre status da sincronizacao
    \item O sistema deve permitir sincronizacao manual sob demanda
    \item O sistema deve manter historico de sincronizacoes
\end{enumerate}

% ------------------------------------------------------------------------------
\section{HU-06: Assistente de Inteligencia Artificial}
\label{hu:ia}

\textbf{Como} estudante ou profissional de acupuntura, \\
\textbf{Quero} consultar um assistente de IA sobre acupuntura, \\
\textbf{Para que} eu possa obter respostas contextualizadas as minhas duvidas.

\subsection*{Criterios de Aceitacao}

\begin{enumerate}
    \item O sistema deve permitir fazer perguntas em linguagem natural
    \item O sistema deve fornecer respostas baseadas em literatura de acupuntura
    \item O sistema deve citar fontes quando apropriado
    \item O sistema deve manter contexto da conversa
    \item O sistema deve sugerir pontos relacionados as perguntas
    \item O sistema deve indicar limitacoes e recomendar consulta profissional
\end{enumerate}

\figura{PROTOTIPO - ASSISTENTE IA}{0.28}{fig/proto-chat.png}{Os autores (2025)}{hu06-chat}{}{}

% ------------------------------------------------------------------------------
\section{HU-07: Mapeamento de Sintomas}
\label{hu:mapper}

\textbf{Como} profissional de acupuntura, \\
\textbf{Quero} mapear sintomas para pontos de acupuntura recomendados, \\
\textbf{Para que} eu possa auxiliar no planejamento de tratamentos.

\subsection*{Criterios de Aceitacao}

\begin{enumerate}
    \item O sistema deve listar sintomas cadastrados no banco de dados
    \item O sistema deve permitir filtrar sintomas por categoria
    \item O sistema deve permitir buscar sintomas por nome
    \item O sistema deve exibir indicador de severidade quando disponivel
    \item O sistema deve navegar para detalhes mostrando pontos relacionados
    \item O sistema deve incluir aviso de uso educacional na tela de detalhes
\end{enumerate}

% ------------------------------------------------------------------------------
\section{HU-08: Autenticacao de Usuario}
\label{hu:autenticacao}

\textbf{Como} usuario do aplicativo, \\
\textbf{Quero} criar uma conta e realizar login de forma segura, \\
\textbf{Para que} eu possa acessar recursos personalizados e sincronizar meus dados.

\subsection*{Criterios de Aceitacao}

\begin{enumerate}
    \item O sistema deve permitir cadastro com nome, email e senha
    \item O sistema deve validar forca minima da senha (6 caracteres)
    \item O sistema deve exigir confirmacao de senha no cadastro
    \item O sistema deve permitir login com email e senha
    \item O sistema deve oferecer login social via Google
    \item O sistema deve permitir recuperacao de senha por email
    \item O sistema deve manter sessao ativa entre execucoes do app
    \item O sistema deve permitir acesso como visitante com funcionalidades limitadas
\end{enumerate}

\figura{PROTOTIPO - LOGIN}{0.28}{fig/proto-login.png}{Os autores (2025)}{hu08-login}{}{}

\figura{PROTOTIPO - CADASTRO}{0.28}{fig/proto-cadastro.png}{Os autores (2025)}{hu08-cadastro}{}{}

% ------------------------------------------------------------------------------
\section{HU-09: Navegacao por Meridianos}
\label{hu:meridianos}

\textbf{Como} estudante ou profissional de acupuntura, \\
\textbf{Quero} navegar pelos meridianos da medicina tradicional chinesa, \\
\textbf{Para que} eu possa estudar os pontos organizados por canal energetico.

\subsection*{Criterios de Aceitacao}

\begin{enumerate}
    \item O sistema deve listar os 12 meridianos principais e 2 vasos extraordinarios
    \item O sistema deve exibir nome em portugues, pinyin e caracteres chineses
    \item O sistema deve mostrar o elemento Wu Xing associado a cada meridiano
    \item O sistema deve indicar o horario de maior atividade energetica
    \item O sistema deve exibir o orgao relacionado ao meridiano
    \item O sistema deve mostrar a quantidade de pontos em cada meridiano
    \item O sistema deve permitir visualizar todos os pontos de um meridiano
    \item O sistema deve usar cores caracteristicas para cada elemento
\end{enumerate}

\figura{PROTOTIPO - NAVEGACAO POR MERIDIANOS}{0.28}{fig/proto-meridianos.png}{Os autores (2025)}{hu09-meridianos}{}{}

% ------------------------------------------------------------------------------
\section{HU-10: Configuracoes e Personalizacao}
\label{hu:configuracoes}

\textbf{Como} usuario do aplicativo, \\
\textbf{Quero} personalizar as configuracoes do aplicativo, \\
\textbf{Para que} eu possa adaptar a experiencia as minhas preferencias.

\subsection*{Criterios de Aceitacao}

\begin{enumerate}
    \item O sistema deve permitir alternar entre tema claro, escuro e automatico
    \item O sistema deve salvar a preferencia de tema do usuario
    \item O sistema deve permitir ativar ou desativar notificacoes push
    \item O sistema deve exibir informacoes da conta do usuario logado
    \item O sistema deve mostrar status de sincronizacao e ultima atualizacao
    \item O sistema deve permitir ativar modo offline para economia de dados
    \item O sistema deve permitir logout com confirmacao
    \item O sistema deve exibir versao do aplicativo e informacoes de suporte
\end{enumerate}

\figura{PROTOTIPO - TELA INICIAL}{0.28}{fig/proto-home.png}{Os autores (2025)}{hu10-home}{}{}

\end{apendicesenv}

\end{apendicesenv}
}{}

% Anexos
\ifthenelse{\equal{\terAnexo}{Sim}}{
\begin{anexosenv}
        \renewcommand{\thechapter}{\Alph{chapter}}
        % ----------------------------------------------------------
% ANEXOS
% ----------------------------------------------------------

\chapter{DOCUMENTAÇÃO COMPLEMENTAR} \label{ax:doc}

Este anexo contém documentação complementar relevante para o projeto \textit{Appunture}.

\section{DIRETRIZES DA OMS PARA ACUPUNTURA}

A Organização Mundial da Saúde (OMS) publicou em 1999 as ``Guidelines on basic training and safety in acupuncture'', que estabelecem os requisitos mínimos para formação e prática segura da acupuntura. Os principais pontos abordados incluem:

\begin{itemize}
    \item Requisitos de formação para diferentes níveis de praticantes;
    \item Protocolos de biossegurança;
    \item Indicações e contraindicações gerais;
    \item Padronização de nomenclatura dos pontos;
    \item Técnicas de inserção de agulhas.
\end{itemize}

\section{NORMAS ISO APLICÁVEIS}

\subsection{ISO/IEC 25010:2011}

Esta norma define o modelo de qualidade para sistemas e produtos de software, estabelecendo oito características de qualidade:

\begin{enumerate}
    \item Adequação funcional
    \item Eficiência de desempenho
    \item Compatibilidade
    \item Usabilidade
    \item Confiabilidade
    \item Segurança
    \item Manutenibilidade
    \item Portabilidade
\end{enumerate}

\subsection{ISO 9241-210:2019}

Esta norma estabelece os princípios do design centrado no ser humano para sistemas interativos, incluindo:

\begin{itemize}
    \item Compreensão dos usuários, tarefas e ambientes;
    \item Envolvimento ativo dos usuários durante o design e desenvolvimento;
    \item Refinamento do design através de avaliação centrada no usuário;
    \item Processo iterativo;
    \item Consideração da experiência do usuário como um todo;
    \item Equipe multidisciplinar.
\end{itemize}

\section{PROJETO DE LEI 5983/2019}

O Projeto de Lei 5983/2019, em tramitação no Senado Federal, propõe a regulamentação do exercício da acupuntura no Brasil. Os principais pontos do projeto incluem:

\begin{itemize}
    \item Definição de acupuntura como prática de saúde;
    \item Requisitos de formação para exercício da profissão;
    \item Criação de registro profissional específico;
    \item Estabelecimento de código de ética;
    \item Fiscalização do exercício profissional.
\end{itemize}

\end{anexosenv}
}{}

% Índice Remissivo
\ifthenelse{\equal{\terIndiceR}{Sim}}{
\phantompart
\printindex
}{}

\end{document}
