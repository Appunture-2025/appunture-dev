% Apendices do TCC Appunture

\begin{apendicesenv}

% ==============================================================================
% APENDICE A - DIAGRAMA DE CASOS DE USO
% ==============================================================================
\chapter{DIAGRAMA DE CASOS DE USO}
\label{apendice:casos-uso}

O diagrama de casos de uso apresenta as principais funcionalidades do sistema Appunture e suas interacoes com os atores do sistema, organizados por ordem de acao do usuario.

\figura{DIAGRAMA DE CASOS DE USO DO SISTEMA APPUNTURE}{0.95}{fig/casos-de-uso.png}{Os autores (2025)}{casos-uso}{}{}

% ==============================================================================
% APENDICE B - DIAGRAMA DE CLASSES
% ==============================================================================
\chapter{DIAGRAMA DE CLASSES}
\label{apendice:classes}

O diagrama de classes representa a estrutura estatica do sistema, mostrando as principais entidades e seus relacionamentos.

\figura{DIAGRAMA DE CLASSES DO SISTEMA APPUNTURE}{0.75}{fig/classes.png}{Os autores (2025)}{classes}{}{}

% ==============================================================================
% APENDICE C - MODELO LOGICO DO BANCO DE DADOS
% ==============================================================================
\chapter{MODELO LOGICO DO BANCO DE DADOS}
\label{apendice:modelo-logico}

O sistema Appunture utiliza uma arquitetura de dados hibrida, com dois bancos de dados distintos que trabalham em conjunto para garantir o funcionamento \textit{offline-first} do aplicativo.

\section{Banco de Dados Local (SQLite)}

O banco de dados SQLite e utilizado no aplicativo movel para armazenamento local dos dados, permitindo o funcionamento completo mesmo sem conexao com a internet. A estrutura foi projetada para sincronizacao bidirecional com o Firestore.

\figura{MODELO LOGICO - SQLITE (MOBILE)}{0.75}{fig/modelo-logico-sqlite.png}{Os autores (2025)}{modelo-sqlite}{}{}

\section{Banco de Dados em Nuvem (Firestore)}

O Firestore e um banco de dados NoSQL orientado a documentos, utilizado no \textit{backend} para armazenamento centralizado e sincronizacao entre dispositivos. Por ser NoSQL, os relacionamentos sao implementados atraves de referencias por ID entre documentos.

\figura{MODELO DE DADOS - FIRESTORE (BACKEND)}{0.75}{fig/modelo-logico-firestore.png}{Os autores (2025)}{modelo-firestore}{}{}

% ==============================================================================
% APENDICE D - DIAGRAMAS DE SEQUENCIA
% ==============================================================================
\chapter{DIAGRAMAS DE SEQUENCIA}
\label{apendice:sequencia}

Os diagramas de sequencia apresentam o fluxo de interacao entre os componentes do sistema para cada historia de usuario implementada.

% ------------------------------------------------------------------------------
\section{HU-01: Busca de Pontos de Acupuntura}
\label{apendice:seq-busca}

Este diagrama representa o fluxo de busca de pontos de acupuntura no sistema.

\figura{DIAGRAMA DE SEQUENCIA - BUSCA DE PONTOS}{0.70}{fig/seq-busca.png}{Os autores (2025)}{seq-busca}{}{}

% ------------------------------------------------------------------------------
\section{HU-02: Detalhes do Ponto de Acupuntura}
\label{apendice:seq-detalhes}

Este diagrama representa o fluxo de visualizacao dos detalhes de um ponto de acupuntura.

\figura{DIAGRAMA DE SEQUENCIA - DETALHES DO PONTO}{0.70}{fig/seq-detalhes.png}{Os autores (2025)}{seq-detalhes}{}{}

% ------------------------------------------------------------------------------
\section{HU-03: Atlas Visual Interativo}
\label{apendice:seq-atlas}

Este diagrama representa o fluxo de interacao com o atlas visual do corpo humano.

\figura{DIAGRAMA DE SEQUENCIA - ATLAS VISUAL}{0.70}{fig/seq-atlas.png}{Os autores (2025)}{seq-atlas}{}{}

% ------------------------------------------------------------------------------
\section{HU-04: Gerenciamento de Favoritos}
\label{apendice:seq-favoritos}

Este diagrama representa o fluxo de adicao e remocao de pontos favoritos.

\figura{DIAGRAMA DE SEQUENCIA - FAVORITOS}{0.70}{fig/seq-favoritos.png}{Os autores (2025)}{seq-favoritos}{}{}

% ------------------------------------------------------------------------------
\section{HU-05: Sincronizacao de Dados}
\label{apendice:seq-sync}

Este diagrama representa o fluxo de sincronizacao de dados entre dispositivo e nuvem.

\figura{DIAGRAMA DE SEQUENCIA - SINCRONIZACAO}{0.70}{fig/seq-sync.png}{Os autores (2025)}{seq-sync}{}{}

% ------------------------------------------------------------------------------
\section{HU-06: Assistente de Inteligencia Artificial}
\label{apendice:seq-ia}

Este diagrama representa o fluxo de interacao com o assistente de IA do sistema.

\figura{DIAGRAMA DE SEQUENCIA - ASSISTENTE IA}{0.70}{fig/seq-ia.png}{Os autores (2025)}{seq-ia}{}{}

% ------------------------------------------------------------------------------
\section{HU-07: Mapeamento de Sintomas}
\label{apendice:seq-mapper}

Este diagrama representa o fluxo de mapeamento de sintomas para pontos de acupuntura.

\figura{DIAGRAMA DE SEQUENCIA - MAPEAMENTO DE SINTOMAS}{0.70}{fig/seq-mapper.png}{Os autores (2025)}{seq-mapper}{}{}

% ------------------------------------------------------------------------------
\section{HU-08: Autenticacao de Usuario}
\label{apendice:seq-auth}

Os diagramas a seguir representam os fluxos de autenticacao do sistema, incluindo login com email/senha, cadastro de novo usuario e login social via Google.

\figura{DIAGRAMA DE SEQUENCIA - LOGIN COM EMAIL/SENHA}{0.65}{fig/seq-loginemailsenha.png}{Os autores (2025)}{seq-login}{}{}

\figura{DIAGRAMA DE SEQUENCIA - CADASTRO DE USUARIO}{0.65}{fig/seq-cadastro.png}{Os autores (2025)}{seq-cadastro}{}{}

\figura{DIAGRAMA DE SEQUENCIA - LOGIN SOCIAL (GOOGLE)}{0.65}{fig/seq-loginsocial.png}{Os autores (2025)}{seq-google}{}{}

% ------------------------------------------------------------------------------
\section{HU-09: Navegacao por Meridianos}
\label{apendice:seq-meridians}

Os diagramas a seguir representam os fluxos de navegacao pelos meridianos do sistema.

\figura{DIAGRAMA DE SEQUENCIA - LISTAR MERIDIANOS}{0.65}{fig/seq-listarmeridianos.png}{Os autores (2025)}{seq-listarmeridianos}{}{}

\figura{DIAGRAMA DE SEQUENCIA - ACESSAR MERIDIANO}{0.65}{fig/seq-acessarmeridiano.png}{Os autores (2025)}{seq-acessarmeridiano}{}{}

\figura{DIAGRAMA DE SEQUENCIA - INFORMACOES DO MERIDIANO}{0.60}{fig/seq-infosmeridiano.png}{Os autores (2025)}{seq-infosmeridiano}{}{}

% ------------------------------------------------------------------------------
\section{HU-10: Configuracoes e Personalizacao}
\label{apendice:seq-config}

Os diagramas a seguir representam os fluxos de configuracao e personalizacao do aplicativo.

\figura{DIAGRAMA DE SEQUENCIA - ALTERNAR TEMA}{0.65}{fig/seq-apptheme.png}{Os autores (2025)}{seq-theme}{}{}

\figura{DIAGRAMA DE SEQUENCIA - NOTIFICACOES}{0.65}{fig/seq-onoffnotifs.png}{Os autores (2025)}{seq-notifs}{}{}

\figura{DIAGRAMA DE SEQUENCIA - MODO OFFLINE}{0.60}{fig/seq-offline.png}{Os autores (2025)}{seq-offline}{}{}

\figura{DIAGRAMA DE SEQUENCIA - LOGOUT}{0.65}{fig/seq-logout.png}{Os autores (2025)}{seq-logout}{}{}

\figura{DIAGRAMA DE SEQUENCIA - STATUS DE SINCRONIZACAO}{0.60}{fig/seq-statussync.png}{Os autores (2025)}{seq-statussync}{}{}

% ==============================================================================
% APENDICE E - HISTORIAS DE USUARIO
% ==============================================================================
\chapter{HISTORIAS DE USUARIO}
\label{apendice:historias}

Este apendice apresenta as historias de usuario detalhadas do sistema Appunture, incluindo criterios de aceitacao e prototipos de tela.

% ------------------------------------------------------------------------------
\section{HU-01: Busca de Pontos de Acupuntura}
\label{hu:busca}

\textbf{Como} estudante ou profissional de acupuntura, \\
\textbf{Quero} buscar pontos de acupuntura por nome, codigo ou funcao, \\
\textbf{Para que} eu possa encontrar rapidamente informacoes sobre pontos especificos.

\subsection*{Criterios de Aceitacao}

\begin{enumerate}
    \item O sistema deve permitir busca por nome do ponto em portugues
    \item O sistema deve permitir busca por codigo alfanumerico (ex: VG20, E36, IG4)
    \item O sistema deve exibir resultados em tempo real com \textit{debounce} de 300ms
    \item O sistema deve funcionar offline com dados em cache local (SQLite)
    \item O sistema deve permitir alternar favoritos diretamente nos resultados
    \item O sistema deve exibir mensagem quando nenhum resultado for encontrado
\end{enumerate}

\figura{PROTOTIPO - BUSCA DE PONTOS}{0.28}{fig/proto-busca.png}{Os autores (2025)}{hu01-busca}{}{}

% ------------------------------------------------------------------------------
\section{HU-02: Detalhes do Ponto de Acupuntura}
\label{hu:detalhes}

\textbf{Como} estudante ou profissional de acupuntura, \\
\textbf{Quero} visualizar informacoes detalhadas de um ponto de acupuntura, \\
\textbf{Para que} eu possa estudar suas caracteristicas e aplicacoes clinicas.

\subsection*{Criterios de Aceitacao}

\begin{enumerate}
    \item O sistema deve exibir nome em portugues, pinyin e caracteres chineses
    \item O sistema deve mostrar a localizacao anatomica precisa
    \item O sistema deve apresentar as funcoes terapeuticas do ponto
    \item O sistema deve listar as indicacoes clinicas baseadas na MTC
    \item O sistema deve exibir imagem ilustrativa da localizacao
    \item O sistema deve mostrar tecnicas de insercao recomendadas
    \item O sistema deve indicar precaucoes e contraindicacoes
\end{enumerate}

\figura{PROTOTIPO - DETALHES DO PONTO}{0.28}{fig/proto-detalhes.png}{Os autores (2025)}{hu02-detalhes}{}{}

% ------------------------------------------------------------------------------
\section{HU-03: Atlas Visual Interativo}
\label{hu:atlas}

\textbf{Como} estudante ou profissional de acupuntura, \\
\textbf{Quero} navegar por um atlas visual do corpo humano, \\
\textbf{Para que} eu possa localizar visualmente os pontos de acupuntura.

\subsection*{Criterios de Aceitacao}

\begin{enumerate}
    \item O sistema deve exibir modelo anatomico SVG interativo do corpo humano
    \item O sistema deve permitir navegacao entre multiplas camadas anatomicas
    \item O sistema deve permitir alternar entre visualizacao frontal e dorsal
    \item O sistema deve exibir marcadores nos pontos com coordenadas cadastradas
    \item O sistema deve permitir toque em marcador para navegar aos detalhes do ponto
    \item O sistema deve exibir nome da camada atual e total de camadas disponiveis
\end{enumerate}

\figura{PROTOTIPO - ATLAS VISUAL}{0.28}{fig/proto-mapa.png}{Os autores (2025)}{hu03-mapa}{}{}

% ------------------------------------------------------------------------------
\section{HU-04: Gerenciamento de Favoritos}
\label{hu:favoritos}

\textbf{Como} estudante ou profissional de acupuntura, \\
\textbf{Quero} salvar pontos de acupuntura como favoritos, \\
\textbf{Para que} eu possa acessa-los rapidamente no futuro.

\subsection*{Criterios de Aceitacao}

\begin{enumerate}
    \item O sistema deve exigir autenticacao para usar a funcionalidade de favoritos
    \item O sistema deve permitir adicionar pontos a lista de favoritos
    \item O sistema deve permitir remover pontos da lista de favoritos
    \item O sistema deve exibir lista de favoritos com contagem total
    \item O sistema deve sincronizar favoritos com a conta do usuario (\textit{offline-first})
    \item O sistema deve usar \textit{optimistic update} para feedback imediato ao usuario
\end{enumerate}

\figura{PROTOTIPO - FAVORITOS}{0.28}{fig/proto-favoritos.png}{Os autores (2025)}{hu04-favoritos}{}{}

% ------------------------------------------------------------------------------
\section{HU-05: Sincronizacao de Dados}
\label{hu:sync}

\textbf{Como} usuario do aplicativo, \\
\textbf{Quero} sincronizar meus dados entre dispositivos, \\
\textbf{Para que} eu possa acessar minhas informacoes em qualquer lugar.

\subsection*{Criterios de Aceitacao}

\begin{enumerate}
    \item O sistema deve sincronizar automaticamente quando conectado a internet
    \item O sistema deve funcionar offline com dados em cache
    \item O sistema deve resolver conflitos de sincronizacao automaticamente
    \item O sistema deve notificar o usuario sobre status da sincronizacao
    \item O sistema deve permitir sincronizacao manual sob demanda
    \item O sistema deve manter historico de sincronizacoes
\end{enumerate}

% ------------------------------------------------------------------------------
\section{HU-06: Assistente de Inteligencia Artificial}
\label{hu:ia}

\textbf{Como} estudante ou profissional de acupuntura, \\
\textbf{Quero} consultar um assistente de IA sobre acupuntura, \\
\textbf{Para que} eu possa obter respostas contextualizadas as minhas duvidas.

\subsection*{Criterios de Aceitacao}

\begin{enumerate}
    \item O sistema deve permitir fazer perguntas em linguagem natural
    \item O sistema deve fornecer respostas baseadas em literatura de acupuntura
    \item O sistema deve citar fontes quando apropriado
    \item O sistema deve manter contexto da conversa
    \item O sistema deve sugerir pontos relacionados as perguntas
    \item O sistema deve indicar limitacoes e recomendar consulta profissional
\end{enumerate}

\figura{PROTOTIPO - ASSISTENTE IA}{0.28}{fig/proto-chat.png}{Os autores (2025)}{hu06-chat}{}{}

% ------------------------------------------------------------------------------
\section{HU-07: Mapeamento de Sintomas}
\label{hu:mapper}

\textbf{Como} profissional de acupuntura, \\
\textbf{Quero} mapear sintomas para pontos de acupuntura recomendados, \\
\textbf{Para que} eu possa auxiliar no planejamento de tratamentos.

\subsection*{Criterios de Aceitacao}

\begin{enumerate}
    \item O sistema deve listar sintomas cadastrados no banco de dados
    \item O sistema deve permitir filtrar sintomas por categoria
    \item O sistema deve permitir buscar sintomas por nome
    \item O sistema deve exibir indicador de severidade quando disponivel
    \item O sistema deve navegar para detalhes mostrando pontos relacionados
    \item O sistema deve incluir aviso de uso educacional na tela de detalhes
\end{enumerate}

% ------------------------------------------------------------------------------
\section{HU-08: Autenticacao de Usuario}
\label{hu:autenticacao}

\textbf{Como} usuario do aplicativo, \\
\textbf{Quero} criar uma conta e realizar login de forma segura, \\
\textbf{Para que} eu possa acessar recursos personalizados e sincronizar meus dados.

\subsection*{Criterios de Aceitacao}

\begin{enumerate}
    \item O sistema deve permitir cadastro com nome, email e senha
    \item O sistema deve validar forca minima da senha (6 caracteres)
    \item O sistema deve exigir confirmacao de senha no cadastro
    \item O sistema deve permitir login com email e senha
    \item O sistema deve oferecer login social via Google
    \item O sistema deve permitir recuperacao de senha por email
    \item O sistema deve manter sessao ativa entre execucoes do app
    \item O sistema deve permitir acesso como visitante com funcionalidades limitadas
\end{enumerate}

\figura{PROTOTIPO - LOGIN}{0.28}{fig/proto-login.png}{Os autores (2025)}{hu08-login}{}{}

\figura{PROTOTIPO - CADASTRO}{0.28}{fig/proto-cadastro.png}{Os autores (2025)}{hu08-cadastro}{}{}

% ------------------------------------------------------------------------------
\section{HU-09: Navegacao por Meridianos}
\label{hu:meridianos}

\textbf{Como} estudante ou profissional de acupuntura, \\
\textbf{Quero} navegar pelos meridianos da medicina tradicional chinesa, \\
\textbf{Para que} eu possa estudar os pontos organizados por canal energetico.

\subsection*{Criterios de Aceitacao}

\begin{enumerate}
    \item O sistema deve listar os 12 meridianos principais e 2 vasos extraordinarios
    \item O sistema deve exibir nome em portugues, pinyin e caracteres chineses
    \item O sistema deve mostrar o elemento Wu Xing associado a cada meridiano
    \item O sistema deve indicar o horario de maior atividade energetica
    \item O sistema deve exibir o orgao relacionado ao meridiano
    \item O sistema deve mostrar a quantidade de pontos em cada meridiano
    \item O sistema deve permitir visualizar todos os pontos de um meridiano
    \item O sistema deve usar cores caracteristicas para cada elemento
\end{enumerate}

\figura{PROTOTIPO - NAVEGACAO POR MERIDIANOS}{0.28}{fig/proto-meridianos.png}{Os autores (2025)}{hu09-meridianos}{}{}

% ------------------------------------------------------------------------------
\section{HU-10: Configuracoes e Personalizacao}
\label{hu:configuracoes}

\textbf{Como} usuario do aplicativo, \\
\textbf{Quero} personalizar as configuracoes do aplicativo, \\
\textbf{Para que} eu possa adaptar a experiencia as minhas preferencias.

\subsection*{Criterios de Aceitacao}

\begin{enumerate}
    \item O sistema deve permitir alternar entre tema claro, escuro e automatico
    \item O sistema deve salvar a preferencia de tema do usuario
    \item O sistema deve permitir ativar ou desativar notificacoes push
    \item O sistema deve exibir informacoes da conta do usuario logado
    \item O sistema deve mostrar status de sincronizacao e ultima atualizacao
    \item O sistema deve permitir ativar modo offline para economia de dados
    \item O sistema deve permitir logout com confirmacao
    \item O sistema deve exibir versao do aplicativo e informacoes de suporte
\end{enumerate}

\figura{PROTOTIPO - TELA INICIAL}{0.28}{fig/proto-home.png}{Os autores (2025)}{hu10-home}{}{}

\end{apendicesenv}
