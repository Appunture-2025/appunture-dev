% Apêndices do TCC Appunture
% Encoding: UTF-8

\begin{apendicesenv}

\partapendices

% ==============================================================================
% APÊNDICE A - DIAGRAMA DE CASOS DE USO
% ==============================================================================
\chapter{DIAGRAMA DE CASOS DE USO}
\label{apendice:casos-uso}

O diagrama de casos de uso apresenta as principais funcionalidades do sistema Appunture e suas interações com os atores do sistema.

\figura{DIAGRAMA DE CASOS DE USO DO SISTEMA APPUNTURE}{0.85}{fig/casos-uso.png}{Os autores (2025)}{casos-uso}{}{}

% ==============================================================================
% APÊNDICE B - DIAGRAMA DE CLASSES
% ==============================================================================
\chapter{DIAGRAMA DE CLASSES}
\label{apendice:classes}

O diagrama de classes representa a estrutura estática do sistema, mostrando as principais entidades e seus relacionamentos.

\figura{DIAGRAMA DE CLASSES DO SISTEMA APPUNTURE}{0.9}{fig/classes.png}{Os autores (2025)}{classes}{}{}

% ==============================================================================
% APÊNDICE C - DIAGRAMAS DE SEQUÊNCIA
% ==============================================================================
\chapter{DIAGRAMAS DE SEQUÊNCIA}
\label{apendice:sequencia}

Os diagramas de sequência apresentam o fluxo de interação entre os componentes do sistema para cada história de usuário implementada.

% ------------------------------------------------------------------------------
\section{HU-01: Busca de Pontos de Acupuntura}
\label{apendice:seq-busca}

Este diagrama representa o fluxo de busca de pontos de acupuntura no sistema.

\figura{DIAGRAMA DE SEQUÊNCIA - BUSCA DE PONTOS}{0.85}{fig/seq-busca.png}{Os autores (2025)}{seq-busca}{}{}

% ------------------------------------------------------------------------------
\section{HU-02: Detalhes do Ponto de Acupuntura}
\label{apendice:seq-detalhes}

Este diagrama representa o fluxo de visualização dos detalhes de um ponto de acupuntura.

\figura{DIAGRAMA DE SEQUÊNCIA - DETALHES DO PONTO}{0.85}{fig/seq-detalhes.png}{Os autores (2025)}{seq-detalhes}{}{}

% ------------------------------------------------------------------------------
\section{HU-03: Atlas Visual Interativo}
\label{apendice:seq-atlas}

Este diagrama representa o fluxo de interação com o atlas visual do corpo humano.

\figura{DIAGRAMA DE SEQUÊNCIA - ATLAS VISUAL}{0.85}{fig/seq-atlas.png}{Os autores (2025)}{seq-atlas}{}{}

% ------------------------------------------------------------------------------
\section{HU-04: Gerenciamento de Favoritos}
\label{apendice:seq-favoritos}

Este diagrama representa o fluxo de adição e remoção de pontos favoritos.

\figura{DIAGRAMA DE SEQUÊNCIA - FAVORITOS}{0.85}{fig/seq-favoritos.png}{Os autores (2025)}{seq-favoritos}{}{}

% ------------------------------------------------------------------------------
\section{HU-05: Anotações Clínicas}
\label{apendice:seq-anotacao}

Este diagrama representa o fluxo de criação e gerenciamento de anotações clínicas.

\figura{DIAGRAMA DE SEQUÊNCIA - ANOTAÇÕES CLÍNICAS}{0.85}{fig/seq-anotacao.png}{Os autores (2025)}{seq-anotacao}{}{}

% ------------------------------------------------------------------------------
\section{HU-06: Sincronização de Dados}
\label{apendice:seq-sync}

Este diagrama representa o fluxo de sincronização de dados entre dispositivo e nuvem.

\figura{DIAGRAMA DE SEQUÊNCIA - SINCRONIZAÇÃO}{0.85}{fig/seq-sync.png}{Os autores (2025)}{seq-sync}{}{}

% ------------------------------------------------------------------------------
\section{HU-07: Assistente de Inteligência Artificial}
\label{apendice:seq-ia}

Este diagrama representa o fluxo de interação com o assistente de IA do sistema.

\figura{DIAGRAMA DE SEQUÊNCIA - ASSISTENTE IA}{0.85}{fig/seq-ia.png}{Os autores (2025)}{seq-ia}{}{}

% ------------------------------------------------------------------------------
\section{HU-08: Mapeamento de Sintomas}
\label{apendice:seq-mapper}

Este diagrama representa o fluxo de mapeamento de sintomas para pontos de acupuntura.

\figura{DIAGRAMA DE SEQUÊNCIA - MAPEAMENTO DE SINTOMAS}{0.85}{fig/seq-mapper.png}{Os autores (2025)}{seq-mapper}{}{}

% ------------------------------------------------------------------------------
\section{HU-09: Autenticação de Usuário}
\label{apendice:seq-auth}

Os diagramas a seguir representam os fluxos de autenticação do sistema, incluindo login com email/senha, cadastro de novo usuário e login social via Google.

\figura{DIAGRAMA DE SEQUÊNCIA - LOGIN COM EMAIL/SENHA}{0.75}{fig/seq-loginemailsenha.png}{Os autores (2025)}{seq-login}{}{}

\figura{DIAGRAMA DE SEQUÊNCIA - CADASTRO DE USUÁRIO}{0.75}{fig/seq-cadastro.png}{Os autores (2025)}{seq-cadastro}{}{}

\figura{DIAGRAMA DE SEQUÊNCIA - LOGIN SOCIAL (GOOGLE)}{0.75}{fig/seq-loginsocial.png}{Os autores (2025)}{seq-google}{}{}

% ------------------------------------------------------------------------------
\section{HU-10: Navegação por Meridianos}
\label{apendice:seq-meridians}

Os diagramas a seguir representam os fluxos de navegação pelos meridianos do sistema.

\figura{DIAGRAMA DE SEQUÊNCIA - LISTAR MERIDIANOS}{0.75}{fig/seq-listarmeridianos.png}{Os autores (2025)}{seq-listarmeridianos}{}{}

\figura{DIAGRAMA DE SEQUÊNCIA - ACESSAR MERIDIANO}{0.75}{fig/seq-acessarmeridiano.png}{Os autores (2025)}{seq-acessarmeridiano}{}{}

\figura{DIAGRAMA DE SEQUÊNCIA - INFORMAÇÕES DO MERIDIANO}{0.70}{fig/seq-infosmeridiano.png}{Os autores (2025)}{seq-infosmeridiano}{}{}

% ------------------------------------------------------------------------------
\section{HU-11: Configurações e Personalização}
\label{apendice:seq-config}

Os diagramas a seguir representam os fluxos de configuração e personalização do aplicativo.

\figura{DIAGRAMA DE SEQUÊNCIA - ALTERNAR TEMA}{0.75}{fig/seq-apptheme.png}{Os autores (2025)}{seq-theme}{}{}

\figura{DIAGRAMA DE SEQUÊNCIA - NOTIFICAÇÕES}{0.75}{fig/seq-onoffnotifs.png}{Os autores (2025)}{seq-notifs}{}{}

\figura{DIAGRAMA DE SEQUÊNCIA - MODO OFFLINE}{0.70}{fig/seq-offline.png}{Os autores (2025)}{seq-offline}{}{}

\figura{DIAGRAMA DE SEQUÊNCIA - LOGOUT}{0.75}{fig/seq-logout.png}{Os autores (2025)}{seq-logout}{}{}

\figura{DIAGRAMA DE SEQUÊNCIA - STATUS DE SINCRONIZAÇÃO}{0.70}{fig/seq-statussync.png}{Os autores (2025)}{seq-statussync}{}{}

% ==============================================================================
% APÊNDICE D - MODELO LÓGICO DO BANCO DE DADOS
% ==============================================================================
\chapter{MODELO LÓGICO DO BANCO DE DADOS}
\label{apendice:modelo-logico}

O modelo lógico do banco de dados apresenta a estrutura de armazenamento de dados do sistema Appunture.

\figura{MODELO LÓGICO DO BANCO DE DADOS}{0.9}{fig/modelo-logico.png}{Os autores (2025)}{modelo-logico}{}{}

% ==============================================================================
% APÊNDICE E - HISTÓRIAS DE USUÁRIO
% ==============================================================================
\chapter{HISTÓRIAS DE USUÁRIO}
\label{apendice:historias}

Este apêndice apresenta as histórias de usuário detalhadas do sistema Appunture, incluindo critérios de aceitação.

% ------------------------------------------------------------------------------
\section{HU-01: Busca de Pontos de Acupuntura}
\label{hu:busca}

\textbf{Como} estudante ou profissional de acupuntura, \\
\textbf{Quero} buscar pontos de acupuntura por nome, código ou função, \\
\textbf{Para que} eu possa encontrar rapidamente informações sobre pontos específicos.

\subsection*{Critérios de Aceitação}

\begin{enumerate}
    \item O sistema deve permitir busca por nome do ponto em português ou pinyin
    \item O sistema deve permitir busca por código alfanumérico (ex: LU1, ST36)
    \item O sistema deve exibir resultados em tempo real conforme o usuário digita
    \item O sistema deve destacar os termos buscados nos resultados
    \item A busca deve retornar resultados em menos de 500ms
    \item O sistema deve sugerir correções para termos digitados incorretamente
\end{enumerate}

\figura{PROTÓTIPO - BUSCA DE PONTOS}{0.30}{fig/proto-busca.png}{Os autores (2025)}{hu01-busca}{}{}

% ------------------------------------------------------------------------------
\section{HU-02: Detalhes do Ponto de Acupuntura}
\label{hu:detalhes}

\textbf{Como} estudante ou profissional de acupuntura, \\
\textbf{Quero} visualizar informações detalhadas de um ponto de acupuntura, \\
\textbf{Para que} eu possa estudar suas características e aplicações clínicas.

\subsection*{Critérios de Aceitação}

\begin{enumerate}
    \item O sistema deve exibir nome em português, pinyin e caracteres chineses
    \item O sistema deve mostrar a localização anatômica precisa
    \item O sistema deve apresentar as funções terapêuticas do ponto
    \item O sistema deve listar as indicações clínicas baseadas na MTC
    \item O sistema deve exibir imagem ilustrativa da localização
    \item O sistema deve mostrar técnicas de inserção recomendadas
    \item O sistema deve indicar precauções e contraindicações
\end{enumerate}

\figura{PROTÓTIPO - DETALHES DO PONTO}{0.30}{fig/proto-detalhes.png}{Os autores (2025)}{hu02-detalhes}{}{}

% ------------------------------------------------------------------------------
\section{HU-03: Atlas Visual Interativo}
\label{hu:atlas}

\textbf{Como} estudante ou profissional de acupuntura, \\
\textbf{Quero} navegar por um atlas visual do corpo humano, \\
\textbf{Para que} eu possa localizar visualmente os pontos de acupuntura.

\subsection*{Critérios de Aceitação}

\begin{enumerate}
    \item O sistema deve exibir modelo anatômico interativo do corpo humano
    \item O sistema deve permitir zoom e rotação do modelo
    \item O sistema deve destacar os meridianos no modelo
    \item O sistema deve permitir seleção de pontos diretamente no modelo
    \item O sistema deve filtrar pontos por meridiano ou região corporal
    \item O sistema deve sincronizar a visualização com a busca textual
\end{enumerate}

\figura{PROTÓTIPO - ATLAS VISUAL}{0.30}{fig/proto-mapa.png}{Os autores (2025)}{hu03-mapa}{}{}

% ------------------------------------------------------------------------------
\section{HU-04: Gerenciamento de Favoritos}
\label{hu:favoritos}

\textbf{Como} estudante ou profissional de acupuntura, \\
\textbf{Quero} salvar pontos de acupuntura como favoritos, \\
\textbf{Para que} eu possa acessá-los rapidamente no futuro.

\subsection*{Critérios de Aceitação}

\begin{enumerate}
    \item O sistema deve permitir adicionar pontos à lista de favoritos
    \item O sistema deve permitir remover pontos da lista de favoritos
    \item O sistema deve exibir lista de favoritos organizada
    \item O sistema deve permitir criar categorias personalizadas de favoritos
    \item O sistema deve sincronizar favoritos com a conta do usuário
    \item O sistema deve permitir exportar lista de favoritos
\end{enumerate}

\figura{PROTÓTIPO - FAVORITOS}{0.30}{fig/proto-favoritos.png}{Os autores (2025)}{hu04-favoritos}{}{}

% ------------------------------------------------------------------------------
\section{HU-05: Anotações Clínicas}
\label{hu:anotacoes}

\textbf{Como} profissional de acupuntura, \\
\textbf{Quero} criar anotações pessoais sobre pontos de acupuntura, \\
\textbf{Para que} eu possa registrar observações clínicas e experiências práticas.

\subsection*{Critérios de Aceitação}

\begin{enumerate}
    \item O sistema deve permitir criar anotações vinculadas a pontos específicos
    \item O sistema deve suportar formatação básica de texto nas anotações
    \item O sistema deve permitir adicionar tags às anotações
    \item O sistema deve permitir buscar anotações por conteúdo ou tags
    \item O sistema deve exibir histórico de anotações com data e hora
    \item O sistema deve permitir editar e excluir anotações
    \item O sistema deve sincronizar anotações com a conta do usuário
\end{enumerate}

% ------------------------------------------------------------------------------
\section{HU-06: Sincronização de Dados}
\label{hu:sync}

\textbf{Como} usuário do aplicativo, \\
\textbf{Quero} sincronizar meus dados entre dispositivos, \\
\textbf{Para que} eu possa acessar minhas informações em qualquer lugar.

\subsection*{Critérios de Aceitação}

\begin{enumerate}
    \item O sistema deve sincronizar automaticamente quando conectado à internet
    \item O sistema deve funcionar offline com dados em cache
    \item O sistema deve resolver conflitos de sincronização automaticamente
    \item O sistema deve notificar o usuário sobre status da sincronização
    \item O sistema deve permitir sincronização manual sob demanda
    \item O sistema deve manter histórico de sincronizações
\end{enumerate}

% ------------------------------------------------------------------------------
\section{HU-07: Assistente de Inteligência Artificial}
\label{hu:ia}

\textbf{Como} estudante ou profissional de acupuntura, \\
\textbf{Quero} consultar um assistente de IA sobre acupuntura, \\
\textbf{Para que} eu possa obter respostas contextualizadas às minhas dúvidas.

\subsection*{Critérios de Aceitação}

\begin{enumerate}
    \item O sistema deve permitir fazer perguntas em linguagem natural
    \item O sistema deve fornecer respostas baseadas em literatura de acupuntura
    \item O sistema deve citar fontes quando apropriado
    \item O sistema deve manter contexto da conversa
    \item O sistema deve sugerir pontos relacionados às perguntas
    \item O sistema deve indicar limitações e recomendar consulta profissional
\end{enumerate}

\figura{PROTÓTIPO - ASSISTENTE IA}{0.30}{fig/proto-chat.png}{Os autores (2025)}{hu07-chat}{}{}

% ------------------------------------------------------------------------------
\section{HU-08: Mapeamento de Sintomas}
\label{hu:mapper}

\textbf{Como} profissional de acupuntura, \\
\textbf{Quero} mapear sintomas para pontos de acupuntura recomendados, \\
\textbf{Para que} eu possa auxiliar no planejamento de tratamentos.

\subsection*{Critérios de Aceitação}

\begin{enumerate}
    \item O sistema deve permitir selecionar múltiplos sintomas
    \item O sistema deve sugerir pontos de acupuntura relacionados aos sintomas
    \item O sistema deve indicar nível de evidência para cada sugestão
    \item O sistema deve permitir filtrar sugestões por meridiano
    \item O sistema deve exibir combinações de pontos recomendadas
    \item O sistema deve salvar histórico de mapeamentos realizados
    \item O sistema deve incluir aviso de uso educacional apenas
\end{enumerate}

\figura{PROTÓTIPO - LISTA DE PONTOS}{0.30}{fig/proto-meridianos.png}{Os autores (2025)}{hu08-meridianos}{}{}

% ------------------------------------------------------------------------------
\section{HU-09: Autenticação de Usuário}
\label{hu:autenticacao}

\textbf{Como} usuário do aplicativo, \\
\textbf{Quero} criar uma conta e realizar login de forma segura, \\
\textbf{Para que} eu possa acessar recursos personalizados e sincronizar meus dados.

\subsection*{Critérios de Aceitação}

\begin{enumerate}
    \item O sistema deve permitir cadastro com nome, email e senha
    \item O sistema deve validar força mínima da senha (6 caracteres)
    \item O sistema deve exigir confirmação de senha no cadastro
    \item O sistema deve permitir login com email e senha
    \item O sistema deve oferecer login social via Google
    \item O sistema deve permitir recuperação de senha por email
    \item O sistema deve manter sessão ativa entre execuções do app
    \item O sistema deve permitir acesso como visitante com funcionalidades limitadas
\end{enumerate}

\figura{PROTÓTIPO - LOGIN}{0.30}{fig/proto-login.png}{Os autores (2025)}{hu09-login}{}{}

\figura{PROTÓTIPO - CADASTRO}{0.30}{fig/proto-cadastro.png}{Os autores (2025)}{hu09-cadastro}{}{}

% ------------------------------------------------------------------------------
\section{HU-10: Navegação por Meridianos}
\label{hu:meridianos}

\textbf{Como} estudante ou profissional de acupuntura, \\
\textbf{Quero} navegar pelos meridianos da medicina tradicional chinesa, \\
\textbf{Para que} eu possa estudar os pontos organizados por canal energético.

\subsection*{Critérios de Aceitação}

\begin{enumerate}
    \item O sistema deve listar os 12 meridianos principais e 2 vasos extraordinários
    \item O sistema deve exibir nome em português, pinyin e caracteres chineses
    \item O sistema deve mostrar o elemento Wu Xing associado a cada meridiano
    \item O sistema deve indicar o horário de maior atividade energética
    \item O sistema deve exibir o órgão relacionado ao meridiano
    \item O sistema deve mostrar a quantidade de pontos em cada meridiano
    \item O sistema deve permitir visualizar todos os pontos de um meridiano
    \item O sistema deve usar cores características para cada elemento
\end{enumerate}

\figura{PROTÓTIPO - NAVEGAÇÃO POR MERIDIANOS}{0.30}{fig/proto-meridianos.png}{Os autores (2025)}{hu10-meridianos}{}{}

% ------------------------------------------------------------------------------
\section{HU-11: Configurações e Personalização}
\label{hu:configuracoes}

\textbf{Como} usuário do aplicativo, \\
\textbf{Quero} personalizar as configurações do aplicativo, \\
\textbf{Para que} eu possa adaptar a experiência às minhas preferências.

\subsection*{Critérios de Aceitação}

\begin{enumerate}
    \item O sistema deve permitir alternar entre tema claro, escuro e automático
    \item O sistema deve salvar a preferência de tema do usuário
    \item O sistema deve permitir ativar ou desativar notificações push
    \item O sistema deve exibir informações da conta do usuário logado
    \item O sistema deve mostrar status de sincronização e última atualização
    \item O sistema deve permitir ativar modo offline para economia de dados
    \item O sistema deve permitir logout com confirmação
    \item O sistema deve exibir versão do aplicativo e informações de suporte
\end{enumerate}

\figura{PROTÓTIPO - TELA INICIAL}{0.30}{fig/proto-home.png}{Os autores (2025)}{hu11-home}{}{}

\end{apendicesenv}
