% ----------------------------------------------------------
% APÊNDICES
% ----------------------------------------------------------

% APÊNDICE A - DIAGRAMA DE CASOS DE USO
\chapter{DIAGRAMA DE CASOS DE USO} \label{ap:casos-uso}

% \figura{DIAGRAMA DE CASOS DE USO}{0.9}{fig/diagrama-casos-uso.png}{Os autores (2025)}{diagrama-casos-uso}{}{}

% APÊNDICE B - HISTÓRIAS DE USUÁRIO
\chapter{HISTÓRIAS DE USUÁRIO} \label{ap:historias}

HU01 -- Buscar pontos por sintoma

HU02 -- Visualizar detalhes do ponto

HU03 -- Navegar pelo atlas anatômico

HU04 -- Favoritar ponto de acupuntura

HU05 -- Adicionar anotação pessoal

HU06 -- Sincronizar dados entre dispositivos

HU07 -- Consultar assistente de IA

HU08 -- Mapear ponto no atlas (administrador)

\section*{HU01 -- Buscar pontos por sintoma}

\textbf{SENDO} um estudante ou profissional de acupuntura

\textbf{QUERO} buscar pontos de acupuntura por sintomas

\textbf{PARA} encontrar rapidamente os pontos indicados para tratamento

\textbf{DESENHO DA(S) TELA(S):}

% \figura{PROTÓTIPO DE BUSCA POR SINTOMA}{0.8}{fig/proto-busca-sintoma.png}{Os autores (2025)}{proto-busca}{}{}

\textbf{CRITÉRIOS DE ACEITAÇÃO}

\begin{enumerate}
    \item Deve aceitar entrada de texto livre descrevendo sintomas
    \item Deve utilizar IA para interpretar a intenção do usuário
    \item Deve ordenar resultados por relevância
    \item Deve requerer conexão com a internet
    \item Deve exibir indicações de cada ponto nos resultados
\end{enumerate}

\textbf{CRITÉRIOS DE ACEITAÇÃO -- DETALHAMENTO}

\textbf{Critério de contexto} (Válido como premissa para todos os critérios):

Dado que desejo buscar pontos de acupuntura

E acessei a tela de busca do aplicativo

\textbf{1. Deve aceitar entrada de texto livre descrevendo sintomas}

Dado que desejo buscar por um sintoma

Quando digito ``dor de cabeça'' no campo de busca

Então o sistema realiza a busca e exibe os pontos relacionados

\textbf{2. Deve utilizar IA para interpretar a intenção}

Dado que digitei ``dor de cabesa'' com erro ortográfico

Quando seleciono buscar

Então o assistente de IA identifica a intenção

E retorna resultados relevantes baseados no contexto

\textbf{3. Deve ordenar resultados por relevância}

Dado que a busca retornou múltiplos pontos

Quando visualizo os resultados

Então os pontos mais relevantes aparecem primeiro

E cada resultado exibe um \textit{score} de relevância

\textbf{4. Deve requerer conexão com a internet}

Dado que o dispositivo está sem conexão com a internet

Quando tento realizar uma busca inteligente

Então o sistema informa que é necessária conexão

E sugere navegação manual pelo atlas

\textbf{5. Deve exibir indicações de cada ponto nos resultados}

Dado que a busca retornou resultados

Quando visualizo a lista de pontos

Então cada ponto exibe seu código, nome e principais indicações

\section*{HU02 -- Visualizar detalhes do ponto}

\textbf{SENDO} um usuário do aplicativo

\textbf{QUERO} visualizar os detalhes completos de um ponto de acupuntura

\textbf{PARA} conhecer sua localização, indicações e técnicas de punção

\textbf{DESENHO DA(S) TELA(S):}

% \figura{PROTÓTIPO DE DETALHES DO PONTO}{0.8}{fig/proto-detalhes-ponto.png}{Os autores (2025)}{proto-detalhes}{}{}

\textbf{CRITÉRIOS DE ACEITAÇÃO}

\begin{enumerate}
    \item Deve exibir todas as informações do ponto (código, nomes, meridiano)
    \item Deve exibir localização anatômica detalhada
    \item Deve exibir indicações clínicas baseadas no padrão WHO
    \item Deve exibir funções energéticas segundo a MTC
    \item Deve mostrar a posição do ponto no atlas SVG
    \item Deve permitir favoritar o ponto
    \item Deve permitir adicionar anotações pessoais
\end{enumerate}

\textbf{CRITÉRIOS DE ACEITAÇÃO -- DETALHAMENTO}

\textbf{1. Deve exibir todas as informações do ponto}

Dado que selecionei o ponto LU-7 na lista de resultados

Quando a tela de detalhes é carregada

Então o sistema exibe: código (LU-7), nome em português (Lieque), nome em pinyin (Lièquē), meridiano (Pulmão)

\textbf{2. Deve exibir localização anatômica detalhada}

Dado que estou visualizando os detalhes do ponto

Quando verifico a seção de localização

Então o sistema exibe a descrição anatômica conforme padrão WHO

E indica a profundidade de punção recomendada

\textbf{3. Deve mostrar a posição do ponto no atlas SVG}

Dado que estou visualizando os detalhes do ponto

Quando verifico a seção do atlas

Então o sistema exibe a vista anatômica correspondente

E destaca a posição exata do ponto no corpo

\section*{HU03 -- Navegar pelo atlas anatômico}

\textbf{SENDO} um usuário do aplicativo

\textbf{QUERO} navegar pelas diferentes vistas do atlas anatômico interativo

\textbf{PARA} localizar visualmente os pontos de acupuntura no corpo humano

\textbf{DESENHO DA(S) TELA(S):}

% \figura{PROTÓTIPO DO ATLAS ANATÔMICO}{0.8}{fig/proto-atlas.png}{Os autores (2025)}{proto-atlas}{}{}

\textbf{CRITÉRIOS DE ACEITAÇÃO}

\begin{enumerate}
    \item Deve permitir selecionar entre as 15 vistas disponíveis
    \item Deve exibir os pontos mapeados como marcadores clicáveis
    \item Deve permitir aplicar \textit{zoom} e arrastar a imagem
    \item Deve exibir detalhes ao tocar em um marcador
    \item Deve funcionar em modo \textit{offline} (navegação manual)
\end{enumerate}

\textbf{CRITÉRIOS DE ACEITAÇÃO -- DETALHAMENTO}

\textbf{1. Deve permitir selecionar entre as 15 vistas disponíveis}

Dado que acessei a tela do atlas anatômico

Quando visualizo as opções de vistas

Então o sistema exibe: corpo anterior, corpo posterior, cabeça frontal, cabeça lateral, mão dorsal, mão palmar, pé dorsal, pé plantar, entre outras

\textbf{2. Deve exibir os pontos mapeados como marcadores clicáveis}

Dado que selecionei a vista ``corpo anterior''

Quando a imagem SVG é carregada

Então o sistema exibe marcadores nos pontos que possuem coordenadas mapeadas

E cada marcador é interativo

\textbf{3. Deve exibir detalhes ao tocar em um marcador}

Dado que visualizo os marcadores no atlas

Quando toco em um marcador de ponto

Então o sistema exibe um \textit{tooltip} com código e nome do ponto

E permite navegar para a tela de detalhes

\section*{HU04 -- Favoritar ponto de acupuntura}

\textbf{SENDO} um usuário autenticado

\textbf{QUERO} marcar pontos de acupuntura como favoritos

\textbf{PARA} acessá-los rapidamente em consultas futuras

\textbf{CRITÉRIOS DE ACEITAÇÃO}

\begin{enumerate}
    \item Deve permitir favoritar/desfavoritar com um toque
    \item Deve salvar favoritos no banco local imediatamente
    \item Deve sincronizar favoritos quando houver conexão
    \item Deve listar todos os favoritos em tela dedicada
\end{enumerate}

\textbf{CRITÉRIOS DE ACEITAÇÃO -- DETALHAMENTO}

\textbf{1. Deve permitir favoritar/desfavoritar com um toque}

Dado que estou visualizando os detalhes de um ponto

Quando toco no ícone de coração/estrela

Então o sistema alterna o estado de favorito do ponto

E atualiza o ícone visualmente

\textbf{2. Deve salvar favoritos no banco local imediatamente}

Dado que favoritei um ponto

Quando fecho e reabro o aplicativo

Então o ponto permanece marcado como favorito

\textbf{3. Deve sincronizar favoritos quando houver conexão}

Dado que o dispositivo está conectado à internet

Quando adiciono ou removo um favorito

Então o sistema sincroniza a alteração com o \textit{backend}

E mantém consistência entre dispositivos

\section*{HU05 -- Adicionar anotação pessoal}

\textbf{SENDO} um usuário autenticado

\textbf{QUERO} adicionar anotações pessoais aos pontos de acupuntura

\textbf{PARA} registrar observações de estudo e experiências clínicas

\textbf{CRITÉRIOS DE ACEITAÇÃO}

\begin{enumerate}
    \item Deve permitir criar anotação em texto livre
    \item Deve associar anotação ao ponto específico
    \item Deve permitir editar e excluir anotações
    \item Deve exibir data de criação/modificação
    \item Deve sincronizar anotações entre dispositivos
\end{enumerate}

\textbf{CRITÉRIOS DE ACEITAÇÃO -- DETALHAMENTO}

\textbf{1. Deve permitir criar anotação em texto livre}

Dado que estou na tela de detalhes do ponto LU-5

Quando seleciono ``adicionar anotação''

Então o sistema exibe campo de texto para digitação

E permite salvar a anotação

\textbf{2. Deve permitir editar e excluir anotações}

Dado que já tenho uma anotação salva no ponto

Quando toco na anotação existente

Então o sistema exibe opções de editar ou excluir

\section*{HU06 -- Sincronizar dados entre dispositivos}

\textbf{SENDO} um usuário autenticado

\textbf{QUERO} que meus dados sejam sincronizados automaticamente

\textbf{PARA} não perder anotações e favoritos ao trocar de dispositivo

\textbf{CRITÉRIOS DE ACEITAÇÃO}

\begin{enumerate}
    \item Deve sincronizar automaticamente quando houver conexão
    \item Deve salvar dados localmente quando \textit{offline}
    \item Deve resolver conflitos mantendo versão mais recente
    \item Deve notificar status da sincronização
    \item Não deve impactar performance do aplicativo
\end{enumerate}

\textbf{CRITÉRIOS DE ACEITAÇÃO -- DETALHAMENTO}

\textbf{1. Deve sincronizar automaticamente quando houver conexão}

Dado que o dispositivo está conectado à internet

Quando abro o aplicativo

Então o sistema verifica e sincroniza dados automaticamente

\textbf{2. Deve salvar dados localmente quando offline}

Dado que o dispositivo está sem conexão

Quando adiciono um favorito ou anotação

Então o sistema salva no banco SQLite local

E marca para sincronização posterior

\textbf{3. Deve resolver conflitos mantendo versão mais recente}

Dado que editei a mesma anotação em dois dispositivos

Quando ocorre a sincronização

Então o sistema mantém a versão com \textit{timestamp} mais recente

\section*{HU07 -- Consultar assistente de IA}

\textbf{SENDO} um usuário do aplicativo

\textbf{QUERO} consultar um assistente de IA sobre acupuntura

\textbf{PARA} obter orientações educacionais sobre pontos e tratamentos

\textbf{CRITÉRIOS DE ACEITAÇÃO}

\begin{enumerate}
    \item Deve aceitar perguntas em linguagem natural
    \item Deve fornecer respostas baseadas na base de dados do sistema (RAG)
    \item Deve incluir \textit{disclaimer} sobre uso educacional
    \item Deve estar disponível apenas com conexão à internet
    \item Deve referenciar os pontos mencionados nas respostas
\end{enumerate}

\textbf{CRITÉRIOS DE ACEITAÇÃO -- DETALHAMENTO}

\textbf{1. Deve aceitar perguntas em linguagem natural}

Dado que acessei a tela do assistente de IA

Quando digito ``quais pontos são indicados para ansiedade?''

Então o sistema processa a pergunta e gera uma resposta

\textbf{2. Deve fornecer respostas baseadas na base de dados do sistema}

Dado que fiz uma pergunta ao assistente

Quando a resposta é gerada

Então o sistema utiliza dados dos pontos cadastrados no sistema

E menciona pontos específicos com seus códigos

\textbf{3. Deve incluir disclaimer sobre uso educacional}

Dado que recebi uma resposta do assistente

Quando visualizo o conteúdo

Então a resposta inclui aviso de que é apenas para fins educacionais

E não substitui orientação profissional

\section*{HU08 -- Mapear ponto no atlas (administrador)}

\textbf{SENDO} um administrador do sistema

\textbf{QUERO} mapear as coordenadas de um ponto no atlas SVG

\textbf{PARA} permitir a visualização correta do ponto pelos usuários

\textbf{DESENHO DA(S) TELA(S):}

% \figura{PROTÓTIPO DO POINT MAPPER}{0.8}{fig/proto-point-mapper.png}{Os autores (2025)}{proto-mapper}{}{}

\textbf{CRITÉRIOS DE ACEITAÇÃO}

\begin{enumerate}
    \item Deve permitir selecionar um ponto da lista
    \item Deve permitir selecionar a vista do atlas
    \item Deve permitir clicar na imagem para definir coordenadas
    \item Deve exibir prévia do marcador na posição selecionada
    \item Deve salvar coordenadas no formato percentual
    \item Deve exportar mapeamentos em formato JSON
\end{enumerate}

\textbf{CRITÉRIOS DE ACEITAÇÃO -- DETALHAMENTO}

\textbf{1. Deve permitir selecionar um ponto da lista}

Dado que acessei a ferramenta Point Mapper

Quando seleciono o meridiano ``LU'' e o ponto ``LU-1''

Então o sistema carrega as informações do ponto

\textbf{2. Deve permitir clicar na imagem para definir coordenadas}

Dado que selecionei um ponto e uma vista do atlas

Quando clico na posição anatômica correta na imagem

Então o sistema registra as coordenadas X e Y em percentual

\textbf{3. Deve exportar mapeamentos em formato JSON}

Dado que mapeei múltiplos pontos

Quando seleciono ``exportar''

Então o sistema gera arquivo JSON com todas as coordenadas

E salva no formato compatível com o seed do sistema

% APÊNDICE C - DIAGRAMA DE CLASSES DE ANÁLISE
\chapter{DIAGRAMA DE CLASSES DE ANÁLISE} \label{ap:classes}

% \figura{DIAGRAMA DE CLASSES DE ANÁLISE}{0.9}{fig/diagrama-classes.png}{Os autores (2025)}{diagrama-classes}{}{}

\section*{CLASSES PRINCIPAIS}

\begin{itemize}
    \item \textbf{User}: Representa os usuários do sistema, com atributos como id, nome, email, senha (hash), tipo de perfil (estudante, profissional, administrador) e data de cadastro.
    
    \item \textbf{AcupuncturePoint}: Representa um ponto de acupuntura, contendo código (ex: LU-7), nome em português, nome em pinyin, localização anatômica, profundidade de punção, indicações, contraindicações, funções energéticas e coordenadas no atlas.
    
    \item \textbf{Meridian}: Representa os meridianos da MTC, com código, nome em português, nome em chinês, elemento associado e característica yin/yang.
    
    \item \textbf{Symptom}: Representa sintomas clínicos que podem ser associados aos pontos de acupuntura, com nome, categoria e sinônimos.
    
    \item \textbf{Favorite}: Representa a relação entre usuário e pontos favoritados, com id do usuário, id do ponto, data de criação e status de sincronização.
    
    \item \textbf{Note}: Representa as anotações pessoais do usuário sobre pontos específicos, com conteúdo em texto, data de criação, data de modificação e status de sincronização.
    
    \item \textbf{BodyMapCoords}: Representa as coordenadas de um ponto em uma vista específica do atlas, com referência à vista, coordenada X e coordenada Y em percentual.
\end{itemize}

% APÊNDICE D - DIAGRAMAS DE SEQUÊNCIA DE ANÁLISE
\chapter{DIAGRAMAS DE SEQUÊNCIA DE ANÁLISE} \label{ap:sequencia}

% \figura{DIAGRAMA DE SEQUÊNCIA - BUSCA POR SINTOMA}{0.9}{fig/seq-busca.png}{Os autores (2025)}{seq-busca}{}{}

% \figura{DIAGRAMA DE SEQUÊNCIA - SINCRONIZAÇÃO DE DADOS}{0.9}{fig/seq-sync.png}{Os autores (2025)}{seq-sync}{}{}

% \figura{DIAGRAMA DE SEQUÊNCIA - CONSULTA ASSISTENTE IA}{0.9}{fig/seq-ia.png}{Os autores (2025)}{seq-ia}{}{}

% \figura{DIAGRAMA DE SEQUÊNCIA - MAPEAMENTO DE PONTO}{0.9}{fig/seq-mapper.png}{Os autores (2025)}{seq-mapper}{}{}

\section*{FLUXO DE BUSCA POR SINTOMA (COM INTERNET)}

\begin{enumerate}
    \item Usuário digita sintoma no campo de busca
    \item Aplicativo envia consulta ao \textit{backend}
    \item \textit{Backend} processa com Spring AI + Google Gemini (RAG)
    \item IA retorna pontos relevantes com contexto
    \item Aplicativo ordena e exibe resultados ao usuário
\end{enumerate}

\section*{FLUXO DE SINCRONIZAÇÃO}

\begin{enumerate}
    \item Aplicativo detecta conexão com internet
    \item Aplicativo envia dados locais com \textit{flag} ``não sincronizado''
    \item \textit{Backend} processa e resolve conflitos por \textit{timestamp}
    \item \textit{Backend} retorna dados consolidados
    \item Aplicativo atualiza banco local e marca como sincronizado
\end{enumerate}

\section*{FLUXO DE CONSULTA AO ASSISTENTE IA}

\begin{enumerate}
    \item Usuário envia pergunta em linguagem natural
    \item \textit{Backend} busca pontos relevantes no banco (contexto)
    \item \textit{Backend} monta \textit{prompt} enriquecido com contexto (RAG)
    \item \textit{Backend} envia para Google Gemini via Spring AI
    \item Modelo retorna resposta contextualizada
    \item \textit{Backend} adiciona \textit{disclaimer} e envia ao aplicativo
\end{enumerate}

% APÊNDICE E - MODELO LÓGICO DE DADOS
\chapter{MODELO LÓGICO DE DADOS} \label{ap:modelo-dados}

% \figura{MODELO LÓGICO DE DADOS}{0.9}{fig/modelo-logico.png}{Os autores (2025)}{modelo-logico}{}{}

\section*{ESTRUTURA DO BANCO DE DADOS LOCAL (SQLITE)}

\begin{lstlisting}[language=SQL, caption={Estrutura do banco SQLite local}]
CREATE TABLE users (
    id INTEGER PRIMARY KEY AUTOINCREMENT,
    email TEXT UNIQUE NOT NULL,
    name TEXT NOT NULL,
    profile_type TEXT NOT NULL CHECK(
        profile_type IN ('student', 'professional', 'admin')),
    created_at DATETIME DEFAULT CURRENT_TIMESTAMP,
    synced_at DATETIME
);

CREATE TABLE meridians (
    id INTEGER PRIMARY KEY,
    code TEXT UNIQUE NOT NULL,
    name_pt TEXT NOT NULL,
    name_cn TEXT,
    element TEXT,
    yin_yang TEXT CHECK(yin_yang IN ('yin', 'yang'))
);

CREATE TABLE acupuncture_points (
    id INTEGER PRIMARY KEY,
    code TEXT UNIQUE NOT NULL,
    name_pt TEXT NOT NULL,
    name_pinyin TEXT,
    meridian_id INTEGER NOT NULL,
    location TEXT,
    depth TEXT,
    indications TEXT,
    contraindications TEXT,
    functions TEXT,
    special_characteristics TEXT,
    content_status TEXT DEFAULT 'draft',
    FOREIGN KEY (meridian_id) REFERENCES meridians(id)
);

CREATE TABLE body_map_coords (
    id INTEGER PRIMARY KEY AUTOINCREMENT,
    point_id INTEGER NOT NULL,
    view TEXT NOT NULL,
    x REAL NOT NULL,
    y REAL NOT NULL,
    mapped_at DATETIME DEFAULT CURRENT_TIMESTAMP,
    FOREIGN KEY (point_id) REFERENCES acupuncture_points(id)
);

CREATE TABLE symptoms (
    id INTEGER PRIMARY KEY,
    name TEXT NOT NULL,
    category TEXT,
    synonyms TEXT
);

CREATE TABLE point_symptoms (
    point_id INTEGER NOT NULL,
    symptom_id INTEGER NOT NULL,
    relevance_score REAL DEFAULT 1.0,
    PRIMARY KEY (point_id, symptom_id),
    FOREIGN KEY (point_id) REFERENCES acupuncture_points(id),
    FOREIGN KEY (symptom_id) REFERENCES symptoms(id)
);

CREATE TABLE favorites (
    id INTEGER PRIMARY KEY AUTOINCREMENT,
    user_id INTEGER NOT NULL,
    point_id INTEGER NOT NULL,
    created_at DATETIME DEFAULT CURRENT_TIMESTAMP,
    synced INTEGER DEFAULT 0,
    UNIQUE(user_id, point_id),
    FOREIGN KEY (user_id) REFERENCES users(id),
    FOREIGN KEY (point_id) REFERENCES acupuncture_points(id)
);

CREATE TABLE notes (
    id INTEGER PRIMARY KEY AUTOINCREMENT,
    user_id INTEGER NOT NULL,
    point_id INTEGER NOT NULL,
    content TEXT NOT NULL,
    created_at DATETIME DEFAULT CURRENT_TIMESTAMP,
    updated_at DATETIME,
    synced INTEGER DEFAULT 0,
    FOREIGN KEY (user_id) REFERENCES users(id),
    FOREIGN KEY (point_id) REFERENCES acupuncture_points(id)
);

-- Indices para otimizacao de buscas
CREATE INDEX idx_points_meridian ON acupuncture_points(meridian_id);
CREATE INDEX idx_points_code ON acupuncture_points(code);
CREATE INDEX idx_coords_point ON body_map_coords(point_id);
CREATE INDEX idx_favorites_user ON favorites(user_id);
CREATE INDEX idx_notes_user ON notes(user_id);
\end{lstlisting}

% APÊNDICE F - ESPECIFICAÇÃO DE APIs
\chapter{ESPECIFICAÇÃO DE APIs} \label{ap:apis}

\section{AUTENTICAÇÃO}

\subsection*{POST /api/auth/login}

\textbf{Descrição:} Autentica um usuário no sistema.

\textbf{Request Body:}
\begin{lstlisting}[language=json]
{
    "email": "usuario@email.com",
    "password": "senha123"
}
\end{lstlisting}

\textbf{Response (200 OK):}
\begin{lstlisting}[language=json]
{
    "token": "eyJhbGciOiJIUzI1NiIs...",
    "user": {
        "id": 1,
        "name": "Usuario",
        "email": "usuario@email.com",
        "profileType": "professional"
    }
}
\end{lstlisting}

\subsection*{POST /api/auth/register}

\textbf{Descrição:} Registra um novo usuário no sistema.

\textbf{Request Body:}
\begin{lstlisting}[language=json]
{
    "name": "Novo Usuario",
    "email": "novo@email.com",
    "password": "senha123",
    "profileType": "student"
}
\end{lstlisting}

\section{PONTOS DE ACUPUNTURA}

\subsection*{GET /api/points}

\textbf{Descrição:} Retorna lista de pontos de acupuntura.

\textbf{Query Parameters:}
\begin{itemize}
    \item \texttt{meridian}: Filtrar por código do meridiano (ex: LU, LI, ST)
    \item \texttt{search}: Termo de busca em nome ou indicações
    \item \texttt{symptoms}: Lista de sintomas separados por vírgula
    \item \texttt{page}: Número da página (paginação)
    \item \texttt{limit}: Quantidade por página (padrão: 20)
\end{itemize}

\textbf{Response (200 OK):}
\begin{lstlisting}[language=json]
{
    "data": [
        {
            "id": 1,
            "code": "LU-1",
            "namePt": "Zhongfu",
            "namePinyin": "Zhongfu",
            "meridian": "LU",
            "location": "Na regiao anterolateral...",
            "indications": ["tosse", "asma", "dor toracica"]
        }
    ],
    "pagination": {
        "page": 1,
        "limit": 20,
        "total": 362
    }
}
\end{lstlisting}

\subsection*{GET /api/points/\{code\}}

\textbf{Descrição:} Retorna detalhes completos de um ponto específico.

\textbf{Response (200 OK):}
\begin{lstlisting}[language=json]
{
    "id": 7,
    "code": "LU-7",
    "namePt": "Lieque",
    "namePinyin": "Lieque",
    "meridian": {
        "code": "LU",
        "namePt": "Pulmao",
        "element": "Metal"
    },
    "location": "Na face anterolateral do antebraco...",
    "depth": "0,3-0,5 cun obliquamente",
    "indications": ["cefaleia", "rigidez cervical", "tosse"],
    "contraindications": [],
    "functions": "Dispersa Vento, libera o Exterior...",
    "specialCharacteristics": "Ponto Luo, Ponto de Abertura...",
    "coordinates": {
        "x": 48.28,
        "y": 25.97
    },
    "bodyMapCoords": [
        {"view": "arm-anterior", "x": 48.28, "y": 25.97}
    ]
}
\end{lstlisting}

\section{SINCRONIZAÇÃO}

\subsection*{POST /api/sync}

\textbf{Descrição:} Sincroniza dados do dispositivo com o servidor.

\textbf{Request Body:}
\begin{lstlisting}[language=json]
{
    "lastSyncTimestamp": "2025-11-15T10:30:00Z",
    "favorites": [
        {"pointId": 7, "action": "add", "timestamp": "..."},
        {"pointId": 12, "action": "remove", "timestamp": "..."}
    ],
    "notes": [
        {
            "pointId": 7,
            "content": "Observacao clinica...",
            "action": "add",
            "timestamp": "..."
        }
    ]
}
\end{lstlisting}

\textbf{Response (200 OK):}
\begin{lstlisting}[language=json]
{
    "syncTimestamp": "2025-11-15T10:35:00Z",
    "favorites": [...],
    "notes": [...],
    "conflicts": []
}
\end{lstlisting}

\section{ASSISTENTE DE IA}

\subsection*{POST /api/assistant/query}

\textbf{Descrição:} Envia pergunta ao assistente de IA com RAG.

\textbf{Request Body:}
\begin{lstlisting}[language=json]
{
    "question": "Quais pontos sao indicados para ansiedade?"
}
\end{lstlisting}

\textbf{Response (200 OK):}
\begin{lstlisting}[language=json]
{
    "answer": "Para auxiliar no tratamento da ansiedade...",
    "referencedPoints": ["HT-7", "PC-6", "SP-6", "GV-20"],
    "disclaimer": "Esta informacao e apenas para fins educacionais..."
}
\end{lstlisting}

% APÊNDICE G - ARQUITETURA RAG DO ASSISTENTE DE IA
\chapter{ARQUITETURA RAG DO ASSISTENTE DE IA} \label{ap:rag}

\section*{CONFIGURAÇÃO DO SPRING AI COM GOOGLE GEMINI}

\begin{lstlisting}[language=Java, caption={Configuração do Spring AI}]
@Configuration
public class AIConfig {
    
    @Value("${spring.ai.vertex.ai.gemini.project-id}")
    private String projectId;
    
    @Value("${spring.ai.vertex.ai.gemini.location}")
    private String location;
    
    @Bean
    public ChatClient chatClient(VertexAiGeminiChatModel model) {
        return ChatClient.builder(model)
            .defaultSystem(ACUPUNCTURE_SYSTEM_PROMPT)
            .build();
    }
}
\end{lstlisting}

\section*{PIPELINE RAG (RETRIEVAL-AUGMENTED GENERATION)}

\begin{enumerate}
    \item \textbf{Recepção}: Usuário envia pergunta em linguagem natural
    \item \textbf{Retrieval}: \textit{Backend} busca pontos relevantes no banco de dados
    \item \textbf{Augmentation}: Monta \textit{prompt} enriquecido com contexto dos pontos
    \item \textbf{Generation}: Google Gemini 1.5 Flash gera resposta contextualizada
    \item \textbf{Pós-processamento}: Adiciona \textit{disclaimer} educacional e referências
\end{enumerate}

\section*{EXEMPLO DE FUNCIONAMENTO}

\textbf{Entrada do usuário:} ``dor de cabeça forte na testa''

\textbf{Processamento RAG:}
\begin{enumerate}
    \item Recepção: Pergunta enviada ao \textit{backend} via API REST
    \item Retrieval: Busca pontos com indicações relacionadas a cefaleia
    \item Augmentation: Contexto montado com GV-20, GB-14, ST-8, EX-HN3
    \item Generation: Gemini gera explicação sobre os pontos e suas indicações
    \item Resposta: ``Para dor de cabeça frontal, os pontos mais indicados são...''
\end{enumerate}

\section*{SYSTEM PROMPT DO ASSISTENTE}

\begin{lstlisting}[language=Java, caption={System prompt para contexto de acupuntura}]
private static final String ACUPUNCTURE_SYSTEM_PROMPT = """
    Voce e um assistente educacional especializado em acupuntura.
    Seu objetivo e auxiliar estudantes e profissionais a compreender
    os pontos de acupuntura e suas aplicacoes clinicas.
    
    IMPORTANTE: Suas respostas sao apenas para fins educacionais
    e NAO substituem a formacao profissional ou consulta medica.
    
    Sempre mencione os codigos dos pontos (ex: LU-7, ST-36)
    e suas principais indicacoes quando relevante.
    """;
\end{lstlisting}
