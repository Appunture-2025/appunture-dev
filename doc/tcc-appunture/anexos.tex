% ----------------------------------------------------------
% ANEXOS
% ----------------------------------------------------------

\chapter{DOCUMENTAÇÃO COMPLEMENTAR} \label{ax:doc}

Este anexo contém documentação complementar relevante para o projeto \textit{Appunture}.

\section{DIRETRIZES DA OMS PARA ACUPUNTURA}

A Organização Mundial da Saúde (OMS) publicou em 1999 as ``Guidelines on basic training and safety in acupuncture'', que estabelecem os requisitos mínimos para formação e prática segura da acupuntura. Os principais pontos abordados incluem:

\begin{itemize}
    \item Requisitos de formação para diferentes níveis de praticantes;
    \item Protocolos de biossegurança;
    \item Indicações e contraindicações gerais;
    \item Padronização de nomenclatura dos pontos;
    \item Técnicas de inserção de agulhas.
\end{itemize}

\section{NORMAS ISO APLICÁVEIS}

\subsection{ISO/IEC 25010:2011}

Esta norma define o modelo de qualidade para sistemas e produtos de software, estabelecendo oito características de qualidade:

\begin{enumerate}
    \item Adequação funcional
    \item Eficiência de desempenho
    \item Compatibilidade
    \item Usabilidade
    \item Confiabilidade
    \item Segurança
    \item Manutenibilidade
    \item Portabilidade
\end{enumerate}

\subsection{ISO 9241-210:2019}

Esta norma estabelece os princípios do design centrado no ser humano para sistemas interativos, incluindo:

\begin{itemize}
    \item Compreensão dos usuários, tarefas e ambientes;
    \item Envolvimento ativo dos usuários durante o design e desenvolvimento;
    \item Refinamento do design através de avaliação centrada no usuário;
    \item Processo iterativo;
    \item Consideração da experiência do usuário como um todo;
    \item Equipe multidisciplinar.
\end{itemize}

\section{PROJETO DE LEI 5983/2019}

O Projeto de Lei 5983/2019, em tramitação no Senado Federal, propõe a regulamentação do exercício da acupuntura no Brasil. Os principais pontos do projeto incluem:

\begin{itemize}
    \item Definição de acupuntura como prática de saúde;
    \item Requisitos de formação para exercício da profissão;
    \item Criação de registro profissional específico;
    \item Estabelecimento de código de ética;
    \item Fiscalização do exercício profissional.
\end{itemize}
