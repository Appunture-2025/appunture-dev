% ----------------------------------------------------------
% Capítulo 2 - FUNDAMENTAÇÃO TEÓRICA
% ----------------------------------------------------------
\chapter{FUNDAMENTAÇÃO TEÓRICA} \label{cha:fundamentacao}

A construção de uma solução digital eficaz voltada ao ensino e à prática da acupuntura exige respaldo teórico multidisciplinar, abrangendo desde os fundamentos da acessibilidade na saúde até os princípios de usabilidade e desenvolvimento de tecnologias educativas. Este capítulo busca contextualizar e embasar as escolhas metodológicas e funcionais do aplicativo \textit{Appunture}, discutindo as bases conceituais que orientam sua proposta. Para isso, são explorados os aspectos relacionados à democratização do acesso à informação técnica em saúde, o papel das Tecnologias Digitais de Informação e Comunicação (TDICs) no apoio à formação e à prática clínica, além da comparação com outras ferramentas disponíveis no mercado. A fundamentação aqui apresentada justifica a necessidade de um aplicativo como o \textit{Appunture} e oferece suporte acadêmico às decisões adotadas no desenvolvimento do projeto.

\section{ACESSIBILIDADE COMO DIREITO E FERRAMENTA DE IGUALDADE NA SAÚDE} \label{sec:acessibilidade}

A acessibilidade ao conhecimento e à prática segura em saúde deve ser compreendida como um direito fundamental, especialmente em contextos nos quais o acesso à formação e à informação de qualidade é desigual. No caso da acupuntura, esse desafio se intensifica pela diversidade de formações entre os praticantes --- que podem ser médicos, fisioterapeutas, terapeutas integrativos ou técnicos em estética --- e pela ausência de uma regulamentação federal unificada que determine diretrizes claras sobre a prática no Brasil. Essa lacuna acarreta diferentes formas de aprendizado, muitas vezes empíricas ou informalizadas, o que torna ainda mais necessária a disponibilização de conteúdos confiáveis e acessíveis.

\citeonline{freire1996} em sua abordagem pedagógica libertadora, enfatiza que o conhecimento só adquire sentido real quando é compartilhado, situado e contextualizado. Nesse sentido, a disseminação de informações em saúde --- sobretudo em áreas complexas como a acupuntura --- deve ir além do conteúdo técnico e considerar o contexto cultural, social e educacional de quem aprende. O acesso à informação, portanto, não pode depender apenas de condições econômicas, infraestrutura local ou formação prévia, pois isso aprofundaria as desigualdades já existentes entre os profissionais de saúde nas diversas regiões do país.

A inclusão digital torna-se uma aliada crucial nesse processo. Segundo \citeonline{booth2021}, tecnologias bem projetadas são capazes de reduzir barreiras estruturais no acesso à formação em saúde, desde que respeitem os diferentes níveis de letramento digital, necessidades pedagógicas e realidades locais dos usuários. Isso significa que o design de soluções digitais deve ser centrado no usuário e incluir funcionalidades como interfaces intuitivas, conteúdos multimodais (texto, áudio, imagem, vídeo), acessibilidade para pessoas com deficiência e responsividade para dispositivos móveis.

Dados do \textit{TIC Saúde 2023} revelam que mais de 65\% dos profissionais de saúde no Brasil usam \textit{smartphones} como ferramenta principal de acesso a informações clínicas \cite{ticsaude2023}. Isso reforça a urgência de produzir plataformas móveis eficazes, que funcionem bem mesmo em condições técnicas limitadas, como conexões lentas ou dispositivos com pouca capacidade de processamento.

Além disso, conforme apontam \citeonline{souza2020}, aplicativos de apoio à prática clínica devem não apenas apresentar informações confiáveis, mas também oferecer recursos de interação e personalização, que favoreçam o aprendizado ativo e a construção de confiança do profissional em sua atuação.

Dessa forma, um aplicativo como o proposto neste trabalho, que visa tornar o conhecimento da acupuntura mais acessível, confiável e intuitivo, cumpre um papel social importante: reduzir desigualdades, promover a formação segura de profissionais, e, consequentemente, aumentar a qualidade do atendimento prestado à população.

\section{TECNOLOGIAS DIGITAIS COMO APOIO À FORMAÇÃO E PRÁTICA SEGURA} \label{sec:tecnologias_digitais}

A utilização de tecnologias digitais aplicadas à saúde vem transformando não apenas os meios de acesso à informação, mas também a forma como os profissionais aplicam seu conhecimento na prática clínica. No campo da acupuntura, que exige domínio anatômico preciso e entendimento das interações energéticas do corpo humano, o uso de recursos visuais e interativos pode potencializar o aprendizado e a execução segura das técnicas.

Segundo \citeonline{moraes2020}, o uso de aplicativos, plataformas interativas e realidade aumentada na educação em saúde estimula a aprendizagem significativa, reduz o tempo de assimilação de conteúdos complexos e melhora a retenção do conhecimento. No caso da acupuntura, isso se torna ainda mais relevante ao considerar que muitos dos pontos utilizados não são visíveis a olho nu, exigindo visualização em profundidade, correlação com estruturas internas e treino constante.

Além disso, estudos como o de \citeonline{oliveira2022} destacam que alunos e profissionais da área de terapias integrativas relataram maior segurança em suas práticas após utilizar aplicativos educativos que oferecem representações tridimensionais e localização anatômica de pontos de acupuntura.

Portanto, investir no desenvolvimento de soluções digitais adaptadas à acupuntura não apenas moderniza o ensino, mas também amplia a segurança da prática clínica. Tais recursos são especialmente relevantes para estudantes em formação e profissionais em início de carreira, que frequentemente enfrentam dificuldades em aplicar corretamente os pontos e técnicas aprendidos de forma teórica.

\section{APLICATIVOS DE ACUPUNTURA: EXPERIÊNCIAS E RECURSOS TECNOLÓGICOS} \label{sec:aplicativos_acupuntura}

O desenvolvimento de tecnologias na área da saúde deve considerar mais do que apenas conteúdo técnico. A forma como o usuário interage com a interface, compreende as informações e se orienta dentro do aplicativo é determinante para a sua eficácia. No contexto da acupuntura --- que envolve a localização precisa de pontos no corpo, o entendimento de trajetos de meridianos e a aplicação de protocolos --- a usabilidade se torna um pilar essencial.

De acordo com \citeonline{norman2013}, um bom design deve ``tornar possível o uso intuitivo'', ou seja, permitir que o usuário compreenda e utilize o sistema sem esforço cognitivo excessivo. Em aplicativos de saúde, isso significa que menus confusos, excesso de informações, termos técnicos não explicados ou falta de \textit{feedback} visual comprometem a experiência do usuário e até mesmo colocam a segurança da prática em risco.

A Organização Mundial da Saúde \cite{oms2020} recomenda que soluções tecnológicas em saúde digital adotem princípios de design centrado no usuário (DCU), considerando desde o início do desenvolvimento as reais necessidades, limitações e habilidades dos usuários. Isso é especialmente importante no caso de profissionais que atuam em contextos com restrições de tempo, estrutura ou formação técnica, como muitos terapeutas integrativos ou estudantes de cursos livres.

Além disso, \citeonline{ferreira2021} ressaltam que aplicativos com interfaces acessíveis e organizadas promovem maior retenção do conhecimento e reduzem erros clínicos, especialmente em áreas que exigem visualização corporal anatômica como a fisioterapia e a acupuntura.

Por isso, o aplicativo proposto neste TCC prioriza:

\begin{itemize}
    \item Interfaces limpas e intuitivas;
    \item Navegação fluida e segmentada por áreas do corpo;
    \item \textit{Feedback} visual ao selecionar pontos de acupuntura;
    \item Compatibilidade com dispositivos móveis de diferentes capacidades;
    \item Inclusão de recursos de busca inteligente, filtros por sintomas e suporte ao idioma técnico com explicações simplificadas.
\end{itemize}

Esses elementos reforçam o compromisso com a acessibilidade e a segurança, pilares fundamentais quando se pensa em soluções tecnológicas que impactam diretamente a formação e atuação clínica dos profissionais da saúde.

\section{COMPARATIVO COM APLICATIVOS EXISTENTES} \label{sec:comparativo}

Atualmente, existem diversos aplicativos disponíveis na Play Store que buscam auxiliar estudantes e profissionais da área da acupuntura por meio de recursos visuais e informativos. No entanto, ao analisar criticamente essas soluções, é possível perceber limitações significativas que comprometem sua eficácia como ferramentas de apoio à prática clínica e ao aprendizado, especialmente no contexto brasileiro.

O \textit{Anatomy Learning}, por exemplo, é um aplicativo renomado para o estudo da anatomia em 3D. Sua qualidade gráfica e riqueza de detalhes são inegáveis, porém o aplicativo não possui foco específico em acupuntura, o que o torna pouco funcional para quem busca compreender os pontos e os meridianos energéticos. Além disso, seu conteúdo é voltado a usuários com formação técnica mais avançada, o que pode representar uma barreira para estudantes de cursos livres ou profissionais de áreas menos técnicas. Outro ponto problemático é seu desempenho em dispositivos de menor capacidade, com relatos de travamentos e lentidão.

Já o \textit{Visual Acupuncture 3D} apresenta um escopo mais próximo ao esperado, oferecendo visualização de pontos e meridianos. No entanto, peca em aspectos de usabilidade. A interface é considerada pouco intuitiva por muitos usuários, com menus confusos, poluição visual e falta de clareza nos textos. Esses fatores dificultam seu uso durante a prática clínica, especialmente em momentos em que o profissional precisa localizar rapidamente um ponto ou revisar um protocolo. Além disso, a ausência de suporte ao idioma português reduz sua acessibilidade para profissionais brasileiros.

Outro aplicativo conhecido é o \textit{Acupuncture 3D}, que também disponibiliza visualizações anatômicas tridimensionais focadas em acupuntura. Entretanto, o aplicativo carece de atualizações e apresenta conteúdos desatualizados em relação às nomenclaturas e práticas adotadas no Brasil. Sua interface é rígida e limitada, com dificuldade de navegação e ausência de funcionalidades complementares como busca por sintomas, indicação de protocolos ou explicações detalhadas para aplicação dos pontos.

Em contraste com essas soluções, o aplicativo proposto neste TCC se destaca por adotar uma abordagem centrada no usuário, com interface limpa, navegação fluida e foco na aplicação prática da acupuntura em contexto clínico. Ele foi pensado para funcionar bem mesmo em dispositivos mais simples, garantindo acessibilidade digital a um público mais amplo. Além disso, traz conteúdo em português, explicações didáticas e filtros de busca por sintomas, aproximando-se mais das reais necessidades dos profissionais e estudantes brasileiros da área.

Essa comparação evidencia que, embora os aplicativos analisados ofereçam contribuições pontuais, nenhum deles contempla simultaneamente os pilares de acessibilidade, usabilidade, confiabilidade e contexto nacional. O desenvolvimento do aplicativo proposto surge, assim, como uma resposta direta às lacunas existentes no mercado digital de apoio à acupuntura, promovendo inclusão, segurança e aprimoramento profissional.

\section{USABILIDADE E DESIGN CENTRADO NO USUÁRIO} \label{sec:usabilidade}

Para que a tecnologia seja de fato acessível e eficaz, é essencial que siga princípios de usabilidade e design centrado no usuário. \citeonline{nielsen1994} define usabilidade como a capacidade de um sistema ser usado por seus usuários com eficiência, efetividade e satisfação. Esse conceito é especialmente importante em contextos de formação em saúde, onde a curva de aprendizado precisa ser suavizada.

Segundo \citeonline{siqueira2021}, aplicações móveis voltadas para profissionais da saúde que incorporam princípios de design centrado no usuário têm maiores taxas de adesão e melhor impacto educacional. A adaptação do conteúdo conforme o perfil do usuário, como proposto no \textit{Appunture}, está em consonância com o modelo de usabilidade responsiva, recomendado por padrões como a norma ISO/IEC 25010.

\section{NORMAS, DIRETRIZES E EVIDÊNCIAS CIENTÍFICAS} \label{sec:normas}

O desenvolvimento de soluções tecnológicas voltadas à saúde deve considerar diretrizes oficiais e evidências científicas. A OMS \cite{oms1999} publicou um guia para treinamento e segurança em acupuntura, enfatizando a necessidade de padronização de pontos, técnicas de punção e biossegurança.

Na área de software, normas como a ISO 9241-210 e a ISO/IEC 25010 orientam o desenvolvimento de interfaces acessíveis, seguras e eficientes, especialmente em sistemas críticos como os de saúde. A adoção dessas diretrizes no \textit{Appunture} garante não apenas a funcionalidade, mas também a segurança e a confiabilidade da ferramenta.
