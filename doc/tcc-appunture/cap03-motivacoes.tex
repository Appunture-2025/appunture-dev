% ----------------------------------------------------------
% Capítulo 3 - MOTIVAÇÕES
% ----------------------------------------------------------
\chapter{MOTIVAÇÕES} \label{cha:motivacoes}

O desenvolvimento do aplicativo \textit{Appunture} surge a partir de uma série de desafios estruturais enfrentados pela prática da acupuntura no Brasil. Estes desafios envolvem a regulamentação da profissão, a formação heterogênea dos profissionais, a fragilidade nos processos de fiscalização, os riscos à biossegurança e a complexidade diagnóstica da Medicina Tradicional Chinesa (MTC). A seguir, são apresentados os principais fatores motivadores que embasam a proposta deste projeto.

\section{REGULAMENTAÇÃO E DISPUTA DE CATEGORIAS PROFISSIONAIS} \label{sec:regulamentacao}

A regulamentação da acupuntura no Brasil é um dos maiores entraves para sua prática estruturada e segura. Apesar de sua origem milenar na MTC e da crescente popularização a partir das décadas finais do século XX, ainda não há consenso nacional sobre quais categorias profissionais estão habilitadas a aplicar acupuntura. Isso tem gerado disputas entre conselhos de diferentes áreas, como medicina, fisioterapia, enfermagem, psicologia, odontologia e farmácia.

Enquanto o Conselho Federal de Medicina (CFM) defende a acupuntura como especialidade médica, conselhos de outras áreas reconhecem a prática multiprofissional e regulamentam internamente sua atuação. Essa disputa tem gerado insegurança jurídica e limitações à atuação de profissionais devidamente capacitados, ao mesmo tempo em que abre espaço para práticas irregulares.

A proposta do Projeto de Lei 5983/2019, atualmente em tramitação no Senado Federal, busca criar critérios nacionais para o exercício da acupuntura, independentemente da formação de base, desde que haja capacitação específica. Enquanto isso, soluções tecnológicas devem ser sensíveis a esse cenário, respeitando os limites legais e adaptando os conteúdos conforme o perfil profissional do usuário.

\section{FORMAÇÃO E FISCALIZAÇÃO INSUFICIENTES} \label{sec:formacao}

A ausência de um conselho exclusivo e de uma legislação unificada gera uma lacuna na fiscalização da prática da acupuntura no Brasil. Muitos profissionais atuam com formações diversas, que vão desde cursos livres até especializações, sem uma padronização mínima nacional. Além disso, terapeutas que não estão vinculados a conselhos de classe podem atuar sem supervisão ou responsabilização adequada.

Nesse contexto, um aplicativo como o \textit{Appunture} pode oferecer suporte educativo e técnico, contribuindo para o aperfeiçoamento de profissionais com diferentes formações. Por meio da diferenciação de perfis de usuários e de um sistema de conteúdos escalonados conforme a formação declarada, o aplicativo propõe uma experiência segura, ética e ajustada à realidade do usuário.

\section{SEGURANÇA DO PACIENTE E BIOSSEGURANÇA} \label{sec:seguranca}

A acupuntura, por envolver a inserção de agulhas no corpo humano, é considerada uma prática invasiva e requer protocolos rigorosos de biossegurança. A utilização inadequada de instrumentos ou a realização do procedimento por pessoas sem conhecimento técnico podem gerar complicações sérias, como infecções, lesões internas, hematomas e até pneumotórax.

Casos documentados, como o de uma paciente hospitalizada após perfuração pulmonar em São Paulo, demonstram os riscos reais da má prática. Muitos desses casos ocorrem em ambientes informais ou com baixa qualificação técnica.

A tecnologia pode ser uma aliada na disseminação de boas práticas. O aplicativo \textit{Appunture} fornece informações claras e baseadas em diretrizes de segurança da OMS, orientando sobre profundidade de pontuação, indicações, contraindicações e características especiais dos pontos, contribuindo para uma atuação mais segura e responsável.

\section{DIAGNÓSTICO ENERGÉTICO: COMPLEXIDADE E SUBJETIVIDADE} \label{sec:diagnostico}

O raciocínio clínico na acupuntura, segundo os princípios da MTC, exige uma abordagem subjetiva e holística, muitas vezes desafiadora para iniciantes. Diagnósticos são baseados em padrões energéticos, observações da língua e do pulso, e uma análise minuciosa do estado geral do paciente.

Diferentemente da medicina ocidental, não se trata apenas de associar um sintoma a uma causa direta, mas de identificar síndromes energéticas como estagnação de Qi, excesso de calor, umidade interna, entre outras. O mesmo sintoma, como dor abdominal, pode ter interpretações energéticas distintas, dependendo do conjunto de sinais apresentados.

Nesse sentido, o aplicativo \textit{Appunture} atua como um apoio ao raciocínio clínico, fornecendo descrições de padrões, relações entre sintomas e síndromes e, inclusive, oferecendo um sistema de busca inteligente por sintomas com suporte de NLP (Processamento de Linguagem Natural), que interpreta a linguagem do usuário e sugere os pontos mais indicados para o caso apresentado.

Diante dos desafios regulatórios, formativos e clínicos que permeiam a prática da acupuntura no Brasil, torna-se essencial o desenvolvimento de ferramentas tecnológicas que aliem conhecimento técnico, segurança e acessibilidade. O \textit{Appunture} surge como resposta a essas motivações, buscando oferecer um recurso digital estratégico para fortalecer a prática da acupuntura com ética, qualidade e base científica.
