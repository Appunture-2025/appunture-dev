%%%%%%%%%%%%%%%%%%%%%%%%%%%%%%%%%%%%%%%%%%%%%%%%%%%%%%%
% Arquivo para entrada de dados para a parte pré textual
% TCC APPUNTURE - UFPR 2025
%%%%%%%%%%%%%%%%%%%%%%%%%%%%%%%%%%%%%%%%%%%%%%%%%%%%%%%

% Informações de dados para CAPA e FOLHA DE ROSTO
%----------------------------------------------------------------------------- 
\tipotrabalho{Trabalho de Conclusão de Curso}

% Marcar Sim para as partes que irão compor o documento pdf
%----------------------------------------------------------------------------- 
\providecommand{\terCapa}{Sim}
\providecommand{\terFolhaRosto}{Sim}
\providecommand{\terTermoAprovacao}{Sim}
\providecommand{\terDedicatoria}{Nao}
\providecommand{\terFichaCatalografica}{Nao}
\providecommand{\terEpigrafe}{Nao}
\providecommand{\terAgradecimentos}{Sim}
\providecommand{\terErrata}{Nao}
\providecommand{\terListaFiguras}{Sim}
\providecommand{\terListaQuadros}{Sim}
\providecommand{\terListaTabelas}{Sim}
\providecommand{\terSiglasAbrev}{Sim} 
\providecommand{\terSimbolos}{Nao}
\providecommand{\terResumos}{Sim}
\providecommand{\terSumario}{Sim}
\providecommand{\terAnexo}{Nao}
\providecommand{\terApendice}{Sim}
\providecommand{\terIndiceR}{Nao}
%----------------------------------------------------------------------------- 

\titulo{APPUNTURE: Sistema Multiplataforma de Suporte à Acupuntura}

% Para múltiplos autores, use o comando autor com todos os nomes
\autor{Bruno Brugnerotto de Lara \and Gabriel Francelino Voidaleski \and Pedro Henrique Lopes}

\local{Curitiba}
\data{2025}

% Orientador ou Orientadora
\orientador{Prof. Dr. Paulo Sobreira Moraes}
\orientadora{}

% Coorientador ou Coorientadora (se houver)
\coorientador{}
\coorientadora{}

% Segundo Coorientador ou Segunda Coorientadora
\scoorientador{}
\scoorientadora{}

% ----------------------------------------------------------
\addbibresource{referencias.bib}

% ----------------------------------------------------------
\instituicao{Universidade Federal do Paraná}

\def \ImprimirSetor{Setor de Ciências Exatas}

\def \ImprimirProgramaPos{}

\def \ImprimirCurso{Curso de Tecnologia em Análise e Desenvolvimento de Sistemas}

\preambulo{Trabalho de Conclusão de Curso apresentado como requisito parcial à obtenção do grau de Tecnólogo em Análise e Desenvolvimento de Sistemas, Setor de Ciências Exatas da Universidade Federal do Paraná.}

%----------------------------------------------------------------------------- 

\newcommand{\imprimirCurso}{Tecnologia em Análise e Desenvolvimento de Sistemas}

\newcommand{\imprimirDataDefesa}{Dezembro de 2025}

\newcommand{\imprimircdu}{004.5:615.814.1}

% ----------------------------------------------------------
% Errata (se necessário)
\newcommand{\imprimirerrata}{}

% ----------------------------------------------------------
% Assinaturas do Termo de Aprovação
\newcommand{\AssinaAprovacao}{
   \assinatura{Prof. Dr. Paulo Sobreira Moraes \\ Orientador - UFPR}
   \assinatura{Professor(a) Avaliador(a) 1 \\ UFPR}
   \assinatura{Professor(a) Avaliador(a) 2 \\ UFPR}
      
   \begin{center}
    \vspace*{0.5cm}
    \imprimirlocal, \imprimirDataDefesa.
    \vspace*{1cm}
  \end{center}
}

% ----------------------------------------------------------
% Epígrafe
\newcommand{\EpigrafeTexto}{
\textit{``A jornada de mil milhas começa com um único passo.''}\\
(Lao Tzu)
}

% ----------------------------------------------------------
% RESUMO em Português
\newcommand{\ResumoTexto}{
O aplicativo \textit{Appunture} tem como objetivo oferecer suporte técnico, educativo e visual para profissionais e estudantes da área de acupuntura, promovendo uma prática mais segura, padronizada e alinhada às exigências legais e sanitárias brasileiras. Em um cenário marcado pela ausência de regulamentação federal clara, disputas entre conselhos profissionais e formação heterogênea, o \textit{Appunture} surge como uma solução inovadora que facilita o acesso a informações de qualidade sobre os pontos de acupuntura corporal.

A plataforma foi desenvolvida para Android, iOS e Web, apresentando um modelo anatômico interativo baseado em um atlas de 15 visualizações vetoriais (SVG) de alta fidelidade, organizadas por meridianos e vistas (frontal e posterior). O sistema conta com funcionalidades de busca avançada, detalhes técnicos de 361 pontos, e um assistente inteligente integrado que utiliza Inteligência Artificial Generativa para correlacionar sintomas e sugerir protocolos baseados em evidências.

A prática da acupuntura envolve desafios complexos, como a subjetividade diagnóstica da Medicina Tradicional Chinesa (MTC), a necessidade de protocolos rigorosos de biossegurança e a fiscalização insuficiente. O \textit{Appunture} contribui diretamente nesses pontos ao fornecer informações padronizadas, baseadas em diretrizes da Organização Mundial da Saúde (OMS).

O desenvolvimento seguiu uma arquitetura híbrida com estratégia ``offline-first'', garantindo acesso total aos dados mesmo sem conexão à internet, com sincronização automática quando online. Para garantir a precisão dos dados anatômicos, foi desenvolvida uma ferramenta proprietária de mapeamento de pontos, assegurando a exatidão das coordenadas em relação às estruturas corporais.
} 

\newcommand{\PalavraschaveTexto}{Acupuntura; Appunture; MTC; Práticas Integrativas; Saúde; Biossegurança; Inteligência Artificial; Spring AI.}

% ----------------------------------------------------------
% ABSTRACT em Inglês
\newcommand{\AbstractTexto}{
The \textit{Appunture} application aims to provide technical, educational, and visual support for professionals and students in the field of acupuncture. The platform features an interactive anatomical model based on a high-fidelity atlas of 15 vector views (SVG), organized by meridians. The system includes advanced search functionalities, technical details of 361 points, and an intelligent assistant powered by Generative AI (Spring AI + Google Gemini) to correlate symptoms and suggest evidence-based protocols.

Developed with a hybrid ``offline-first'' architecture using React Native and Java Spring Boot, the application ensures full data access even without internet connection, with automatic synchronization. To ensure anatomical accuracy, a proprietary point mapping tool was developed to precisely locate acupuncture points on the digital atlas.
}

\newcommand{\KeywordsTexto}{Acupuncture; Appunture; TCM; Integrative Health; Biosafety; Artificial Intelligence; Spring AI.}

% ----------------------------------------------------------
% Resumo em Francês (opcional)
\newcommand{\Resume}{}
\newcommand{\Motscles}{}

% ----------------------------------------------------------
% Resumo em Espanhol (opcional)
\newcommand{\Resumen}{}
\newcommand{\Palabrasclave}{}

% ----------------------------------------------------------
% Agradecimentos
\newcommand{\AgradecimentosTexto}{
Agradecemos primeiramente a Deus por nos dar força e sabedoria para concluir este trabalho.

Ao nosso orientador, Prof. Dr. Paulo Sobreira Moraes, pela orientação, paciência e dedicação durante todo o desenvolvimento deste projeto.

À Universidade Federal do Paraná, pela oportunidade de formação e pelo ambiente propício ao aprendizado e à pesquisa.

Aos nossos familiares e amigos, pelo apoio incondicional e compreensão nos momentos de ausência.

A todos os profissionais da área de acupuntura que contribuíram com suas experiências e conhecimentos para a construção deste aplicativo.

Por fim, agradecemos a todos que, direta ou indiretamente, contribuíram para a realização deste trabalho.
}

% ----------------------------------------------------------
% Dedicatória
\newcommand{\DedicatoriaTexto}{
\textit{Dedicamos este trabalho a todos os profissionais\\
da saúde que buscam constantemente aprimorar\\
seus conhecimentos em prol do bem-estar dos pacientes.}
}

% ----------------------------------------------------------
% Lista de Siglas e Abreviaturas
% Utilizar no texto: \sigla{MTC}{Medicina Tradicional Chinesa}
% ----------------------------------------------------------

% ----------------------------------------------------------
% Lista de Símbolos
% ----------------------------------------------------------
