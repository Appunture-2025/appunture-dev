%%%%%%%%%%%%%%%%%%%%%%%%%%%%%%%%%%%%%%%%%%%%%%%%%%%%%%%
% Arquivo para entrada de dados para a parte pré textual
% TCC APPUNTURE - UFPR 2025
%%%%%%%%%%%%%%%%%%%%%%%%%%%%%%%%%%%%%%%%%%%%%%%%%%%%%%%

% Informações de dados para CAPA e FOLHA DE ROSTO
%----------------------------------------------------------------------------- 
\tipotrabalho{Trabalho de Conclusão de Curso}

% Marcar Sim para as partes que irão compor o documento pdf
%----------------------------------------------------------------------------- 
\providecommand{\terCapa}{Sim}
\providecommand{\terFolhaRosto}{Sim}
\providecommand{\terTermoAprovacao}{Nao}
\providecommand{\terDedicatoria}{Nao}
\providecommand{\terFichaCatalografica}{Nao}
\providecommand{\terEpigrafe}{Nao}
\providecommand{\terAgradecimentos}{Nao}
\providecommand{\terErrata}{Nao}
\providecommand{\terListaFiguras}{Sim}
\providecommand{\terListaQuadros}{Sim}
\providecommand{\terListaTabelas}{Sim}
\providecommand{\terSiglasAbrev}{Sim} 
\providecommand{\terSimbolos}{Nao}
\providecommand{\terResumos}{Sim}
\providecommand{\terSumario}{Sim}
\providecommand{\terAnexo}{Nao}
\providecommand{\terApendice}{Sim}
\providecommand{\terIndiceR}{Nao}
%----------------------------------------------------------------------------- 

\titulo{APPUNTURE: Atlas Digital Educativo de Acupuntura}

% Para múltiplos autores, use o comando autor com todos os nomes
\autor{Bruno Brugnerotto de Lara \and Gabriel Francelino Voidaleski \and Pedro Henrique Lopes}

\local{Curitiba}
\data{2025}

% Orientador ou Orientadora
\orientador{Prof. Dr. Paulo Sobreira Moraes}
\orientadora{}

% Coorientador ou Coorientadora (se houver)
\coorientador{}
\coorientadora{}

% Segundo Coorientador ou Segunda Coorientadora
\scoorientador{}
\scoorientadora{}

% ----------------------------------------------------------
% Arquivo de referências (usado pelo abntex2cite com bibtex)
% O comando \bibliography{referencias} está no main.tex

% ----------------------------------------------------------
\instituicao{Universidade Federal do Paraná}

\def \ImprimirSetor{Setor de Educação Profissional e Tecnológica}

\def \ImprimirProgramaPos{}

\def \ImprimirCurso{Curso de Tecnologia em Análise e Desenvolvimento de Sistemas}

\preambulo{Trabalho de Conclusão de Curso apresentado como requisito parcial à obtenção do grau de Tecnólogo em Análise e Desenvolvimento de Sistemas, Setor de Ciências Exatas da Universidade Federal do Paraná.}

%----------------------------------------------------------------------------- 

\newcommand{\imprimirCurso}{Tecnologia em Análise e Desenvolvimento de Sistemas}

\newcommand{\imprimirDataDefesa}{Dezembro de 2025}

\newcommand{\imprimircdu}{004.5:615.814.1}

% ----------------------------------------------------------
% Errata (se necessário)
\newcommand{\imprimirerrata}{}

% ----------------------------------------------------------
% Assinaturas do Termo de Aprovação
%\newcommand{\AssinaAprovacao}{
 %  \assinatura{Prof. Dr. Paulo Sobreira Moraes \\ Orientador - UFPR}
  % \assinatura{Professor(a) Avaliador(a) 1 \\ UFPR}
   %\assinatura{Professor(a) Avaliador(a) 2 \\ UFPR}
    %  
   %\begin{center}
    %\vspace*{0.5cm}
    %\imprimirlocal, \imprimirDataDefesa.
    %\vspace*{1cm}
  %\end{center}
%}

% ----------------------------------------------------------
% Epígrafe
% \newcommand{\EpigrafeTexto}{
% \textit{``A jornada de mil milhas começa com um único passo.''}\\
% (Lao Tzu)
% }

% ----------------------------------------------------------
% RESUMO em Português
\newcommand{\ResumoTexto}{
O \textit{Appunture} é um atlas digital educativo voltado a estudantes e profissionais interessados em acupuntura, com o objetivo de facilitar o acesso a informações sobre os pontos de acupuntura corporal. Em um cenário onde a formação na área é heterogênea e o acesso a materiais de qualidade pode ser limitado, a ferramenta busca auxiliar no processo de aprendizagem e consulta, disponibilizando um atlas anatômico digital organizado por meridianos.

A plataforma foi desenvolvida para Android, apresentando 15 visualizações vetoriais (SVG) organizadas por meridianos e vistas (frontal e posterior). O sistema oferece funcionalidades de busca, detalhes dos 361 pontos principais, e um assistente baseado em Inteligência Artificial para auxiliar na correlação entre sintomas e pontos.

O desenvolvimento seguiu uma arquitetura ``offline-first'', permitindo o acesso aos dados mesmo sem conexão à internet. Para auxiliar no mapeamento dos pontos sobre as imagens anatômicas, foi desenvolvida uma ferramenta interna de posicionamento de coordenadas.

É importante ressaltar que o \textit{Appunture} não substitui a formação profissional adequada nem o acompanhamento de profissionais qualificados. A ferramenta foi concebida exclusivamente como um recurso educativo complementar de apoio ao estudo e à consulta de informações sobre acupuntura, não devendo ser utilizada para fins de diagnóstico, prescrição ou tratamento.
} 

\newcommand{\PalavraschaveTexto}{Acupuntura; Atlas Digital; Aplicativo Educativo; Medicina Tradicional Chinesa; Tecnologia Educacional.}

% ----------------------------------------------------------
% ABSTRACT em Inglês
\newcommand{\AbstractTexto}{
\textit{Appunture} is an educational digital atlas designed to facilitate access to information about acupuncture points for students and professionals interested in the field. The Android application features a digital anatomical atlas with 15 vector views (SVG), organized by meridians. The system includes search functionalities, details of 361 main points, and an AI-based assistant to help correlate symptoms and points.

Developed with an ``offline-first'' architecture using React Native and Java Spring Boot, the application allows data access even without internet connection. An internal mapping tool was developed to assist in positioning point coordinates on the anatomical images.

It is important to note that \textit{Appunture} does not replace proper professional training. The tool was designed exclusively as a complementary educational resource to support learning and information lookup about acupuncture, and should not be used for diagnosis, prescription, or treatment purposes.
}

\newcommand{\KeywordsTexto}{Acupuncture; Digital Atlas; Educational Application; Traditional Chinese Medicine; Educational Technology.}

% Resumos em outros idiomas (francês e espanhol) - deixar vazio se não usar
\newcommand{\Resume}{}
\newcommand{\Motscles}{}
\newcommand{\Resumen}{}
\newcommand{\Palabrasclave}{}

% ----------------------------------------------------------
% Agradecimentos
%\newcommand{\AgradecimentosTexto}{
%Agradecemos primeiramente a Deus por nos dar força e sabedoria para concluir este trabalho.

%Ao nosso orientador, Prof. Dr. Paulo Sobreira Moraes, pela orientação, paciência e dedicação durante todo o desenvolvimento deste projeto.

%À Universidade Federal do Paraná, pela oportunidade de formação e pelo ambiente propício ao aprendizado e à pesquisa.

%Aos nossos familiares e amigos, pelo apoio incondicional e compreensão nos momentos de ausência.

%A todos os profissionais da área de acupuntura que contribuíram com suas experiências e conhecimentos para a construção deste aplicativo.

%Por fim, agradecemos a todos que, direta ou indiretamente, contribuíram para a realização deste trabalho.
%}

% ----------------------------------------------------------
% Dedicatória
\newcommand{\DedicatoriaTexto}{
\textit{Dedicamos este trabalho a todos os estudantes e\\profissionais que buscam constantemente aprimorar\\seus conhecimentos na área da acupuntura.}
}

% ----------------------------------------------------------
% Lista de Siglas e Abreviaturas
% Utilizar no texto: \sigla{MTC}{Medicina Tradicional Chinesa}
% ----------------------------------------------------------
\criarsigla{API}{Application Programming Interface}
\criarsigla{JWT}{JSON Web Token}
\criarsigla{LLM}{Large Language Model}
\criarsigla{MMKV}{Memory Mapped Key-Value}
\criarsigla{MTC}{Medicina Tradicional Chinesa}
\criarsigla{OMS}{Organização Mundial da Saúde}
\criarsigla{RAG}{Retrieval-Augmented Generation}
\criarsigla{RBAC}{Role-Based Access Control}
\criarsigla{SDK}{Software Development Kit}
\criarsigla{SQL}{Structured Query Language}
\criarsigla{SVG}{Scalable Vector Graphics}
\criarsigla{UML}{Unified Modeling Language}

% ----------------------------------------------------------
% Lista de Símbolos
% ----------------------------------------------------------
